\documentclass[11pt,pdftex,dvipsnames,usenames,helvetica]{beamer}
\setbeameroption{show notes}
\usepackage[round]{natbib}
\usepackage{textcomp}
\usepackage{amsmath}
\usepackage{amssymb}
\usepackage{graphicx}
\DeclareGraphicsExtensions{.ps,.eps,.pdf,.jpg,.png}
\usefonttheme[onlymath]{serif}
\usepackage[cmintegrals,cmbraces]{newtxmath}
\usetheme{default}
\usecolortheme{dove}

\usepackage{cancel}
\usepackage{tikz}
\usepackage[english]{babel}

\usepackage{verbatim}
\usepackage{color}
\usepackage{pgf} %portable graphics format
\usepackage[autobold]{statex2}
\mode<presentation>
{
  %\usetheme{Warsaw}
  % or ...
  \setbeamercovered{transparent}
  % or whatever (possibly just delete it)

  \setbeamertemplate{navigation symbols}{}
  \usefonttheme[onlysmall]{structurebold}
  %\usefonttheme{structurebold}
}
\addtobeamertemplate{navigation symbols}{}{%
    \usebeamerfont{footline}%
  \setbeamertemplate{navigation symbols}{}
    \usebeamercolor[fg]{footline}%
    \hspace{1em}%
    \insertframenumber/\inserttotalframenumber
}
\newcommand*{\red}[1]{\textcolor{red}{#1}}
\newcommand*{\blue}[1]{\textcolor{blue}{#1}}

\begin{document}
\boldmath
% 0. Intro

\begin{frame}
\frametitle{Graduate School Class Reminders}

\begin{itemize}
% \item Keep your facial covering on, including over your nose
\item Maintain six feet of distancing
\item Please sit in the same chair each class time
\item Observe entry/exit doors as marked
\item Use hand sanitizer when you enter/exit the classroom
\item Use a disinfectant wipe/spray to wipe down your learning space
  before and after class
\item Media Services: 414 955-4357 option 2
%\item START RECORDING
\end{itemize}

\end{frame}

\begin{frame}
\frametitle{Documentation on the web}

\begin{itemize}
\item CRAN: \url{http://cran.r-project.org}
\item R manuals: \url{https://cran.r-project.org/manuals.html}
\item SAS: \url{http://support.sas.com/documentation}
\item 
\textcolor{blue}{SAS 9.3: \url{https://support.sas.com/en/documentation/documentation-for-SAS-93-and-earlier.html}}
\item Step-by-Step Programming with Base SAS 9.4 (SbS): \\
\url{https://documentation.sas.com/api/docsets/basess/9.4/content/basess.pdf}
\item SAS 9.4 Programmer s Guide: Essentials (PGE): \\
\url{https://documentation.sas.com/api/docsets/lepg/9.4/content/lepg.pdf}
\item Wiki: \url{https://wiki.biostat.mcw.edu} 
\textcolor{red}{(MCW/VPN)}
\end{itemize}

\end{frame}

\begin{frame}[fragile]
\frametitle{Hands-on HW: reading data from multiple files of the PTB}
\begin{itemize}
\item As we have seen, the ECG information requires reading in multiple files
\item A header file, {\tt .hea}, a data file for the 12 standard
  leads, {\tt .dat}, and a data file for the 3 Frank leads, {\tt .xyz}
\item Let's re-visit: why should we use the {\tt IBw.d} informat?
\item Create your own SAS library in the directory {\tt
    $~$/libname/ecg} and read in all of the ECGs for controls into the
  permanent SAS data set {\tt ecg.controls}
\item And utilize your two's complement correction\\
how many ECGs are there for which the sum total is incorrect?
\item see the example for the cases: {\tt PTB/cases.sas}
\item BEWARE: the cases data set is quite large with almost 60M observations
yet still surprisingly fast: run-time is about \\ 5 minutes!
\item The controls data set will be smaller, but if you have a mistake
in your SAS code,
then the error messages can create an ENORMOUS {\tt .log} file
\end{itemize}

\end{frame}

\begin{frame}[fragile]
\frametitle{SAS and Emacs/ESS[SAS]}
\begin{itemize}
\item Let's take a look at {\tt autoexec.sas}\\
{\tt cp $~$rsparapa/autoexec.sas $~$}
\item N.B.\ similarly, you have to copy {\tt .sas} files
to your home directory to edit/submit them (just like {\tt .R} files)
\item Submit {\tt autoexec} with {\tt F3}, goto the {\tt .log} with
  {\tt F5}\\ return with {\tt F4} and goto the {\tt *shell*} 
buffer with {\tt F8}
\item Test of ESS[SAS] function keys%: {\tt F9} and {\tt import.sas}
\item see {\tt PTB/ecg.sas} and {\tt PTB/acf.sas}
\item {\tt F9} opens a SAS data set for viewing
\item {\tt F12} opens a graphics file for viewing
\item {\tt C-F10} makes the font smaller
\item {\tt C-F11} makes the font larger
\item {\tt TAB} and {\tt C-TAB}
\item Emacs and {\tt tabbar}
\end{itemize}

\end{frame}

\begin{frame}[fragile]
\frametitle{DATASTEP STATEMENTS}
\begin{itemize}
\item {\tt data NEW; set OLD; STATEMENT; ... run;}\\
the impact of {\tt STATEMENT} is on the creation of {\tt NEW}\\
 {\tt OLD} is unchanged
\item {\tt drop VAR1 ... VARn;}\\
  drop these variables from the data set\\
  if the variable names actually end in numbers, then you can use a
  {\it variable list}
  which can be used in many places\\
  \textcolor{red}{\tt drop VAR1-VARn;}
\item {\tt keep VAR1 VAR2 ... VARn;}\\ 
keep ONLY these variables in the data set
\item {\tt rename OLD1=NEW1 ... OLDn=NEWn;}\\
rename these variables in the data set\\
if you can use a variable list\\
{\tt rename OLD1-OLDn=NEW1-NEWn;}
\item {\tt where CONDITION;}\\
limit to observations from {\tt OLD} satisfying {\tt CONDITION}\\
many PROCS also accept a {\tt where} statement\\
\textcolor{blue}{\tt proc NAME data=OLD; where CONDITION; ...\ run;}
\end{itemize}
\end{frame}

\begin{frame}[fragile]
\frametitle{SAS NUMBERS, CHARACTERS and CONDITIONS}
\begin{itemize}
\item SAS makes no distinction for integers\\ 
every number is double-precision floating point\\
which obviously can represent integers as well\\
SAS has no concept of {\tt NaN} or {\tt Inf}\\
Division by zero results in a missing value
\item if you attempt to perform arithmetic with a character expression,
SAS will attempt to convert it to number
\item if a CONDITION is simply a number or a numeric variable, then it is TRUE
UNLESS the number is 0 \textcolor{blue}{or missing!}\\
{\tt x=.;} is SAS' missing value for numeric\\
Also \textcolor{red}{\tt .a ...\ .z .\_}\\
{\tt y=" ";} is SAS' missing value for character
\item 
a CONDITION resulting in a character expression\\ 
(that can't be converted into a numeric expression)\\ is an error
\item So these are all TRUE:
{\tt 1, "1", 2.1}
\item And these are all FALSE:
{\tt 0, "0", ., ".", " "}

\end{itemize}
\end{frame}

\begin{frame}[fragile]
\frametitle{SAS NUMBERS, CHARACTERS and CONDITIONS}
\begin{itemize}
\item Typically, a CONDITION is a comparison between two variables
  (or between a variable and a literal)\\
  \textcolor{blue}{\tt =} or {\tt EQ}; {\tt $<$} or {\tt LT}; {\tt
    $<$=} or {\tt LE}; {\tt $>$} or {\tt GT}; {\tt $>$=} or {\tt GE};
  and {\tt \textasciicircum=} or {\tt NE} % for NOT EQUAL
\item CONDITIONS can be chained together\\
 by either {\tt \&} 
  \textcolor{red}{\tt and} for AND; either {\tt |}
  \textcolor{blue}{\tt or} for OR
\item you can also chain together interval CONDITIONS like so\\
{\tt 3<x<=6} for {\tt x} in the interval (3, 6]
\item you can specify a subset by {\tt x in(3, 4, 5)}
\item you can compare a variable to missing (unlike R): {\tt x=.}\\
also {\tt n(ARG1, ..., ARGn)} and {\tt nmiss(ARG1, ..., ARGn)} 
\item \textcolor{red}{BEWARE:  {\tt .<x} is TRUE unless {\tt x} is also 
missing}\\
\textcolor{blue}{In comparisons, missing behaves like negative infinity!}
\item you can specify STARTS-WITH by the colon operator\\ {\tt y=:"a"}
  which is TRUE for all character strings starting with {\tt a}
\item alphabetic comparisons are based on the order of ASCII\\
"A"$<$"Z" is TRUE and " "$<$"A" is also TRUE\\
ASCII collating sequence on SbS pg. 204
\end{itemize}
\end{frame}

\begin{frame}[fragile]
\frametitle{DATASTEP STATEMENTS: {\tt length}}
\begin{itemize}
\item The {\tt length} statement can be used for both numeric and
  character variables\\

\item BEWARE: it is risky to reduce the length of numeric variables\\
do NOT lessen the length below the default which is 8\\
can create errors which are VERY difficult to debug
since no error message is generated\\
to set, you would do: {\tt length NUMBER 8;}
\item The default length for character variables is also 8
\item Typically, longer character variables are either 40 or 200 characters\\
  {\tt length SHORT \$ 40 LONG \$ 200;}
\item if you need to change the length of a variable already created\\
then place the {\tt length} statement BEFORE the {\tt set} statement\\
\textcolor{blue}{\tt data NEW; length VAR \$ 200; set OLD; ...}\\
\end{itemize}
\end{frame}

\begin{frame}
\frametitle{DATASTEP STATEMENTS: {\tt if-then-else}\\ 
\textcolor{red}{RED} and 
\textcolor{blue}{BLUE} lines optional}
{\tt if IF-CONDITIONAL-EXPRESSION then IF-THEN-BLOCK \\
\textcolor{blue}{else if ELIF-CONDITIONAL-EXPRESSION-1\\
then ELSE-IF-BLOCK-1} \\
$\vdots$ \\
\textcolor{blue}{else if ELIF-CONDITIONAL-EXPRESSION-M\\
then ELSE-IF-BLOCK-M} \\
\textcolor{red}{else ELSE-BLOCK}}
\begin{itemize}
\item {\tt BLOCK} can either be a single line followed by a semi-colon
\item or it can be several lines surrounded by {\tt do;} and {\tt end;}\\
\end{itemize}
{\tt do;\\ \qquad LINE1;\\ \qquad $\vdots$\\ \qquad LINEn;\\ end;}
\end{frame}

\begin{frame}[fragile]
\frametitle{DATASTEP STATEMENTS: {\tt output}}
\begin{itemize}
\item You can selectively remove or save observations
\item The default is to save each observation when the program
reaches the {\tt run;} statement\\
or you can force that behavior with the {\tt output} statement\\
these are the same\\
\textcolor{red}{\tt data NEW; set OLD; run;}\\
\textcolor{blue}{\tt data NEW; set OLD; output; run;}
\item You can selectively pick the saved observations\\
{\tt data NEW; set OLD; if CONDITION then output; run;}
\item You can selectively save observations to different data sets\\
{\tt data NEW1 NEW2; set OLD;\\
 if CONDITION1 then output NEW1;\\
 else if CONDITION2 then output NEW2;\\
 run;}
\end{itemize}
\end{frame}

\begin{frame}[fragile]
\frametitle{DATASTEP STATEMENTS: {\tt delete} and subsetting {\tt if}}
\begin{itemize}
\item You can selectively remove observations\\
if not removed, then they are saved\\
{\tt data NEW; set OLD; if CONDITION1 then delete; run;}
\item Similarly, you can use an {\tt if} statement\\ 
(don't confuse with an {\tt if-then-else})
\item This is the same as above\\
{\tt data NEW; set OLD; if CONDITION2; run;}\\
where {\tt CONDITION2} is equivalent to {\tt not(CONDITION1)}\\
the {\tt not} function reverses the predicate
\item Typically, a {\tt where} statement is preferable to an {\tt if} for
computational efficiency especially with large data sets\\
a {\tt where} clause omits observations from even being considered\\
while {\tt if} has to selectively process each one\\
\textcolor{blue}{\tt data NEW; set OLD; where CONDITION2; run;}\\
\end{itemize}
\end{frame}

\begin{frame}[fragile]
\frametitle{SAS data set options}
\begin{itemize}
\item Operate on a data set: {\tt NAME(OPTION)}\\
and these can be combined:  {\tt NAME(OPTION1 ...\ OPTIONn)}\\
\item Generally, can be used anywhere a data set {\tt NAME} can appear\\
i.e., in PROCS and the DATASTEP (NOT just the DATASTEP like 
most DATASTEP statements)
\item {\tt NAME(drop=VAR1 ...\ VARn)}\\ 
drop these variables from the data set\\
if you can use a {\it variable list}: 
\textcolor{red}{\tt NAME(drop=VAR1-VARn)}
\item {\tt NAME(keep=VAR1 VAR2 ...\ VARn)}\\ 
keep ONLY these variables in the data set\\
%if you can use a {\it list}: {\tt NAME(keep=VAR1-VARn)}
\item {\tt NAME(rename=(OLD1=NEW1 ...\ OLDn=NEWn))}\\
rename these variables in the data set\\
or if you can use variable lists\\
\textcolor{blue}{\tt NAME(rename=(OLD1-OLDn=NEW1-NEWn))}
\end{itemize}
\end{frame}

\begin{frame}[fragile]
\frametitle{SAS data set options}
\begin{itemize}
\item {\tt NAME(firstobs=N)}\\
start with observation number {\tt N} by skipping 1 to {\tt N-1}
\item {\tt NAME(obs=N)}\\
limit to observations number 1 to {\tt N}\\
{\tt NAME(firstobs=M obs=N)}\\
OR limit to observations number {\tt M} to {\tt N}\\
\item \textcolor{red}{\tt NAME(where=(CONDITION))}\\
limit to observations satisfying {\tt CONDITION}\\
very useful for those PROCS that don't accept a {\tt where} statement
\item {\tt INDEX=(KEY=(VAR1 ...\ VARn))}\\
create a new {\tt KEY} to retrieve observations in {\tt KEY} order\\ 
{\tt INDEX=(KEY=(VAR1 ...\ VARn)/unique)}\\
use the {\tt unique} keyword if it is a unique key\\
\textcolor{blue}{What is a unique key?}
\end{itemize}
\end{frame}

\begin{frame}[fragile]
\frametitle{Creating output with {\tt proc print}}
\begin{itemize}
\item {\tt proc print data=NAME;\\ var VAR1 ...\ VARn;\\ run;}
\end{itemize}
\begin{verbatim}
proc print UNIFORM N data=ntdb.elder;
    where sbp=0 | pulse=0;
    var new_doa doa dead sbp pulse;
run;
\end{verbatim}
\end{frame}

\end{document}
