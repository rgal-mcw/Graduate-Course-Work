\documentclass[11pt,pdftex,dvipsnames,usenames,helvetica]{beamer}
\setbeameroption{show notes}
\usepackage[round]{natbib}
\usepackage{amsmath}
\usepackage{amssymb}
\usepackage{graphicx}
\DeclareGraphicsExtensions{.ps,.eps,.pdf,.jpg,.png}
\usefonttheme[onlymath]{serif}
\usepackage[cmintegrals,cmbraces]{newtxmath}
\usetheme{default}
\usecolortheme{dove}

\usepackage{cancel}
\usepackage{tikz}
\usepackage[english]{babel}

\usepackage{verbatim}
\usepackage{color}
\usepackage{pgf} %portable graphics format

\usepackage[autobold]{statex2}
\mode<presentation>
{
  %\usetheme{Warsaw}
  % or ...
  \setbeamercovered{transparent}
  % or whatever (possibly just delete it)

  \setbeamertemplate{navigation symbols}{}
  \usefonttheme[onlysmall]{structurebold}
  %\usefonttheme{structurebold}
}
\addtobeamertemplate{navigation symbols}{}{%
    \usebeamerfont{footline}%
  \setbeamertemplate{navigation symbols}{}
    \usebeamercolor[fg]{footline}%
    \hspace{1em}%
    \insertframenumber/\inserttotalframenumber
}
\newcommand*{\red}[1]{\textcolor{red}{#1}}
\newcommand*{\blue}[1]{\textcolor{blue}{#1}}

\begin{document}
\boldmath
% 0. Intro

\begin{frame}
\frametitle{Graduate School Class Reminders}

\begin{itemize}
% \item Keep your facial covering on, including over your nose
\item Maintain six feet of distancing
\item Please sit in the same chair each class time
\item Observe entry/exit doors as marked
\item Use hand sanitizer when you enter/exit the classroom
\item Use a disinfectant wipe/spray to wipe down your learning space
  before and after class
\item Media Services: 414 955-4357 option 2
%\item START RECORDING
\end{itemize}

\end{frame}

\begin{frame}
\frametitle{Outline for today}

\begin{itemize}
\item Introductions
\item Syllabus 
\item Documentation 
\item Computing on the Cheese Cluster
\item Welcome to Emacs
% \begin{itemize}
% \item MCWCorp authentication, WiFi and the Solstice client
% \item Secure Shell and Secure Copy 
% \item Data shares, LAN and WiFi test
% \item Linux commands
% \item x2go checkup (TODO from home: Internet and VPN test)
% \item emacs and ESS
% \item R and SAS
% \end{itemize}
\end{itemize}

\end{frame}

\begin{frame}
\frametitle{Introductions and syllabus}

\begin{itemize}
\item Introduce yourself
\begin{itemize}
\item Pronounce your name
\item Where are you from?
\item What graduate program are you in?
\item What schools did you go to and your major?
\item What are your other interests?
\end{itemize}
\item Do you have a laptop?  If so, what OS?  Bring it along
\item Programming experience with R, SAS and Linux\\ %, C++, Emacs and LaTeX
(the biostatistical ``stack'' and basis for this course)
\item For example, here's my background
\item A statistician AND a programmer: you have to be both
\item I consider myself a data scientist, NOT a computer scientist
\item From 1991,  heavy use of SAS
\item From 1991,  heavy use of UNIX/Linux/Unix-like
%\item From 1998,  heavy use of Emacs
\item From 2009,  heavy use of R
%\item From 2009,  heavy use of C++
%\item From 2009,  heavy use of LaTeX (most of course materials)
\end{itemize}

\end{frame}

\begin{frame}
\frametitle{Documentation on the web}

\begin{itemize}
%\item Syllabus
\item CRAN: \url{http://cran.r-project.org}
\item R manuals: \url{https://cran.r-project.org/manuals.html}
\item SAS: \url{http://support.sas.com/documentation}
\item Step-by-Step Programming with Base SAS 9.4 (SbS): \\
\url{https://documentation.sas.com/api/docsets/basess/9.4/content/basess.pdf}
\item SAS 9.4 Programmer’s Guide: Essentials (PGE): \\
\url{https://documentation.sas.com/api/docsets/lepg/9.4/content/lepg.pdf}
\item Wiki: \url{https://wiki.biostat.mcw.edu} \textcolor{red}{(MCW/VPN)}
\end{itemize}

\end{frame}

\begin{frame}
\frametitle{Wikipedia: a moment in my history}

\begin{itemize}
\item for computer science and technology, it is an excellent free
  choice when others are unavailable
\item most articles have citations for those who need to dig further
\item it is very helpful as an introduction to a new topic
\item but doesn't have much to offer for those learning R and SAS
\item the follow-on to Nupedia which I was a part of in 2000
\item Nupedia founded by Jimmy Wales %with an estimated net worth
%  today of \$1M (via google search) \\
who later founded Wikipedia in January 2001 which arose from it
% \item I worked with Larry Sanger Editor-in-chief of Nupedia\\
%  (later Editor-in-chief of Wikipedia) with an estimated net worth of
%   \$650K (a google search did not reveal my net worth!)
\item Nupedia was a dramatic failure and a footnote in history
\item But it spawned Wikipedia a major unintended success
\item wiki is simple hence its popularity: ironically I far prefer LaTeX
\item \url{https://en.wikipedia.org/wiki/Wiki}
%\item \url{https://wiki.biostat.mcw.edu}
\end{itemize}
\end{frame}

\begin{frame}
\frametitle{RTFM: Read the F****ing Manual}

``If all else fails read the instructions.''
 Donald Knuth, a renowned computer scientist and 
the inventor of TeX (precursor to LaTeX) and many 
other achievements, might (or might not) have said this

\begin{itemize}
\item Unix means UNIX, Linux, macOS or similar Unix-like operating systems
\item To identify Unix-like: {\tt .Platform\$OS.type=="unix"}
\item A counter-example: it does NOT mean Microsoft Windows
\item To identify MS Windows: {\tt .Platform\$OS.type=="windows"}
\item  \red{on gouda} {\tt \$ man} Unix-command like {\tt \$ man man}
\item  \red{on gouda} {\tt \$ info} Unix-command like {\tt \$ info info}
\item Emacs has the capability to render both types as we'll see
\item {\tt M-x man}
\item {\tt M-x info}
\item N.B.\ the Helvetica sans serif proportional font used for text
  whereas a fixed-width {\tt teletype/typewriter} font with
  serifs is used for commands and URLs, etc., i.e., what you type
\end{itemize}

\end{frame}

\begin{comment}
\begin{frame}
\frametitle{Recurring themes}

\begin{itemize}
\item Linux/UNIX/Unix-like support multi-threading/processing
\item Windows is unable to ``fork'': R multi-threading is crippled
\item Spreadsheets are not type safe: {\tt wikipedia Type\_safety}
\item SAS excels at sorting and merging data especially with unique
  keys providing user-friendly errors/warnings/notes
\item R can sort/merge but is often silent about mis-specification
\item R has a rich interactive environment and SAS does NOT
\item R and C++ are object-oriented and Rcpp glues them together
\item Emacs is a programmer's editor and we are programmers
\item Emacs supports many languages like C/C++ and LaTeX
\item ESS (Emacs Speaks Statistics) supports R and SAS
\item RStudio is a waste of time
\item x2go ``accelerates'' X11 traffic over the Internet 
\item Reproducible errors vs.\ sporadic errors
\end{itemize}

\end{frame}
\end{comment}

\begin{frame}
\frametitle{Typical biostatistical computing stack}
(TIOBE Index of the Top 100 Programming Languages:\\ 
August 2022)
\url{https://www.tiobe.com/tiobe-index}

\begin{itemize}
\item GNU/Linux kernel
\item GNU Compiler Collection (GCC): C(2)/C++(4)/Fortran(19)
\item GNU tools: bash(43), core utilities
\item GNU Emacs: programmer' editor \\
mostly written in Emacs Lisp(30) and the rest in C
\item GNU R(16): its free and is supported by other free software\\
Comprehensive R Archive Network (CRAN): 18558 packages\\
descendent of S and S-Plus
\item SAS(24) is not free, but MCW has a site-wide license\\
(that I negotiated)
\item X Window System and GNOME2/MATE
\item Structured Query Language: SQL(9) for relational databases
\item TeX Live bundle of TeX/LaTeX 
\item Python(1)/Java(3)/Perl(20) NOT used much in biostatistics\\
so they will not be discussed further in this course\\
\red{N.B. Python is interactive (slow) so we naturally use R instead}
\end{itemize}

\end{frame}

\begin{frame}\frametitle{Data Advantages of R and/or SAS vs.\ everything else}
\begin{itemize}
\item \textcolor{blue}{programming languages for data processing and analysis} 
\item convenient environments that are naturally \textcolor{red}{vectorized}
\item many years of refinement: SAS (1966) and S/R (1976)
\item wide availability on many platforms
\item particularly well-adapted to Linux/UNIX (and macOS for R)\\
neither SAS nor R adapt well to Windows as we'll see
\item reasonably priced: R is free and SAS is paid for by employers
\item at MCW, SAS is free for students and cheap/free for others
\item large user community of statisticians and related professions
\item interfaces for data input and data management systems
\item many functions of interest such as probability distributions
\item facilitate the creation of graphical visualizations 
\item provide access to advanced statistical analysis routines:
NOT the focus of this course so listed last, but immensely important
\end{itemize}
\end{frame}

\begin{frame}
\frametitle{Advantages of the Client-Server Model: \textcolor{blue}{the Server}}
Modern GNU/Linux Server Technology Layers
\begin{itemize}
%\item Standards: %Single UNIX Specification (SUS) and 
%	   Portable Operating System Interface (POSIX)
\item Linux kernel:  always in memory facilitating hardware/software
	  % UNIX flavors: Bell Labs, Berkeley Software Distribution (BSD) 
	  % and CMU Mach
	  % GNU Linux flavors: Red Hat/CentOS and Debian
\item Software tools:
%	  UNIX flavors: Bell Labs and BSD 
    	  GNU Project
\item Dynamic shared libraries and package managers:
\item %Networking: 
	  Transmission Control Protocol/Internet Protocol (TCP/IP)\\
Secure Sockets Layer/Transport Level Security (SSL/TLS) %and Secure Shell (SSH)
%	  Virtual Private Network (VPN)
\item Programming language: 
%	  ISO/IEC C/C++ programming language
	  GNU Compiler Collection (GCC)%: C/C++/Fortran
\item Graphical User Interface:
	  X.org X Window System\\ X version 11 (X11) protocol
%	  Tcl/Tk Tool Command Language (Tcl) and Toolkit (Tk)
%	  NX technology
	  and GNOME2/MATE
\item Text: 
%     	  7-bit 
American Standard Code for Information Interchange (ASCII)
%     	  8/16/32-bit ISO/IEC 10646 
and Unicode Transformation Format (UTF)
%\item Font technology:
          % Adobe Type 1
	  % Apple TrueType
          % fontconfig
%\item Printing: Adobe PostScript (PS) and Portable Document Format (PDF)
\item File system:
     Redundant Array of Independent Disks (RAID), 
     Journaling,
     %Silicon Graphics/Red Hat 
Extents File System (XFS) and \textcolor{red}{tape backup}
\item Central Processing Unit (CPU):
%     Instruction Set: 
AMD64/x86-64
     Multi-threading: servers typically have 2 or 4 CPUs and each has multiple cores capable of 1 or 2 threads %(most laptops too)
\item General Purpose Computing on Graphics Processing Units (GPGPU):
     As a consequence, incapable of audio/video 
\end{itemize}

\end{frame}

\begin{frame}
\frametitle{Advantages of the Client-Server Model: \textcolor{blue}{the Client}}
Many layers are similar: just listing important differences
\begin{itemize}
\item Use your PC/Mac as the client rather than 
                comparable server hardware/software which does not
		work as well for a variety of reasons: not up-to-date,
                lacking hardware/drivers, %Microsoft Office support, 
		etc.
\item Client software: email software with calendaring/filtering/etc., 
modern web browsers and Microsoft Office support
\item Graphics Processing Units (GPU):
          Capable of processing audio/video
\item File system: \textcolor{red}{NO TAPE BACKUP} \\ 
\textcolor{blue}{Store everything that you want
to keep on the server}
\item Virtual Private Network (VPN): with VPN software
the client can reach the server from ANYWHERE via the Internet
(ANYWHERE except MCW because the VPN hardware is on the
Internet side of our local connection so we can't get to it,
but we don't need to when we are on campus)
\end{itemize}

\end{frame}

\begin{comment}
\begin{frame}
\frametitle{Distant third phenomenon}

Slight preference for first over second and a wide distance to third
Example 1, hierarchy of biomedical evidence
\begin{enumerate}
\item[1] randomized clinical trials
\item[2] meta-analyses and reviews
\item[3] observational studies
\end{enumerate}

Example 2a, preference for article/dissertation citations
\begin{enumerate}
\item[1] journal articles 
\item[2] books in print
\item[3] technical reports, manuals, URLs, etc.
\end{enumerate}

Example 2b, reference material for computer science and technology
\begin{enumerate}
\item[1] books in print and online manuals/compendiums
\item[2] journal articles including prestigious proceedings
\item[3] Wikipedia
\end{enumerate}

\end{frame}
\end{comment}

\begin{frame}
\frametitle{A Brief History of UNIX (R)}
\begin{itemize}
\item[1969:] AT\&T Bell Labs starts work on UNIX
\item[1970:] open source UNIX provided for small fee: Bell flavored UNIX
\item[1972-3:] Bell Labs develops C, re-writes UNIX in C 
\item[1973-8:] DARPA invents TCP/IP network protocol 
\item[1978:] University of California releases Berkeley Software Distribution 
BSD flavored UNIX 
\item[1981-3:] ARPANET goes TCP/IP creating the Internet
\item[1987:] MIT/DEC release the X Window System\\
X protocol version 11 AKA X11
\item[1988-94:] Apple lawsuit vs.\ Microsoft and HP over\\
 GUI \textcolor{blue}{``Look and Feel''} copyright/patent infringement
\item[1990:] AT\&T and BSD are merged into UNIX SVR4 
\item[1992-4:] \textcolor{red}{UNIX Wars:} Free BSD release blocked by AT\&T lawsuit
%\item[1992:] SUN releases Solaris (SVR4) with X11 
\item[1993:] CDE is released: standard UNIX GUI 
\item[2000:] TLS (1999) and SSH (2000) are released for Internet security 
\end{itemize}
\end{frame}

\begin{frame}
\frametitle{A Brief History of GNU Linux}

\begin{itemize}
\item[1984:] Richard Stallman creates GNU (GNU is Not Unix) as\\ 
``a complete UNIX-compatible software system'' \\
along with the GNU General Public License (GPL) 
\item[1991:] Linux kernel by Linus Torvalds mimics UNIX 
\item[1992:] Linux kernel GPLed and paired with GNU tools 
\item[1992-4:] \textcolor{red}{UNIX Wars:} Free BSD release blocked by AT\&T lawsuit gifts Linux mindshare with Debian and Red Hat releases 
\item[1997-8:] KDE/GNOME GUIs head start\\ (GNOME for Solaris not until 2000)
\item[1999:] Intelligent package installers \\ GNU Compiler Collection (GCC): C/C++/FORTRAN 
\item[2002:] Red Hat Enterprise Linux (RHEL): Red Hat flavored Linux
\item[2003:] Fedora Project including Fedora Linux for lap-/desk-top PCs and
 Extra Packages for Enterprise Linux (EPEL)
\item[2004:] Lap-/desk-top friendly Ubuntu Linux: Debian flavored \\ 
Linux shipped on 20-50\% of new servers 
\item[2006:] the free RHEL clone CentOS is announced
\item[2014-20:] CentOS: an ``official'' clone of Red Hat
\end{itemize}
\end{frame}

\begin{frame}
\frametitle{A Brief History of GNU R}

\begin{itemize}
\item[1976:] John Chambers creates S at Bell labs for UNIX: free software
\item[1988:] S-Plus released for DOS: a commercial successor to S
\item[1995-8:] S-Plus released for Windows (1995), UNIX (1996) and\\
 Linux (1998)
\item[1997:] Ross Ihaka and Robert Gentleman release GNU R as a free
successor to S that is integrated with MacOS and Windows\\
with the R Development Core Team (including Chambers)
\item[1999:] Comprehensive R Archive Network (CRAN) debuts
\item[2001:] R supports Mac OS X (macOS) and enters the Unix-like era
\item[2019:] R v3.6 released
\item[2020:] R v4.0 released
\end{itemize}

\end{frame}

\begin{frame}
\frametitle{A Brief History of SAS}

\begin{itemize}
\item[1966-8:] Anthony Barr and Jim Goodnight begin development at\\
 North Carolina State University on IBM mainframe computers
\item[1972:] First release of SAS, but they lose their NIH funding 
\item[1973:] John Sall joins the project
\item[1976:] SAS Institute is incorporated in Cary, NC
\item[1985:] SAS re-written in C for OS portability: modern SAS era begins
\item[1990:] SAS v6 released with added support for UNIX and Windows
\item[1999:] SAS v8 released with added support for Linux
\item[2013:] SAS v9.4 Analytical Products 12.3 released
\item[2018:] SAS v9.4 Analytical Products 15.1 released
\item[2020:] SAS v9.4 Analytical Products 15.2 released
\end{itemize}

\end{frame}

\begin{frame}
\frametitle{MCWCorp authentication and Solstice}

\begin{itemize}
\item start here on day 2
\item one username/password to rule them all?
\item cross-platform MCWCORP ActiveDirectory authentication
\item WiFi, email, Office365, D2L, \textcolor{blue}{VPN} and 
\textcolor{red}{the Cheese Cluster}
\item download and install the Solstice client\\
\url{https://www.mersive.com/download}
\item hands-on: taking over the projector\\
Caveat: M1 Macs don't work well in my experience 
%\item SHARES AVAILABLE OVER WIFI FOR MCWCORP ONLY
%\url{https://wiki.biostat.mcw.edu/Shares}
%\item this course: {\tt /data/shared/04224} or the {\tt 04224} share
%\item login to gouda and go to the next slide
\end{itemize}

\end{frame}

\begin{frame}
\frametitle{\textcolor{red}{The Cheese Cluster}}% and x2go clients}

\begin{itemize}
\item login to gouda %and go to the next slide
\item Your WiFi gateway
\item To run demanding code start on an interactive shell on colby
\item The Cheese Cluster uses the Terascale Open-source Resource and QUEue Manager (TORQUE)
\item For single threading\\ 
{\tt qsub -I -X}\\
{\tt qsub -I -X -l nodes=1} 
\item For multi-threading with {\tt P} threads\\
{\tt qsub -I -X -l nodes=1:ppn=P} 
\item See the wiki for more info on TORQUE

\end{itemize}

\end{frame}


\begin{frame}[fragile]
\frametitle{Secure shell, ssh, and Secure copy, scp}

\begin{itemize}
\item The commands {\tt ssh} and {\tt scp} are included in Linux and macOS
\item For Windows, see the wiki
\end{itemize}
\url{https://wiki.biostat.mcw.edu/Secure_shell_and_secure_copy}\\
\url{https://wiki.biostat.mcw.edu/Secure_copy}
\end{frame}
\begin{comment}
\item x2go provides these commands for Windows from PuTTY\\
as {\tt plink} and {\tt pscp} respectively
\item But you need to have them in your {\tt PATH}
\item Create/edit {\tt AUTOEXEC.BAT} with the following lines
%and run it by typing ``autoexec'' on the command line %: {\tt $>$ autoexec}
\end{itemize}
{\tt :: this is a comment for the file AUTOEXEC.BAT }\\
{\tt set path=\%path\%C:$\diagdown$Program~Files~(x86)$\diagdown$x2goclient;}
How to run a batch file at startup of Windows 8 and 10?
\begin{enumerate}
\item Create a shortcut to the batch file.
\item Once the shortcut is created, right-click the shortcut file and select Cut.
\item Press Start, type ``run'', and press Enter.
\item In the Run window, type ``shell:startup'' to open the Startup folder.
\item Once the Startup folder is opened, click the Home tab at the top of the folder. Then, select Paste to paste the shortcut file into the Startup folder.
\end{enumerate}
\end{comment}

\begin{frame}
\frametitle{Secure shell, ssh, and Secure copy, scp}

\begin{itemize}
\item First, we need to create keys: a public key
and a private key
\item Let's look at the manual for {\tt ssh-keygen}
\item check the directory on client and/or server: {\tt \$ ls -la .ssh}
\item for example, on Linux or macOS, we would do as follows
\item create it if not there: {\tt \$ mkdir .ssh}
\item and protect it! {\tt \$ chmod 700 .ssh}
\item check it again: {\tt \$ ls -la .ssh}
\item generate your keys on the server and copy them to your client
\item {\tt \$ ssh-keygen -t rsa}
\item {\tt \$ cp .ssh/id\_rsa.pub .ssh/authorized\_keys}
\end{itemize}
{\tt scp USER@gouda.biostat.mcw.edu:.ssh/id\_rsa.pub .ssh}\\
{\tt scp USER@gouda.biostat.mcw.edu:.ssh/id\_rsa .ssh}\\
login to gouda with {\tt ssh} or x2go with your key (no password)
%\item login to gouda: {\tt \$ ssh -X USER@gouda.biostat.mcw.edu}
%{\tt \$ scp .ssh/id\_rsa.pub USER@gouda.biostat.mcw.edu:.ssh}
\end{frame}

\begin{frame}
\frametitle{x2go clients}

\begin{itemize}
\item download and install x2go client
\item \url{https://wiki.x2go.org/doku.php/doc:installation:x2goclient}
\item For Windows, this is the best version according to Chris
\url{https://code.x2go.org/releases/binary-win32/x2goclient/releases/4.1.2.0-2018.06.22}
\item let's go over the x2go settings and test
\end{itemize}
\end{frame}

\begin{comment}
\begin{verbatim}
mkdir ~/.ssh
ssh-keygen -t rsa
chmod 700 ~/.ssh
cp ~/.ssh/id_rsa.pub ~/.ssh/authorized_keys
chmod 600 ~/.ssh/authorized_keys ~/.ssh/id_rsa
\end{verbatim}
\end{frame}
\end{comment}

\begin{frame}
\frametitle{A Brief History of Emacs and ESS}
\begin{itemize}
\item[1975:] Emacs created by Richard Stallman (AKA RMS) at MIT 
\item[1984:] RMS re-writes GNU Emacs (GPL) in C \\ 
\textcolor{red}{Apple Macintosh Human Interface Guidelines (HIG)} 
\item[1986:] emacs FORTRAN-mode: intelligent editing for FORTRAN 
\item[1987:] IBM Common User Access (CUA) failed altnerative to HIG
\item[1988-94:] Apple lawsuit vs. Microsoft and HP over\\
 GUI \textcolor{blue}{``Look and Feel''} copyright/patent infringement 
\item[1990:] John Sall adds some SAS support to GNU Emacs 
\item[1991:] Multi-lingual XEmacs (GPL) for X11 released 
\item[1994:] GNU Emacs (GPL) for X11 released \\ 
Tom Cook releases SAS-mode (GPL) 
\item[1994-7:] Anthony Rossini creates ESS (GPL) containing\\
the Emacs modes ESS[SAS], ESS[R] and ESS[Stata] 
\item[1999+:] Rich Heiberger and I improve ESS[SAS] for batch processing 
\item[2000+:] ESS and Emacs evolve together for more user-friendliness
\end{itemize}
\end{frame}

\begin{frame}[fragile]
\frametitle{Welcome to Emacs}
\url{https://wiki.biostat.mcw.edu/Emacs}
\begin{itemize}
\item Modifier Keys: Emacs documentation looks like this
\item {\tt C-KEY} means hold down the Control key while pressing {\tt KEY}
\item For example, {\tt C-x} means hold down Control while pressing {\tt x}
\item {\tt M-KEY} means hold down the Meta key while pressing {\tt KEY}
\item On PC, the Meta key is usually the Alt key 
\item On Mac, the Meta key should be the Option key: see last slide
% {\tt $\sim$/.emacs}
% \begin{verbatim}
% (setq mac-command-modifier 'alt)
% (setq mac-option-modifier 'meta)
% \end{verbatim}
\item Or, you can press {\tt Esc}, release,
  and then press {\tt KEY}
\item Execute an emacs command: {\tt M-x COMMAND} which is followed by
  pressing {\tt Enter}
\item For example, {\tt M-x man} to bring up a man page\\
 or {\tt M-x info} the directory of info pages
\item {\tt S-KEY} means hold down the Shift key while pressing {\tt KEY}
% \item Some examples. M-x list-packages M-x list-fontsets C-u M-x
%   list-fontsets ({\tt C-u} is called prefix argument which alters the
%   next command).
\end{itemize}

\end{frame}

\begin{frame}
\frametitle{Hands-on running Emacs}
\begin{itemize}
\item To run more recent versions of GCC and the GNU Debugger\\ 
(these are needed to compile R packages)
\item {\tt \$ module load gcc/9.2 gdb/9.2}
%\item {\tt \$ module load emacs/26.3 R/3.6.2 \&\& emacs "\$@"}
\item Let's create a bash shell script to avoid typing that again
\item Open the file to edit: {\tt C-x C-f emacs-26.3}
\item {\tt \#!/bin/bash} \\ 
{\tt (module load gcc/9.2 gdb/9.2; emacs "\$@")}\\
Encasing commands in parentheses is a \textcolor{red}{subshell}
%{\tt module load emacs/26.3 R/3.6.2 \&\& emacs "\$@"}
%\item Or copy it: {\tt C-x C-f /data/shared/04224/emacs-26.3}
%\item By saving it: {\tt C-x C-w $~$/emacs-26.3}
\item Try to quit emacs: {\tt C-x C-c}
\item Interrupt command: {\tt C-g}
\item Save the file: {\tt C-x C-s}
\item Really quit emacs: {\tt C-x C-c}
\item Or just copy it:\\ {\tt \$ cp /data/shared/04224/emacs/emacs-26.3 $~$}
\item Check the files permissions: {\tt \$ ls -l emacs-26.3}
\item Add the execute permission: {\tt \$ chmod +x  emacs-26.3}
\item Check the files permissions: {\tt \$ ls -l emacs-26.3}
\item Run the script in the background: {\tt \$ emacs-26.3 \& }
\end{itemize}
\end{frame}

\begin{frame}
\frametitle{Common Emacs Shortcuts}

\begin{itemize}
\item {\tt C-h} is the help key and {\tt F1} is its alias
\item But you have to get your laptop to generate an {\tt F1}
\item For example, {\tt C-h k} describes the next key pressed
\item Try {\tt C-h k F1 k}
\item {\tt M-F1} opens a new frame (MCW)
\item {\tt C-F10} make the font smaller (MCW)
\item {\tt C-F11} make the font bigger (MCW)
\item {\tt C-x C-f} is open a file or a directory
\item {\tt F2} is refresh (ESS)
\item {\tt F8} is go to {\tt *shell*} buffer (ESS)
\item {\tt M-F8} is go to {\tt *shell*} buffer in the current directory (MCW)
\item {\tt M-w} is \textcolor{red}{copy}
\item {\tt C-y} is \textcolor{blue}{paste}
\item {\tt C-w} and {\tt Delete} are cut 
\item {\tt C-delete} or {\tt S-delete} is cut (MCW)

\end{itemize}

\end{frame}

\begin{frame}
\frametitle{Common Emacs Shortcuts}

\begin{itemize}
\item {\tt C-c} comments a region (an area of text selected)
\item {\tt C-u} is the prefix command so 
{\tt C-u C-c} uncomments a region
\item {\tt M-up} go to the beginning of the file (MCW)
\item {\tt M-down} go to the end of the file (MCW)
%\item {\tt M-F1} creates a new emacs frame (MCW)
\item {\tt C-x 2} splits the buffer top over bottom
\item {\tt C-x 1} unsplits the buffer
\item {\tt C-x 3} splits the buffer left and right
\item {\tt C-s} starts a forward search
\item Repeating {\tt C-s} searches for the same string again
\item {\tt C-r} starts a reverse search
\item {\tt C-u C-s} starts a forward regular expression search
\item See Search:Regexps entry of emacs manual : {\tt M-x info}
\item {\tt M-x man re\_format} man page on regular expressions
\end{itemize}

\end{frame}

\begin{frame}
\frametitle{R and Emacs/ESS for your client}

\begin{itemize}
\item \textcolor{blue}{\url{https://cran.r-project.org/bin}}
\item Vincent Goulet has developed installable binaries
for Windows and macOS
\item For Windows: \url{https://vigou3.gitlab.io/emacs-modified-windows}
%\item For macOS: \url{https://vigou3.gitlab.io/emacs-modified-macos}
\item Check ESS is working with {\tt M-x ess-version}
\item But, for macOS, Vincent's binaries cause
frequent crashes 
\end{itemize}

\end{frame}

\begin{frame}
\frametitle{Emacs/ESS on macOS}

\begin{itemize}
\item So I created my own binary based on {\tt homebrew} from
\url{https://github.com/railwaycat/homebrew-emacsmacport/releases}
%\item (these might work on earlier versions of macOS: haven't tried)
\item Copy this file (newer versions of emacs also have issues): 
{\tt /data/shared/04224/emacs/emacs-26.3.tar.gz}
\item And extract: {\tt \$ tar xzf emacs-26.3.tar.gz -C /}
%\item Copy this file: {\tt /data/shared/04224/emacs/emacs.tar.gz}
%\item And extract: {\tt \$ tar xzf emacs.tar.gz -C /}
\item \textcolor{red}{Mac Command key settings for {\tt $~$/.emacs}\\ 
{\tt /data/shared/04224/emacs/emacs-macOS.el}}
\item You also have to install XQuartz\\
\url{https://www.xquartz.org}
\item And set its Preferences accordingly on the Input tab:\\
check ``Option keys send Alt\_L and Alt\_R''
\item Also, in the Keyboard System Preferences
\item On the Keyboard tab\\
check ``Use F1, F2, etc., keys as standard function keys''
\item On the Shortcuts tab you will see \textasciicircum UpArrow
and \textasciicircum DownArrow\\
uncheck {\tt C-up} and {\tt C-down} 
\item Check ESS is working with {\tt M-x ess-version}
\end{itemize}
\end{frame}

\end{document}
