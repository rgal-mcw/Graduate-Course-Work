\documentclass[11pt,pdftex,dvipsnames,usenames,helvetica]{beamer}
\setbeameroption{show notes}
\usepackage[round]{natbib}
\usepackage{textcomp}
\usepackage{amsmath}
\usepackage{amssymb}
\usepackage{graphicx}
\DeclareGraphicsExtensions{.ps,.eps,.pdf,.jpg,.png}
\usefonttheme[onlymath]{serif}
\usepackage[cmintegrals,cmbraces]{newtxmath}
\usetheme{default}
\usecolortheme{dove}

\usepackage{cancel}
\usepackage{tikz}
\usepackage[english]{babel}

\usepackage{verbatim}
\usepackage{color}
\usepackage{pgf} %portable graphics format
\usepackage[autobold]{statex2}
\mode<presentation>
{
  %\usetheme{Warsaw}
  % or ...
  \setbeamercovered{transparent}
  % or whatever (possibly just delete it)

  \setbeamertemplate{navigation symbols}{}
  \usefonttheme[onlysmall]{structurebold}
  %\usefonttheme{structurebold}
}
\addtobeamertemplate{navigation symbols}{}{%
    \usebeamerfont{footline}%
  \setbeamertemplate{navigation symbols}{}
    \usebeamercolor[fg]{footline}%
    \hspace{1em}%
    \insertframenumber/\inserttotalframenumber
}
\newcommand*{\red}[1]{\textcolor{red}{#1}}
\newcommand*{\blue}[1]{\textcolor{blue}{#1}}

\begin{document}
\boldmath
% 0. Intro

\begin{frame}
\frametitle{Graduate School Class Reminders}

\begin{itemize}
% \item Keep your facial covering on, including over your nose
\item Maintain six feet of distancing
\item Please sit in the same chair each class time
\item Observe entry/exit doors as marked
\item Use hand sanitizer when you enter/exit the classroom
\item Use a disinfectant wipe/spray to wipe down your learning space
  before and after class
\item Media Services: 414 955-4357 option 2
%\item START RECORDING
\end{itemize}

\end{frame}

\begin{frame}
\frametitle{Documentation on the web}

\begin{itemize}
\item CRAN: \url{http://cran.r-project.org}
\item R manuals: \url{https://cran.r-project.org/manuals.html}
\item SAS: \url{http://support.sas.com/documentation}
\item Step-by-Step Programming with Base SAS 9.4 (SbS): \\
\url{https://documentation.sas.com/api/docsets/basess/9.4/content/basess.pdf}
\item SAS 9.4 Programmer s Guide: Essentials (PGE): \\
\url{https://documentation.sas.com/api/docsets/lepg/9.4/content/lepg.pdf}
\item Wiki: \url{https://wiki.biostat.mcw.edu} 
\textcolor{red}{(MCW/VPN)}
\end{itemize}

\end{frame}

\begin{frame}[fragile]
\frametitle{HW part 1: NTDB annual hospital case volume}

\begin{itemize}
\item The number of patients seen by a given hospital is thought to be
  prognostically beneficial for many types of treatment like cancer,
  cardiovascular, trauma, etc.\\
 i.e., the {\it practice makes perfect} theory
\item The NTDB case reporting is required for those hospitals which
  are designated trauma centers
\item And their total case volume is fairly trivial to calculate
\item However, the trauma designation is appearently an annual 
designation and some hospitals go in and out of case reporting by
year of hospital visit making
their total volume a poor representation of their actual practice
\item Therefore, average annual case volume is needed: calculate it
\item Copy the data and my short program as a starting point
\item {\tt $>$ cp /data/shared/04224/NTDB/NTDB18.csv $~$ }
\item {\tt $>$ cp /data/shared/04224/NTDB/R/NTDB18.R $~$ }
\end{itemize}

\end{frame}

\begin{frame}[fragile]
\frametitle{HW part 2: NTDB \blue{cold}-decking missing imputation}

\begin{itemize}
\item For many a US Census, they were conducted on paper
\item Often items were either mistakenly or intentionally left blank
\item If the collected data is \textcolor{red}{(NOT)}
  \textcolor{blue}{Missing Completely at Random}, you can
  \textcolor{red}{(NOT)} drop missing records from statistical
  analysis
\item \red{Hot}-decking is the substitution of a \red{nearby} neighbor's
  value to replace a missing value on the resident's form
\item \blue{Cold}-decking is the substitution of \blue{any} neighbor's
  value to replace a missing value on the resident's form
\item For one or more missing covariates, record-level \blue{cold}-decking
  imputation can be employed that
  is biased towards the null, i.e., non-missing values from another
  record are randomly selected regardless of the outcome
\item This simple
  missing data imputation method is sufficient for data sets with
  relatively few missing values 
\item Perform \blue{cold}-decking for the NTDB variables: {\tt white}, {\tt
    black}, {\tt other}, {\tt hispanic}, {\tt male}, {\tt sbp},
{\tt pulse} and {\tt rr} (respiratory rate) 
\item Hints: use nested {\tt for/while} loops and 
{\tt set.seed/sample.int} for random integers
\end{itemize}

\end{frame}

\begin{frame}[fragile]
\frametitle{HW part 2: NTDB \blue{cold}-decking missing imputation}

\begin{itemize}
\item More details
\item Suppose that we have the following 5 variables:\\ 
household income, owned home vs.\ renting,\\ 
age of home, number of rooms and number of occupants
\item It is reasonable to assume that these variables have a relationship
between them
\item Suppose record $i$ has the observed/missingness pattern\\
{\tt \textcolor{blue}{A$_i$ B$_i$} NA NA NA}
\item And we randomly draw record $j$ to replace its values\\
{\tt C$_{j}$ D$_{j}$ NA \textcolor{red}{E$_{j}$ F$_{j}$}}
\item Now, record $i$ looks like this\\
{\tt \textcolor{blue}{A$_i$ B$_i$} NA \textcolor{red}{E$_{j}$ F$_{j}$}}
\item So, we randomly draw again: record $k$\\
{\tt G$_{k}$ NA H$_{k}$ I$_{k}$ NA}
\item Now, record $i$ looks like this\\
{\tt \textcolor{blue}{A$_i$ B$_i$} H$_{k}$ \textcolor{red}{E$_{j}$ F$_{j}$}}
\end{itemize}

\end{frame}

\begin{frame}[fragile]
\frametitle{HW part 3 (mid-term): NTDB mortality data analysis}

\begin{itemize}
\item A proper statistical analysis of the NTDB is FAR beyond the
  scope of this course
\item However, it is a very convenient data set to practice a 
few of the tools described in Sections 11 and 12
\item We will perform an analysis for the outcome {\tt dead} based on 
annual average volume, {\tt white}, {\tt
    black}, {\tt other}, {\tt hispanic}, {\tt male} and {\tt sbp}
\item N.B. the data is clustered by the hospital: {\tt traumactr}\\
for such a large data set, typical clustering techniques are too
consuming so we will include {\tt traumactr} as a variable above\\
there are only 18 of them so this is a reasonable short-cut
\item We will compute common statistics and make graphical figures
\item For example, we will compute Receiver Operating Characteristic
(ROC) curves
\end{itemize}

\end{frame}

\begin{frame}[fragile]
\frametitle{HW part 4: NTDB stratified random sampling}

\begin{itemize}
\item the {\tt gcstot} variable is the Glasgow Coma Scale (GCS)\\
prognostically: 3 is the worst and 15 is the best
\item as we will see, this is the most powerful predictor of
mortality
\item suppose that we want to divide the patient cohort into, 
say M=5, random samples of roughly equal size
\item imbalances in a strong predictor like GCS will make the
random samples substantially differ in mortality relationships
due to these unwanted imbalances alone 
\item create a function to perform stratified random sampling (SRS)
that accepts the data, {\it strata} and a setting for M
\item SRS {\it stratifies}
on one (or more) important variables called {strata} variable(s)
\item \textcolor{blue}{sort} the data by the strata variable(s) and
  then use \textcolor{red}{random permutations} of 1:M to divide
the data into M samples
\item \textcolor{blue}{unsort} the data and return it in the original 
order, i.e., generally, you want to return data in the same order as
that provided by the user
\item Hint: use the {\tt order} function to sort/unsort
\end{itemize}

\end{frame}

\begin{frame}[fragile]
\frametitle{R expressions: {\tt expr}}
\begin{itemize}
\item Expressions can end in a semi-colon or end-of-line
\item Expressions can be spread over multiple lines IF each
line ends in an operator like + so that the R interpreter
knows that there is more to come
%\item In the documentation, an R expression is {\tt expr}
%\item This expression returns a value which can be assigned, passed to a function, etc.
\item Each expression returns a value 
\item You can group several expressions in curly brackets:\\ 
{\tt \{ expr$_1$; \dots; expr$_m$ \}}\\
where the return value comes from the last: {\tt expr$_m$ }
%\item Often used in multi-threading
\end{itemize}
\begin{verbatim}
library(parallel) ## an example of multi-threading
library(tools)
for(i in 1:mc.cores) mcparallel({psnice(value=19); expr})
\end{verbatim}
We will return to multi-threading later
% Paraphrasing the {\tt psnice} documentation\\
% Unix schedules processes to execute according to their priority.\\  Priority
% is assigned values from 0 to 39 with 20 being the normal priority and
% (counter-intuitively) larger numeric values denoting lower priority.
% Adding to the complexity, there is a {\it nice} value:\\ the amount by
% which the priority exceeds 20.  Processes with higher nice values will
% receive less CPU time than those with normal priority.  Generally,
% processes with nice value 19 are only run when the system would
% otherwise be idle \textcolor{blue}{to enhance system interactivity}.
\end{frame}

\begin{frame}[fragile]
\frametitle{R expressions and logical conditions}

\begin{itemize}
\item A condition is an R expression that can be evaluated as\\ a TRUE or FALSE logical value either directly or indirectly
\item A direct evaluation of a condition will yield TRUE or FALSE 
\item For example, {\tt i \%in\% 2:3}
\item An indirect evaluation occurs by type conversion or coercion
\item A character value will yield an error
\item A numeric or logical value of {\tt NA} will yield an error\\
\textcolor{red}{often this will force you to use {\tt \&\&} or {\tt ||}
instead of {\tt \&} or {\tt |}}\\
\textcolor{blue}{such as in HW part 2}
\item A numeric value of {\tt NaN} will yield an error
\item A numeric value of zero is FALSE and all others are TRUE\\
(except {\tt NA} and {\tt NaN})
\end{itemize}

\end{frame}

\begin{frame}[fragile] 
\frametitle{R {\tt if} command syntax:  \textcolor{red}{RED} and
\textcolor{blue}{BLUE} lines optional}

{\tt if( cond$_1$ ) \{ expr$_1$ \}  \\
\textcolor{red}{else if( cond$_2$ ) \{ expr$_2$ \}} \\
$\vdots$ \\
\textcolor{red}{else if( cond$_m$ ) \{ expr$_{m}$ \}} \\
\textcolor{blue}{else \{ expr$_{m+1}$ \}} }

\begin{verbatim}
obj.list = mccollect() ## continuing multi-threading ex
obj = obj.list[[1]]
if(mc.cores==1 | class(obj)[1]!=type) {
    return(obj)
} else {
    m = length(obj.list)
    if(mc.cores!=m) 
        warning(paste0("The number of items is only ", m))
    ...
}
\end{verbatim}
\end{frame}

\begin{frame}[fragile]
\frametitle{R {\tt for} command syntax}
%: \textcolor{red}{RED} and \textcolor{blue}{BLUE} clauses are optional}

{\tt for( VARIABLE in \textcolor{blue}{ expr$_1$ }) expr$_2$}

\begin{itemize}
\item If the {\tt VARIABLE} is {\tt i}, typically, {\tt expr$_2$} will involve {\tt i}
\item {\tt expr$_2$} can include \textcolor{red}{\tt break} 
which ends the loop
\item {\tt expr$_2$} can include \textcolor{blue}{\tt next} 
which indexes {\tt i} and starts {\tt expr$_2$} over
\item BEWARE: loops are rather slow for the R interpreter
%\item What is a large loop is a moving target as computer technology progresses
\item But hardware is constantly getting faster: its all relative
\item 100,000,000 loops of this example takes 11s on my laptop 
\end{itemize}
\begin{verbatim}
system.time({
M = 12
A = matrix(nrow=M, ncol=M)
dimnames(A)[[1]]=paste0(1:M)
dimnames(A)[[2]]=paste0(1:M)
for(i in 1:M)
    for(j in 1:M)
        A[i, j] = i*j ## times tables
if(M==12) print(A)
})
\end{verbatim}
\end{frame}

\begin{frame}[fragile]
\frametitle{R {\tt while} command syntax}

{\tt while(cond) expr}

\begin{itemize}
\item {\tt expr} can include the \textcolor{red}{\tt break} 
command which ceases
\end{itemize}
\end{frame}

\begin{frame}
\frametitle{Writing your own R functions }

{\tt NAME1 = function( name$_1$, \dots, name$_m$ ) expr } 
\begin{itemize}
\item the value of {\tt expr} is returned
\item but you have more control than in a typical R expression\\
you can return a value at any place within {\tt expr}
via the {\tt return} function
\item Call this function via {\tt NAME1 (expr$_1$, \dots, expr$_m$ )}
\item Or: {\tt NAME1 (name$_m$=expr$_1$, \dots, name$_1$=expr$_m$ )}
\item Some, or all, arguments can have default values so that you do not 
have to supply every single argument
\end{itemize}
{\tt NAME2 = function( name$_1$=expr$_1$, \dots, name$_m$=expr$_{m}$ ) \dots } \\
To call: {\tt NAME2 ( expr$_1$, \dots, expr$_m$ )  } \\
Or: {\tt NAME2 ( )  } \\
Or: {\tt NAME2 ( name$_n$=expr$_1$ )  } \\
Also, you can abbreviate the {\tt name}s, but it can't be ambiguous
\end{frame}

\begin{frame}
\frametitle{The {\tt ...} argument}

\begin{itemize}
\item Often from within your new function, {\tt NAME1}, you are
  calling another function (or functions): for example, let's say {\tt
    NAME2}
\item But {\tt NAME2} might have a large number of arguments
\item So it would be painful to include all of them in {\tt NAME1}
\item The {\tt ...} argument is a catch-all for named arguments
that are unknown to {\tt NAME1}
\item {\tt dots = function(...) c(...)}
\end{itemize}
\end{frame}

\begin{frame}[fragile]
\frametitle{Lexical scope and nested environments}

\begin{itemize}
\item S and S-Plus have {\it static} scope
\item R has lexical scope
\end{itemize}

\begin{verbatim}
cube = function(n) {
  sq = function() n*n ## nested function definition
  n*sq()
}
## first evaluation in S
S> cube(2)
Error in sq(): Object "n" not found
S> n = 3
S> cube(2)
[1] 18
## then the same function evaluated in R
R> cube(2)
[1] 8
\end{verbatim}

\end{frame}

\begin{frame}[fragile]
\frametitle{Lexical scope}

\begin{itemize}
\item There are three types of variables in a function\\
named arguments, local variables and free variables
\item A free variable becomes a local variable upon assignment
\item Otherwise, a free variable is sought by recursively 
widening the scope to the nested environments all the way
to the global environment if necessary\\
if not found,
then an error is generated when the function runs\\
it is not an error when the function is defined
because the variable could be created at a later time
\item EXCEPTION: a free variable (in its wider environment) can be altered from within a function
  by the {\it super-assignment} operator {\tt
    <<-} or via the {\tt assign} function\\
in my experience, this is rarely necessary and probably should be avoided
whenever possible
\end{itemize}
\end{frame}

\begin{frame}
\frametitle{Object-orientation}

\begin{itemize}
\item {\it Generic} functions can be specialized to have different actions
for objects of different types
\item They are named as {\tt NAME.CLASS} 
where {\tt NAME} is the function name to be called, i.e., {\tt NAME(object)}
\item And {\tt CLASS} is the object type as determined by the {\tt class}
attribute, i.e., {\tt class(object)}
\item Generic functions have a {\tt NAME.default} action that is
  used when either there is no {\tt class} attribute for an object or
  the specific {\tt NAME.CLASS} does not exist
\item For {\it visible} functions (and some {\it invisible}), you can
just type their name and see their definition
\item For others, you need the {\tt getAnywhere} function
\item See {\tt lecture4.R} and the generic {\tt print} function
\item There is typically help for the default method\\
 {\tt $>$ ?print.default}
\item But others may, or may not, be documented
\end{itemize}

\end{frame}

\begin{frame}
\frametitle{Section 11: Statistical models in R}

\begin{itemize}
\item This reading is assigned due to the co-requisite
requirement of this course, i.e., Statistical Models and Methods I
\item However, in the interest of time, I'm not going to lecture on this 
material since there are many other things that need to be discussed
\item I will show some examples based on the NTDB, etc.
\item And assign homework/exam problems based on it
\item So, if you have any questions, please feel free to ask
\item But a firm understanding of statistical concepts is not 
an expected outcome of this course
\item Rather, it is imperative that you learn effective R programming skills
\item Learning R intimately will help you perform statistical
  analyses and complete related statistical programming tasks, but we
  will only discuss statistical practice in a cursory manner
\end{itemize}

\end{frame}

\begin{frame}
\frametitle{American College of Surgeons National Trauma Data Bank (NTDB)}

In the US Code of Federal Regulations (CFR), 45 CFR 46.101 (title 45,
Public Welfare, section 46.101),
%Per 45 Code of Federal Regulations (CFR) 46.101, 
research using certain publicly available data sets does not involve
``human subjects''. The data contained within these specific data sets
are neither identifiable nor private and thus do not meet the federal
definition of ``human subject'' as defined in 45 CFR 46.102.  These
research projects do not need to be reviewed and approved by the
Institutional Review Board, (IRB). MCW's Human Research Protection
Program (HRPP) has a list of their recognized ``publicly available data sets''.\\
% or ``restricted use data sets''.\\
\url{https://www.mcw.edu/departments/human-research-protection-program/researchers/smartform-instructions/public-data-sets}
\end{frame}

\begin{frame}
\frametitle{NTDB and data frames}

\begin{itemize}
\item the NTDB has all de-identified trauma admissions for those\\
 aged 65 and older (restricted to 89 or less here)\\
at all designated trauma center hospitals in the US\\
about 1,000,000 admissions in the 2015 edition\\
including the years 2007 to 2015
\item the NTDB does NOT come as a CSV
\item but I created a CSV for the first 18 trauma centers\\
{\tt /data/shared/04224/NTDB/NTDB18.csv}\\
about 50,000 admissions:
there is an (incomplete) data dictionary in the same directory
\item this is for educational purposes ONLY (HIPAA exception)
\item the {\tt read.csv} function creates a {\tt data.frame}
\item a {\tt data.frame} is the typical R data set object\\ 
essentially a list of vectors of the same length
\item like a matrix where the rows are lists\\
and the columns are vectors of \textcolor{red}{potentially diverse types}
\item the {\tt subset} function creates a subset {\tt data.frame} 
\end{itemize}
\end{frame}

\end{document}
