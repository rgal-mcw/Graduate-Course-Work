\documentclass[11pt,pdftex,dvipsnames,usenames,helvetica]{beamer}
\setbeameroption{show notes}
\usepackage[round]{natbib}
\usepackage{textcomp}
\usepackage{amsmath}
\usepackage{amssymb}
\usepackage{graphicx}
\DeclareGraphicsExtensions{.ps,.eps,.pdf,.jpg,.png}
\usefonttheme[onlymath]{serif}
\usepackage[cmintegrals,cmbraces]{newtxmath}
\usetheme{default}
\usecolortheme{dove}

\usepackage{cancel}
\usepackage{tikz}
\usepackage[english]{babel}

\usepackage{verbatim}
\usepackage{color}
\usepackage{pgf} %portable graphics format
\usepackage[autobold]{statex2}
\mode<presentation>
{
  %\usetheme{Warsaw}
  % or ...
  \setbeamercovered{transparent}
  % or whatever (possibly just delete it)

  \setbeamertemplate{navigation symbols}{}
  \usefonttheme[onlysmall]{structurebold}
  %\usefonttheme{structurebold}
}
\addtobeamertemplate{navigation symbols}{}{%
    \usebeamerfont{footline}%
  \setbeamertemplate{navigation symbols}{}
    \usebeamercolor[fg]{footline}%
    \hspace{1em}%
    \insertframenumber/\inserttotalframenumber
}
\newcommand*{\red}[1]{\textcolor{red}{#1}}
\newcommand*{\blue}[1]{\textcolor{blue}{#1}}

\begin{document}
\boldmath
% 0. Intro

\begin{frame}
\frametitle{Graduate School Class Reminders}

\begin{itemize}
% \item Keep your facial covering on, including over your nose
\item Maintain six feet of distancing
\item Please sit in the same chair each class time
\item Observe entry/exit doors as marked
\item Use hand sanitizer when you enter/exit the classroom
\item Use a disinfectant wipe/spray to wipe down your learning space
  before and after class
\item Media Services: 414 955-4357 option 2
%\item START RECORDING
\end{itemize}

\end{frame}

\begin{frame}
\frametitle{Documentation on the web}

\begin{itemize}
\item CRAN: \url{http://cran.r-project.org}
\item R manuals: \url{https://cran.r-project.org/manuals.html}
\item SAS: \url{http://support.sas.com/documentation}
\item Step-by-Step Programming with Base SAS 9.4 (SbS): \\
\url{https://documentation.sas.com/api/docsets/basess/9.4/content/basess.pdf}
\item SAS 9.4 Programmer s Guide: Essentials (PGE): \\
\url{https://documentation.sas.com/api/docsets/lepg/9.4/content/lepg.pdf}
\item Wiki: \url{https://wiki.biostat.mcw.edu} 
\textcolor{red}{(MCW/VPN)}
\end{itemize}

\end{frame}

\begin{frame}[fragile]
\frametitle{C and C++}

\begin{itemize}
\item On the TIOBE Index of Programming Language Popularity:
currently, C is first and C++ is fourth
\item C is a mid-level language designed for hardware portability
\item C developed by Bell Labs in 1972 % file extension {\tt .c}
\item Recent C standards (reaching maturity): \textcolor{red}{C99} in 1999;\\ 
C11 in 2011; and C18 in 2018
\item C++ is backwards compatible with C
\item Recent C++ standards (still evolving): C++11 in 2011;\\
 \textcolor{blue}{C++14} in 2014; C++17 in 2017 and C++20 in 2020
\item C++ developed by Bell Labs in 1982 %: file extension {\tt .cpp}
\item C++ a high-level multi-paradigm language with four flavors:
C, Object-oriented C++, Template C++ and\\ 
the Standard Template Library (STL)
\item C/C++ are compiled languages: The GNU Compiler Collection (GCC)
provides open source compilers 
\item GCC docs: \url{https://gcc.gnu.org/onlinedocs/gcc-9.2.0}
\item Excellent C++ documentation: \url{https://cppreference.com}
\end{itemize}

\end{frame}

\begin{frame}[fragile]
\frametitle{C and C++}

\begin{itemize}
\item R relies on the system's C/C++ compiler setup via\\ {\tt R CMD \dots}
\item %File extension naming convention\\
R requires the following: {\tt .c} for C and {\tt .cpp} for C++
\item Statically typed languages that require variable definitions\\
{\tt double} for double-precision numbers and {\tt int} for integers\\
{\tt double x, y=0.; int i, j=0;}
\item Unlike R, C/C++ have {\it block} comments\\
\textcolor{red}{RED} is commented out\\
{\tt \textcolor{blue}{/*} 
\textcolor{red}{double x, y=0.;\\ 
\quad integer i, j=0;} \textcolor{blue}{*/} \\}
\item C++ also has single line comments: {\tt \textcolor{blue}{//} 
\textcolor{red}{double x, y=0.;}}
\item Additional functionality provided by C pre-processor 
header files: don't confuse the C pre-processor command, {\tt cpp}, 
with C++, {\tt .cpp} filename extension\\
%File extension naming convention\\
\item C++ expressions are followed by semi-colons\\
but C pre-processor commands do NOT end in semi-colons 
\end{itemize}

\end{frame}

\begin{frame}[fragile]
\frametitle{C and C++ header files}

\begin{itemize}
\item Added functionality like R add-on packages
\item C: header files and compiled libraries
\item C++: uncompiled source header-only class libraries\\
the R/Rcpp recognizes the following extensions:\\
 {\tt .h} or none at all (we will talk about build\\
{\tt \#include $<$Rcpp.h$>$ // Rcpp header}\\ 
{\tt \#include $<$complex$>$ // complex number header}
\item {\tt \#include $<$FILENAME$>$} used for headers in the
  compiler's path of header files: typically, within the {\tt
    /usr/include} directories/sub-directories or within R's
package library
\item {\tt \#include "FILENAME"} used for headers that are elsewhere
\item The C/C++ headers are documentation all by themselves
since they define the application programming interface (API)\\
you can find them at {\tt /usr/include/c++/9.2.0}
\item For R packages, you can find their headers
 with the {\tt system.file} function\\
{\tt $>$ system.file("include", package="Rcpp")}
\end{itemize}

\end{frame}

\begin{frame}[fragile]
\frametitle{C/C++ expressions}

\begin{itemize}
\item R and C/C++ expressions are similar: R is written in C
\item Each expression returns a value and ends in a semi-colon
\item You can group several expressions in curly brackets:\\ 
{\tt \{ expr$_1$; \dots; expr$_m$; \}}\\
where the return value comes from the last: {\tt expr$_m$; }
\end{itemize}

\end{frame}

\begin{frame}[fragile] 
\frametitle{C/C++ {\tt if} command syntax:\\  \textcolor{red}{RED} and
\textcolor{blue}{BLUE} lines optional}
R and C/C++ if commands are similar\\
{\tt if( cond$_1$ ) expr$_1$; \\
\textcolor{red}{else if( cond$_2$ ) expr$_2$;} \\
$\vdots$ \\
\textcolor{red}{else if( cond$_m$ ) expr$_{m}$;} \\
\textcolor{blue}{else expr$_{m+1}$};}
\begin{itemize}
\item you can combine conditions with {\tt ||} for OR
and/or\\ {\tt \&\&} for AND
\end{itemize}
\end{frame}

\begin{frame}[fragile]
\frametitle{C/C++ {\tt for} command syntax}
\begin{itemize}
\item {\tt for(INITIALIZATION; CONDITION; ITERATOR) expr;}
\item There is also a \textcolor{blue}{\tt break} statement like R
\item However, there is NO \textcolor{blue}{\tt next} statement\\
rather it is {\tt continue}
\item But, conveniently, the {\tt CONDITION} can be more complex 
\item Example:\\
{\tt for(int k=0; k<n; ++k) expr;}
%\item \textcolor{blue}{size\_t} is a type for counting, i.e., 
%the integers 0, 1, \dots
\item This example iterates \textcolor{red}{\tt k=0, \dots, n-1}
\item These are C/C++ vector indices as opposed to\\
R indices like \textcolor{red}{\tt 1, \dots, n}
\item \textcolor{blue}{In my experience, the 0-based indices are the
    biggest transitional challenge to learning C/C++ vs.\ other
    languages}
\item It is based on memory pointer address arithmetic\\
an area where C++ is much more user-friendly\\
but it inherits the 0-basis and other C-isms
\end{itemize}
\end{frame}

\begin{frame}
\frametitle{Writing your own C++ functions }

{\tt TYPE NAME1(TYPE$_1$ name$_1$, \dots, TYPE$_m$ name$_m$) 
\textcolor{red}{\{ expr \}} } 
\begin{itemize}
\item the value of {\tt expr} is NOT automatically returned like R
\item You return a value at any place within {\tt expr}
via the {\tt return} command: {\tt return expr$_r$;}
\item Call this function via {\tt NAME1(expr$_1$, \dots, expr$_m$);}\\
you must supply all of the arguments unless they have defaults
\item Some, or all, arguments can have default values so that you do not 
have to supply every single argument
\end{itemize}
{\tt TYPE NAME2(TYPE$_1$  name$_1$=expr$_1$,\dots,TYPE$_m$ name$_m$=expr$_{m}$)} \\
But you still have to call {\tt NAME2}
with the \textcolor{blue}{ arguments in order}: 
{\tt NAME2 ( expr$_1$, \dots, expr$_{m-n}$ )  } \\
\end{frame}

\begin{frame}[fragile]
\frametitle{Rcpp and R}

\begin{itemize}
\item The {C++} interface to
{R} is seamlessly provided by the {Rcpp} package
 which efficiently passes object references from
{R} to {C++} (and vice versa) as well as providing
direct access to the {R} random number generator
\item Rcpp is mainly an Object-oriented C++ flavor\\
but behind the scenes it utilizes Template C++ and\\ it was heavily
influenced by the STL
\item see EddeFran11 in the lit directory for an intro
\item The Rcpp Gallery has lots of examples \url{https://gallery.rcpp.org}
\end{itemize}
\end{frame}

\begin{frame}[fragile]
\frametitle{An example of the C interface to R}
\begin{verbatim}
SEXP a;
PROTECT(a = allocVector(REALSXP, 2));
REAL(a)[0] = 123.45;
REAL(a)[1] = 67.89;
UNPROTECT(1);
\end{verbatim}
\begin{itemize}
\item The C interface is \textcolor{red}{NOT user-friendly} unless you
  are intimately familiar with the R source code
\item There is documentation, but it is written {\it by programmers for
other programmers} to read
%\item Every line of C/C++ source code ends in a semi-colon
\item \textcolor{blue}{\tt SEXP} is a C pointer, or memory address, corresponding to an R object which can also be useful with Rcpp
\item 
the {\tt Rcpp::wrap} function returns a \textcolor{blue}{\tt SEXP}
generally back to R as a return value from a function call
\end{itemize}
\end{frame}

\begin{frame}[fragile]
\frametitle{Same example of the C++ interface to R provided by Rcpp}
\begin{verbatim}
Rcpp::NumericVector a(2); // a new length 2 numeric vector 
                          // created in R's memory space
a[0] = 123.45;
a[1] = 67.89;
\end{verbatim}
\begin{itemize}
\item The C++ interface is \textcolor{blue}{very user-friendly}
\item Notice that C/C++ vector indices range from 0 to n-1\\
as opposed to R vector indices that range from 1 to n
\item {\tt Rcpp::} is a C++ namespace and you reference 
Rcpp classes by {\tt Rcpp::CLASS}\\
similar to the way you can reference {\it visible} 
R functions by specifying their packages, i.e., {\tt parallel::detectCores}
\item there is also an {\tt R::} C++ namespace that you can
use to reference R-like C functions such as {\tt R::dxxx}, {\tt R::pxxx},
{\tt R::qxxx} and {\tt R::rxxx} from what is known as\\ 
the Standalone Rmath Library which is part of the R project\\
we will see the {\tt Rcpp\_rnorm} example later
\end{itemize}
\end{frame}

\begin{frame}[fragile]
\frametitle{R and Rcpp: vectors, matrices, lists and random numbers}
\begin{center}
\begin{tabular}{l|l}
R code & Rcpp code \\ \hline
{\tt a = numeric(n)} & {\tt Rcpp::NumericVector a(n);}\\
{\tt a[i]}           & {\tt a[i-1]} is a \textcolor{blue}{\tt double}\\
\textcolor{blue}{\tt a = rnorm(n)}  & {\tt Rcpp::RNGScope state;}\\
\textcolor{red}{\tt for(i in 1:n)} & {\tt for(int k=0; k<n; ++k)}\\
\textcolor{red}{\tt\quad a[i]=rnorm(1)} & {\tt\quad a[k]=R::rnorm(0., 1.);}\\
{\tt b = integer(n)} & {\tt Rcpp::IntegerVector b(n);}\\
{\tt c = logical(n)} & {\tt Rcpp::LogicalVector c(n);}\\
{\tt d = character(n)} & {\tt Rcpp::CharacterVector d(n);} \\
{\tt A = matrix(nrow=m, ncol=n)} & {\tt Rcpp::NumericMatrix A(m, n);}\\
{\tt A[i, j]}       & {\tt A(i-1, j-1)} is a \textcolor{blue}{\tt double}\\
{\tt B = matrix(nrow=m, ncol=n)} & {\tt Rcpp::IntegerMatrix B(m, n);}\\
{\tt x = list()}    & {\tt Rcpp::List x;}\\
{\tt x\$NAME}       & {\tt x["NAME"]}\\
\end{tabular}
\end{center}
\end{frame}

\begin{frame}[fragile]
\frametitle{R {\tt numeric} vs.\ Rcpp {\tt Numeric}}
\begin{itemize}
\item R's {\tt numeric} is the default atomic and matrix type\\
  but it is a hybrid of \textcolor{red}{\tt integer} and
  \textcolor{blue}{\tt double} (for floating point)
\item If the object is equally well represented by an integer
  expression, then an \textcolor{red}{\tt integer} object is created\\
a literal ending in \textcolor{red}{\tt L} is automatically
an \textcolor{red}{\tt integer}: {\tt 0L}
\item However, if any operations performed on the object require
  floating point operations, then it is automatically changed to
  \textcolor{blue}{\tt double} which is {\tt numeric}'s alter-ego
\item There are conversion functions {\tt as.numeric}, 
\textcolor{blue}{\tt as.double} and \textcolor{red}{\tt as.integer}
\item \textcolor{red}{\tt as.integer} truncates the decimal portion
  TOWARDS ZERO\\
like {\tt floor(object)} for positive or zero values\\
with {\tt -floor(-object)} for negative values
\item Rcpp's {\tt NumericVector} and {\tt NumericMatrix}
are not ambidextrous: they are made of \textcolor{blue}{\tt double} elements
\item Use {\tt IntegerVector} and {\tt IntegerMatrix} as needed\\
they are made of \textcolor{red}{\tt int} elements
\end{itemize}

\end{frame}

\begin{frame}[fragile]
\frametitle{Simple Rcpp examples}
\begin{itemize}
\item {\tt fibonacci.R} and {\tt fibonacci.cpp}
\item {\tt Rcpp\_rnorm.R} and {\tt Rcpp\_rnorm.cpp}
\item These use the {\tt sourceCpp} function
\item Common debugging options are {\tt verbose=TRUE}
and {\tt rebuild=TRUE}
\end{itemize}
\end{frame}

\begin{frame}[fragile]
\frametitle{HW hands-on: Rcpp and the Mandelbrot set}
\begin{itemize}
\item With Rcpp, we can make an extremely fast version of
the {\tt mandelbrot} function: far faster than multi-threading\\ 
see {\tt RcppMandelbrot.R}
\item You need to create {\tt RcppMandelbrot.cpp}
\item Hints:  R and C++ have similar structure so it is convenient to 
start with {\tt mandelbrot.R} as a guide
\item The {\tt complex} header is required to
create complex numbers so you need to {\tt \#include <complex>}
\item The class/type and constructor line looks like this\\
{\tt std::complex<double> c(x[i], y[j]), d(0., 0.);}
\item The {\tt std::pow} function is used for exponentiation
and it works with complex numbers as well as other types\\
(C/C++ has no exponentiation operator like the caret in R)
\item There is no modulus/magnitude function: instead they
provide the {\tt std::norm} function which is the square of
the modulus/magnitude
\item email me two files: your C++ program and your PDF file
\end{itemize}
\end{frame}

\end{document}
