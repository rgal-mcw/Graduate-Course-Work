\documentclass[11pt,pdftex,dvipsnames,usenames,helvetica]{beamer}
\setbeameroption{show notes}
\usepackage[round]{natbib}
\usepackage{textcomp}
\usepackage{amsmath}
\usepackage{amssymb}
\usepackage{graphicx}
\DeclareGraphicsExtensions{.ps,.eps,.pdf,.jpg,.png}
\usefonttheme[onlymath]{serif}
\usepackage[cmintegrals,cmbraces]{newtxmath}
\usetheme{default}
\usecolortheme{dove}

\usepackage{cancel}
\usepackage{tikz}
\usepackage[english]{babel}

\usepackage{verbatim}
\usepackage{color}
\usepackage{pgf} %portable graphics format
\usepackage[autobold]{statex2}
\mode<presentation>
{
  %\usetheme{Warsaw}
  % or ...
  \setbeamercovered{transparent}
  % or whatever (possibly just delete it)

  \setbeamertemplate{navigation symbols}{}
  \usefonttheme[onlysmall]{structurebold}
  %\usefonttheme{structurebold}
}
\addtobeamertemplate{navigation symbols}{}{%
    \usebeamerfont{footline}%
  \setbeamertemplate{navigation symbols}{}
    \usebeamercolor[fg]{footline}%
    \hspace{1em}%
    \insertframenumber/\inserttotalframenumber
}
\newcommand*{\red}[1]{\textcolor{red}{#1}}
\newcommand*{\blue}[1]{\textcolor{blue}{#1}}

\begin{document}
\boldmath
% 0. Intro

\begin{frame}
\frametitle{Graduate School Class Reminders}

\begin{itemize}
% \item Keep your facial covering on, including over your nose
\item Maintain six feet of distancing
\item Please sit in the same chair each class time
\item Observe entry/exit doors as marked
\item Use hand sanitizer when you enter/exit the classroom
\item Use a disinfectant wipe/spray to wipe down your learning space
  before and after class
\item Media Services: 414 955-4357 option 2
%\item START RECORDING
\end{itemize}

\end{frame}

\begin{frame}
\frametitle{Documentation on the web}

\begin{itemize}
\item CRAN: \url{http://cran.r-project.org}
\item R manuals: \url{https://cran.r-project.org/manuals.html}
\item SAS: \url{http://support.sas.com/documentation}
\item Step-by-Step Programming with Base SAS 9.4 (SbS): \\
\url{https://documentation.sas.com/api/docsets/basess/9.4/content/basess.pdf}
\item SAS 9.4 Programmer s Guide: Essentials (PGE): \\
\url{https://documentation.sas.com/api/docsets/lepg/9.4/content/lepg.pdf}
\item Wiki: \url{https://wiki.biostat.mcw.edu} 
\textcolor{red}{(MCW/VPN)}
\end{itemize}

\end{frame}

\begin{frame}
\frametitle{HW 2: ISO 8601 and the Proleptic Gregorian Calendar}
\begin{itemize}
\item ISO 8601 dictates that the Proleptic Gregorian Calendar be used
for dates prior to the start of the Gregorian Calendar
\item Is the R {\tt Date} class ISO 8601 compliant?
\item If not, then at what date does it diverge?
\item Unix/R define 1970-01-01 as date 0 counting for-/back-wards
\item The Julian Period covers all of written human history
starting on Monday 4713BCE-01-01 which is \\
ISO 8601 date -2,440,588
\item What Year-Month-Day is this in the corresponding\\
Proleptic Gregorian Calendar?
\item Hints: do this without {\tt for} loops\\
the R interpreter will take a long time for 2.4M iterations\\
%i.e., multiply the amount of time for each operation by 2.4M\\
using {\tt data.frames} and vectorization this is pretty fast\\
for example, in {\tt julian.R} see my implementation of the 
Proleptic Julian Calendar
\item Use two {\tt data.frames}: one for each Calendar\\
and combine them (but you can't use {\tt merge} for 2.4M records)
%\item But don't try to {\tt merge} two 2M record {\tt data.frame}
\item Use {\tt sprintf} to concatenate character vectors
%\item More slides follow on ISO 8601 and other details
\end{itemize}

\end{frame}

\begin{frame} 
\frametitle{Calendars, dates and ISO 8601: see Wikipedia}
\begin{itemize}
\item The {\tt Date} class represents dates
%\item Much of this information is from Wikipedia
\item The Egyptian Calendar had 365 days/year circa 3000BCE
\item Leap days were added haphazardly as seen fit
\item Emperor Julius Caesar solidified the calendar
\item The Julian Calendar created a leap day once every 
4 years (with some initial issues and corresponding corrections)
\item The first leap year of the Julian Calendar (as planned) 41BCE
\item The first day of the Julian Calendar is 45BCE-01-01 
\item The last day of the Julian Calendar is 1582-10-4 (Thursday)
\item The first day of the Gregorian Calendar is 1582-10-15 (Friday)
\item In 1582, dates 10-5 through 10-14 didn't occur
in Roman Catholic countries (it is named after Pope Gregory XIII)
\item This change returned the equinoxes to the same days as
the first year of the Julian Calendar
\item Gregorian Calendar Leap Year Rule\\
Every year that is divisible by four is a leap year\\
Except for years that are divisible by 100\\ 
Unless they are divisible by 400, e.g., Y2K
\end{itemize}

\end{frame}

\begin{frame} 
\frametitle{The Julian Period, Proleptic Calendars and ISO 8601}
\begin{itemize}
\item Proleptic Calendars adopt the rules of a Calendar\\
 and they run backwards in time before that Calendar starts
\item The Julian Period starts with Proleptic Julian Date 4713BCE-01-01
and covers all known recorded human history
\item ISO 8601 dictates a Proleptic Gregorian Calendar
\item The Julian and Gregorian Calendars have no year zero
\item \dots, 2BCE, 1BCE, 1CE, 2CE, \dots
\item But ISO 8601 dicates the following definition of year
 %(note CE is suppressed)
%\item \dots, (year=-1, 2BCE), (year=0, 1BCE), (year=1, 1), \dots
\end{itemize}
\begin{tabular}{lrl}
Year & ISO 8601 & Julian \\
& -4712         & 4713BCE \\
& \vdots        & \vdots \\
& -1            & 2BCE \\
& 0             & 1BCE \\
& 1             & 1CE \\
& \vdots        & \vdots
\end{tabular}

\end{frame}

\begin{frame} 
\frametitle{A brief overview of R}
\begin{itemize}
\item Interpreted language (very slow as opposed to compiled)
\item Interpreter written in C and Fortran
\item Considered a multi-paradigm language
\item For example, it is considered to be object-oriented and
  functional (as well as other paradigms)
\item Object-oriented and vectorized for user-friendliness: vectors
  and matrices are natural choices for {\it objects}
\item ``Everything is an object''
\item Dynamically Typed (as opposed to Statically Typed)\\
implicitly typed by first usage\\
rather than explicitly by definition/declaration 
\item Provides multi-threading by {\it forking} \red{EXCEPT on Windows}\\
Windows does not have this capability so no multi-threading\\
An odd choice since MS was an early licensee of UNIX\\
Pioneering the port of UNIX to many CPUs as XENIX\\
And MS-DOS was a ``dumbed-down'' Unix-like system\\
If GNU Linux can clone forking, then why not MS?
\end{itemize}
\end{frame}

\begin{frame} 
\frametitle{A brief overview of R}
\begin{itemize}
\item Much R functionality is written in R itself\\
with calls to compiled C/C++/Fortran as needed (very fast)
\item Mainly, these are R {\it add-on} packages as explained in\\
 the R FAQ;
see section 5.1 in \url{https://cran.r-project.org/doc/FAQ/R-FAQ.html}
\item There are two groups of packages distributed with R
\item The 14 so-called {\it base} packages: \textcolor{blue}{base},
  compiler, datasets, grDevices, graphics, grid, methods,
  \textcolor{red}{parallel}, splines, stats, stats4, tcltk, tools and
  utils\\
{\tt $>$ installed.packages(priority="base")}\\
%% VERY SLOW IF LOTS OF PACKAGES ARE INSTALLED
\item The 15 so-called {\it recommended} packages: KernSmooth, MASS,
  Matrix, boot, class, cluster, codetools, foreign,
  \textcolor{red}{lattice}, mgcv, nlme, nnet, rpart, spatial and
  survival\\
{\tt $>$ installed.packages(priority="recommended")}
\item Comprehensive R Archive Network: 18558 packages
\url{https://cran.r-project.org/web/packages/index.html}\\
Task Views \url{https://cran.r-project.org/web/views}
\end{itemize}
\end{frame}

\begin{frame} 
\frametitle{Introduction to R by Venables and Smith: R-intro}

\begin{itemize}
\item Without emacs, to launch R: {\tt \$ module load R/4.0.4; R}  
\item With emacs, launch the latest version of R: {\tt M-x R-4.0.4}\\
or simply {\tt M-x R}
\item The R quit command: {\tt q()}
\item Getting help: {\tt $>$ ?q} or {\tt $>$ help("q")}
\item In the {\tt *R:$~$*} at the {\tt $>$}, type {\tt ??} for 
\textcolor{red}{help on help}
\item Almost everything is an object (except keywords)
\item To see the definition of {\tt q}, type {\tt $>$ q}
\item To search R help and installed packages help\\
 type {\tt $>$ help.start()}\\
 VERY SLOW IF LOTS OF PACKAGES ARE INSTALLED
\item Double quotes vs.\ single quotes: the docs
shows double quotes in most places, but single quotes seem to work fine
and they are easier to type: I will try to use double in my notes
\end{itemize} 

\end{frame}

\begin{frame} 
\frametitle{Section 1: Storing and re-loading objects}

\begin{itemize}
\item Names contain alphanumeric characters, underscore or period
\item If they start with period, the second character must be a letter
\item A case-sensitive language, i.e., distinct variables {\tt a} and {\tt A}
\item The {\tt source} function is convenient for loading objects like
data, functions, etc.
\item {\tt source("lecture2.R")}
\item However, re-creating some objects (like large data sets or
  complex analyses) might be very time-consuming
\item DON'T DO THIS: R-intro describes saving R objects in {\tt
    .RData} which is automatically re-loaded when you re-launch R in
  that directory: IT IS TOO EASY TO FORGET/MISUSE
\item Save the objects that you want to keep and re-load as needed
\item {\tt saveRDS(object, "object.rds")}
\item {\tt object = readRDS("object.rds")}
\item R loads all objects into memory so be cognizant of their size
\item {\tt object.size(object)}
\item {\tt rm(object); gc()}
\item {\tt ?gc}
\end{itemize} 

\end{frame}

\begin{frame} 
\frametitle{Section 1: View-/edit-ing objects}

\begin{itemize}
\item You can view an object just by typing its name at the
prompt: {\tt $>$ object}
\item However, for large objects, this is not very user-friendly
\item For large objects, it might be better to view/edit them
\item If you just want to view (and NOT edit):\\
%{\tt $>$ write(object, "object.R")}\\
{\tt $>$ sink("object.Rout"); object; sink()}\\
{\tt C-x C-f object.Rout}
\item To edit, start an {\tt emacsclient} by {\tt M-x server-start}
\item {\tt $>$ sink("/dev/null") \#\# shut off the R console}
\item {\tt $>$ ls() \#\# notice that it produces NO output}
\item {\tt $>$ edit(object, "object.R")}
\item When you are done: \textcolor{red}{\tt C-x \#}
\item {\tt $>$ sink() \#\# restore the R console}
\item {\tt $>$ ls() \#\# now produces output as usual}
\item N.B. the file {\tt object.R} remains for your viewing
\end{itemize} 

\end{frame}

\begin{frame} 
\frametitle{Section 2: R is naturally vectorized}

\begin{itemize}
\item All {\tt atomic} types are vectors of length zero or more:\\
numeric, logical, character, complex and raw
\item Roughly in that order of importance
\item {\tt length} function returns their current length
\item Vectors can easily expand by allocatting beyond it
\item {\tt $>$ x = 1} creates a vector of length 1
\item {\tt $>$ (x = 1)} parentheses echo assignment values
\item Like {\tt  $>$ x = 1; x}
\item Or to echo any expression {\tt $>$ 1/x}
\item {\tt  $>$ x[2]=0} expands to length of 2
\item {\tt $>$ x = c(10.4, 5.6, 3.1, 6.4, 21.7)} creates a vector
\item {\tt $>$ ?c}
\item see {\tt lecture3.R}
\end{itemize} 

\end{frame}

\begin{frame} 
\frametitle{Section 2.3: Sequences}

\begin{itemize}
\item Integer sequences: {\tt 1:30} same as {\tt c(1, 2, \dots, 29, 30)}
\item General sequences: {\tt seq(-5, 5, by=0.2)} same as \\
{\tt c(-5.0, -4.8, -4.6, \dots, 4.6, 4.8, 5.0)}
\item See {\tt $>$ ?seq} for all of the possibilities
\end{itemize} 

\end{frame}

\begin{frame}
\frametitle{Section 2.4: Logical vectors}

\begin{itemize}
\item Three possible values: \blue{TRUE}, \red{FALSE} and NA (BEWARE!)
\item The comparison operators: {\tt $<$, $<$=, $>$, $>$=, ==, !=}
\item Two equal signs for equality like C
\item Combining expressions with {\tt |, ||, \&, \&\&}
\item N.B. the difference between ORs {\tt |} and {\tt ||}\\
and the difference between ANDs {\tt \&} and {\tt \&\&}
\item {\tt ||} and {\tt \&\&} SHOULD ONLY BE USED WITH SCALARS\\
they stop evaluating once the result is known\\
so very useful for conditional expressions\\
{\tt FALSE || \textcolor{blue}{TRUE} || FALSE}\\
{\tt TRUE \&\& \textcolor{red}{FALSE} \&\& TRUE}\\
\item {\tt |} and {\tt \&} are for vector operations 
(similar to addition/etc.)
\item Functions that return a scalar from vectors: {\tt any} and {\tt all}
\item {\tt TRUE} can be coerced to 1 and {\tt FALSE} can be coerced to 0\\
% \red{BEWARE: this can be dangerous for indices\\
% consider {\tt a = (x<6)}\\
% {\tt x[a] = c(5.6, 3.1)} uncoerced and {\tt x[1]=10.4} coerced}\\
% \blue{Or make the right coercion: {\tt a = which(x<6)}}
\item see {\tt lecture3.R}
\end{itemize}

\end{frame}

\begin{frame}
\frametitle{Section 2.6: Character vectors}

\begin{itemize}
\item Working with character strings is NOT one of R's strengths
\item But this is true of most languages that statistician's use
\item The focus of statistics is largely on numerical computing
\item Many functions inherited from the AWK language\\
(mainly New AWK or NAWK) from the UNIX family\\
(cloned by the GNU as GAWK)
\item For example, see {\tt sub}, {\tt gsub} and {\tt substr}
\item Three functions commonly used (NOT inherited from AWK):\\ 
{\tt paste0}, {\tt paste}, \textcolor{red}{\tt nchar}
and \textcolor{blue}{\tt sprintf}
\item see {\tt lecture3.R}
\end{itemize}

\end{frame}

\begin{frame}
\frametitle{Section 2.7: Vector indices}

\begin{itemize}
\item Integer indices: {\tt x[1:3]} returns the first three items
\item If there are fewer, then returns {\tt NA}
\item {\tt x[0]} returns nothing
\item {\tt x[-1]} returns everything EXCEPT {\tt x[1]} 
\item Logical indices: {\tt x[!is.na(x)]}
\item BEWARE: due to coercion of complex expressions you can
  accidentally request TRUE as index 1 and FALSE as index 0 
(which is nothing): can be very difficult to debug because
it happens without warning
\item You can protect yourself by enclosing all logical index
expressions within {\tt which} function. See {\tt ?which}
\item {\tt x[which(!is.na(x))]}
\end{itemize}

\end{frame}

\begin{frame}
\frametitle{Section 3: Object attributes}

\begin{itemize}
\item All objects have a {\it mode}: {\tt $>$ mode(x)}
\item But mainly of interest for atomic vectors
\item Some objects have a {\it class}: {\tt $>$ class(x)}
\item A capable function for object structure: {\tt $>$ str(x)}
\item {\tt as.} functions for manual coercion between types
\item {\tt $>$ as.integer(x)}
\item {\tt $>$ as.character(x)}
\item All of an object's attributes, if any, can be listed:
{\tt attributes(x)}
\item They can be extracted: {\tt attr(x, "names")}
\item And set: {\tt attr(x, "names")=letters[1:5]}

\end{itemize}

\end{frame}

\begin{frame} 
\frametitle{Appendices}

\begin{itemize}
\item Appendix A lists a lot of interesting functions that we will learn about
\item But, {\tt attach} is generally frowned upon today
\item A short version of the help provided in Appendix B.1 \\
%(R from Unix command line) 
is generated by {\tt \$ R --help}
\end{itemize}

\end{frame}

\begin{frame} 
\frametitle{Appendix B.4: Scripting with R}

\begin{itemize}
\item It is not about ``Scripting'' per se
\item Rather, asynchronous/background/batch R jobs\\
 i.e., NOT interactive
\item It does provide useful information (which is why it is assigned)
\item Particularly, the function {\tt commandArgs} and {\tt stdin} discussion
\item Generally, I do something like the following 
\item the standard Unix way of doing things with re-direction\\
building on what is shown in Appendix B.1
\item {\tt stdin} is comming from {\tt trees.R}
\item {\tt stdout} and {\tt stderr} are going to {\tt trees.Rout}
\end{itemize}

{ \tt nohup R --no-save $<$ trees.R $>$\& trees.Rout \&}
\end{frame}
\end{document}

\begin{frame}
\frametitle{Matrices}
Mathematically, a matrix is represented as follows.
\begin{align*}
A=\wrap{\begin{array}{cccc}
a_{11} & a_{12} & \cdots & a_{1n} \\
a_{21} & a_{22} & \cdots & a_{2n} \\
\vdots& \vdots & \ddots & \vdots \\
a_{m1} & a_{m2} & \cdots & a_{mn} \\
\end{array}}
\end{align*}
{R} is a column-major language, i.e., matrices are laid out
in consecutive memory locations by traversing the columns:
$\wrap{a_{11}, a_{21}, \., a_{12}, a_{22}, \.}$.  {R} is written in
{C} and {Fortran} where {Fortran} is a column-major language as well.
However, {C} and {C++} are row-major languages, i.e., matrices are
laid out in consecutive memory locations by traversing the rows:
$\wrap{a_{11}, a_{12}, \., a_{21}, a_{22}, \.}$.  So, if you have
written an {R} function in {C}/{C++}, then pass it $A$ transpose
rather than $A$ itself, i.e., {\tt t(A)} as {\tt t()} is the
R transpose function.
\end{frame}

\begin{frame}
\frametitle{Section 5.7: Matrices (two-dimensional arrays)}

\begin{itemize}
\item {\tt A=matrix(DATA, nrow=n, ncol=p)} creates {\tt A$_{n\times p}$}\\
{\tt DATA} goes down the columns filling in column-major order
\item {\tt A=matrix(DATA, nrow=n, ncol=p, byrow=TRUE)}\\
{\tt DATA} goes across the rows filling in row-major order\\
but {\tt A$_{n\times p}$} is still a column-major matrix!
\item {\tt diag(n)} creates an identity matrix {\tt n} by {\tt n}
%\item {\tt state.x77} is a matrix 50 by 8
%\item {\tt Matrix} package is one of the Recommended packages that
%  comes with your R installation
\item {\tt A \%*\% B} is matrix multiplication
\item {\tt crossprod(A, B)} is the cross-product\\
like {\tt t(A) \%*\% B} but more efficient
\item {\tt crossprod(A)} is like {\tt t(A) \%*\% A}
\item {\tt solve(A, b)} solves the equation {\tt b = A \%*\% x } for {\tt x}
\item {\tt solve(A)} creates {\tt A$^{-1}$}, but matrix inversion can be unstable
\item For example, when calculating {\tt b \%*\% A$^{-1}$ \%*\% b}, it is 
more numerically stable as {\tt b \%*\% solve(A, b) }
\end{itemize}

\end{frame}

\begin{frame}
\frametitle{Section 5.4: The recycling rule for mixed vector and array
  (or matrix) arithmetic} 

The precise rule affecting element by element mixed calculations with
vectors and arrays is somewhat quirky and hard to find in the
references.
\begin{itemize}
\item The expression is scanned from left to right.
\item Any short vector operands are extended by recycling their values
  until they match the size of any other operands.
\item As long as short vectors and arrays only are encountered, the
  arrays must all have the same dim attribute or an error results.
\item Any vector operand longer than a matrix or array operand
  generates an error.
\item If array structures are present and no error or coercion to
  vector has been precipitated, the result is an array structure with
  the common dim attribute of its array operands.
\end{itemize}

\end{frame}

\begin{frame}
\frametitle{Section 5.4: The recycling rule for mixed vector and array
  (or matrix) arithmetic} 
Examples
\begin{align*}
A_{n\times p} - b_n & = A-B \where B=\wrap{b_n^1, \dots, b_n^p}_{n \times p} \\
& \mbox{Recycling is what you likely intended} \\
A_{n\times p} - b_p & = A-C %\where C=\wrap{b^1, \dots, b^n}_{n \times p} \\
\mbox{\ difficult to demonstrate in general}\\
& \mbox{so assume $n=4,\ p=3$ and\ } b=\wrap{b_1, b_2, b_3}\\
& \mbox{which results in\ } C=\wrap{\begin{array}{ccc} 
b_1 & b_2 & b_3 \\
b_2 & b_3 & b_1 \\
b_3 & b_1 & b_2 \\
b_1 & b_2 & b_3
\end{array}} \\
& \mbox{Recycling is NOT what you likely intended} \\
A^t_{p \times n} - b_p & = A^t- D \where D=\wrap{b_p^1, \dots, b_p^n}_{p \times n} \\
& \mbox{Recycling is what you likely intended} \\
\end{align*}
\end{frame}

\begin{frame}
\frametitle{Section 6: Lists}

%{\tt $>$ Lst = list(husband="Fred", wife="Mary", no.children=3,\\
%\qquad child.ages=c(4, 7, 9))}
{\tt case = list(type="breast", ICDO="C50.3", stage="IIA",\\
\qquad sex="F", side="L", birth=as.Date("1970/04/05"),\\
\qquad diag=as.Date("2019/12/05"), \\
\qquad surg=as.Date("2020/03/02"), \\
\qquad T=list(P="T1a", size=0.45),\\
\qquad N=list(P="N1", C="N0", surg="SLNB",\\
\qquad positive=3, removed=7, max=0.25), M="M0")}
\begin{itemize}
\item A {\it list} is an object consisting of an ordered collection of
  objects of singular or diverse types known as {\it components}
\item Components are {\it numbered} and can always be referred to as such:
{\tt case[[1]], case[[2]], case[[3]], \dots}
\item Components of lists may also be {\it named}, and in this case
  the component may be referred to either by giving the component name
  or as a character string
\item So, the first component can be referenced by
{\tt case[[1]]} or {\tt case\$type} or {\tt case\$"type"}
\item Or, rarely, as a list itself called a {\it sublist}: {\tt case[1]}
\end{itemize}

\end{frame}

\begin{frame}
\frametitle{Comma separated values (CSV) files}

\begin{itemize}
\item CSV files are a standard that arose from Fortran data files
evolving late-60s to mid-70s: roughly what we have today
\item Sadly, subject matter experts (SME) LOVE Excel and CSVs\\ 
(we tell them to stop, but they don't listen)
\item So CSV files are the de facto standard for data exchange
\item But CSVs from spreadsheets are \textcolor{red}{not type safe}\\
 e.g., a numeric column can have text entries, etc.
\item In the column for weights, they can type {\tt deceased} in a cell
\item So the {\tt weight} column will be character instead of numeric
\item Although, theoretically, they are trivial to import,\\
the practice of SMEs will make your life difficult
\item There does not appear to be a complete solution to this problem
with R (other than manual editing)
\item SAS has a few tricks that might help as we will see later
\end{itemize}

\end{frame}

\begin{frame}
\frametitle{American College of Surgeons National Trauma Data Bank (NTDB)}

In the US Code of Federal Regulations (CFR), 45 CFR 46.101 (title 45,
Public Welfare, section 46.101),
%Per 45 Code of Federal Regulations (CFR) 46.101, 
research using certain publicly available data sets does not involve
``human subjects''. The data contained within these specific data sets
are neither identifiable nor private and thus do not meet the federal
definition of ``human subject'' as defined in 45 CFR 46.102.  These
research projects do not need to be reviewed and approved by the
Institutional Review Board, (IRB). MCW's Human Research Protection
Program (HRPP) has a list of their recognized ``publicly available data sets''.\\
% or ``restricted use data sets''.\\
\url{https://www.mcw.edu/departments/human-research-protection-program/researchers/smartform-instructions/public-data-sets}
\end{frame}

\begin{frame}
\frametitle{NTDB and data frames}

\begin{itemize}
\item the NTDB has all de-identified trauma admissions for those\\
 aged 65 and older (restricted to 89 or less here)\\
at all designated trauma center hospitals in the US\\
about 1,000,000 admissions in the 2014 edition
\item the NTDB does NOT come as a CSV
\item but I created a CSV for the first 18 trauma centers\\
{\tt /data/shared/04224/NTDB/NTDB18.csv}\\
about 50,000 admissions:
there is an (incomplete) data dictionary in the same directory
\item this is for educational purposes ONLY (HIPAA exception)
\item the {\tt read.csv} function creates a {\tt data.frame}
\item a {\tt data.frame} is the typical R data set object\\ 
essentially a list of vectors of the same length
\item like a matrix where the rows are lists\\
and the columns are vectors of \textcolor{red}{potentially diverse types}
\item the {\tt subset} function creates a subset {\tt data.frame} 
\end{itemize}
\end{frame}

\begin{frame}
\frametitle{Section 8: Probability distributions }

\begin{itemize}
\item Consider a discrete random variable (RV): $Y \in \{a, \ldots, b\} $
\item Probability density function (PDF):
  $f(c) = \P{Y=c}\I{c \in \{a, \ldots, b\}} \in [0, 1]$
\item Cumulative density function (CDF): $F(c) = \P{Y\le c}= 
\sum_{y \in \{a, \ldots, b\} \cap y\le c} f(y) \in [0, 1]$
\item $\P{c<Y \le d}=\P{Y \le d}-\P{Y \le c}=F(d)-F(c)$
\item Consider a continuous RV: $Y \in (a, b)$
\item PDF: $f(c)\I{c \in (a, b)} \ge 0 $
\item CDF: $F(c) = %\P{Y\le c}= 
\P{Y < c}=
\I{a < c} \int_{a}^{\min(b,c)} f(y) \mathrm{d} y \in [0, 1]$
\item $\P{c<Y < d}=\P{Y < d}-\P{Y < c}=F(d)-F(c)$
\item For either a discrete or a continuous RV $Y$\\
Inverse CDF: $p=F(c) \Rightarrow c = F^{-1}(p) = F^{-1}(F(c))$
\end{itemize}

\end{frame} 

\begin{frame}
\frametitle{Pseudo-random numbers and Monte Carlo integration }

\begin{itemize}
\item CDFs can be used for many integral/summation calculations
\item The Expectation of a function of a random variable, $g(Y)$\\ 
Discrete RV: $\E{g(Y)} = \sum_{y \in \{a, \ldots, b\}} g(y) f(y)$\\ 
Continuous RV: $\E{g(Y)} = \int_{a}^{b} g(y) f(y) \mathrm{d} y$
\item Mean $=\E{Y}=\mu$, i.e., $g(Y)=Y$, the identity function
\item Variance $=\E{Y^2}-\E{Y}^2=\sd^2$
\item For some $g(Y)$ functions, there is no closed form solution
\item Monte Carlo Integration: 
generate a {\it large} sample of pseudo-random numbers $Y_1, \dots, Y_N$
and approximate \\
$\E{g(Y)} \approx N^{-1}\sum_{i=1}^N g(Y_i)$ 
\end{itemize}

\end{frame} 

\begin{frame}
\frametitle{Section 8: Probability distributions }

\begin{itemize}
\item The PDF R function is {\tt dxxx} for density
\item The CDF R function is {\tt pxxx} for probability
\item The Inverse CDF R function is {\tt qxxx} for quantiles
\item The Random Number R function is {\tt rxxx} for random\\
(paired with \textcolor{red}{\tt set.seed} for reproducibility)
\item {\tt xxx} is one of the following suffixes:\\
{\tt beta, binom, cauchy, chisq, exp, f, gamma, geom, hyper, lnorm, logis, nbinom, norm, pois, signrank, t, unif, weibull, wilcox}
\item These functions mean that we don't need all of those tables at
  the end of our textbooks
\item We can calculate probabilities,
  summations/integrals, etc.\ for these distributions with R
 \end{itemize}

\end{frame} 

\begin{frame}
\frametitle{Probability and discrete RVs: sports examples}
\begin{itemize}
\item Coach K and the Duke Blue Devils college basketball team
have been to 12 NCAA Final Fours
\item During these appearances, they have won 5 NCAA Championships
\item What is the probability that they would have won less than 5?
\item Aaron Rodgers and the Green Bay Packers football team are\\ 
1 for 4 in the NFC Championship Playoff Game?
\item What is the probability that they would have won more than 1 game?
\end{itemize}
\end{frame}

\begin{frame}
\frametitle{Section 8.3: Hypothesis testing}

\begin{itemize}
\item This course is a co-requisite with\\
 Statistical Models and Methods I
\item The intent is not to teach you statistics, but to be prepared
\item For example, what hypothesis tests are available?
\item Here's a list of those in the {\tt stats} package
\item {\tt $>$ ls("package:stats", pattern="test\$")}
 \end{itemize}

\end{frame} 

\end{document}
