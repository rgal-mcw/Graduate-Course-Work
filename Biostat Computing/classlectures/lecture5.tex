\documentclass[11pt,pdftex,dvipsnames,usenames,helvetica]{beamer}
\setbeameroption{show notes}
\usepackage[round]{natbib}
\usepackage{textcomp}
\usepackage{amsmath}
\usepackage{amssymb}
\usepackage{graphicx}
\DeclareGraphicsExtensions{.ps,.eps,.pdf,.jpg,.png}
\usefonttheme[onlymath]{serif}
\usepackage[cmintegrals,cmbraces]{newtxmath}
\usetheme{default}
\usecolortheme{dove}

\usepackage{cancel}
\usepackage{tikz}
\usepackage[english]{babel}

\usepackage{verbatim}
\usepackage{color}
\usepackage{pgf} %portable graphics format
\usepackage[autobold]{statex2}
\mode<presentation>
{
  %\usetheme{Warsaw}
  % or ...
  \setbeamercovered{transparent}
  % or whatever (possibly just delete it)

  \setbeamertemplate{navigation symbols}{}
  \usefonttheme[onlysmall]{structurebold}
  %\usefonttheme{structurebold}
}
\addtobeamertemplate{navigation symbols}{}{%
    \usebeamerfont{footline}%
  \setbeamertemplate{navigation symbols}{}
    \usebeamercolor[fg]{footline}%
    \hspace{1em}%
    \insertframenumber/\inserttotalframenumber
}
\newcommand*{\red}[1]{\textcolor{red}{#1}}
\newcommand*{\blue}[1]{\textcolor{blue}{#1}}

\begin{document}
\boldmath
% 0. Intro

\begin{frame}
\frametitle{Graduate School Class Reminders}

\begin{itemize}
% \item Keep your facial covering on, including over your nose
\item Maintain six feet of distancing
\item Please sit in the same chair each class time
\item Observe entry/exit doors as marked
\item Use hand sanitizer when you enter/exit the classroom
\item Use a disinfectant wipe/spray to wipe down your learning space
  before and after class
\item Media Services: 414 955-4357 option 2
%\item START RECORDING
\end{itemize}

\end{frame}

\begin{frame}
\frametitle{Documentation on the web}

\begin{itemize}
\item CRAN: \url{http://cran.r-project.org}
\item R manuals: \url{https://cran.r-project.org/manuals.html}
\item SAS: \url{http://support.sas.com/documentation}
\item Step-by-Step Programming with Base SAS 9.4 (SbS): \\
\url{https://documentation.sas.com/api/docsets/basess/9.4/content/basess.pdf}
\item SAS 9.4 Programmer s Guide: Essentials (PGE): \\
\url{https://documentation.sas.com/api/docsets/lepg/9.4/content/lepg.pdf}
\item Wiki: \url{https://wiki.biostat.mcw.edu} 
\textcolor{red}{(MCW/VPN)}
\item {\tt lattice}: \url{http://lmdvr.r-forge.r-project.org/figures/figures.html}
\end{itemize}

\end{frame}

\begin{comment}
\begin{frame}[fragile]
\frametitle{HW part 1 (mid-term): NTDB time series with {\tt xyplot}}

\begin{itemize}
\item Create several models for {\tt dead$~$gcstot+age}
\item For {\it nonparametric} flexible fits, estimate a series
of models based on data frame subsets
\item Each of these separate models for levels of the following
variables: {\tt male}, {\tt tbi\_icd9} and {\tt year}, i.e., 
32 models total
\item Plot time series for the probability of death by {\tt year}
\item The $x$-axis is {\tt year}
\item The $y$-axis is the probability of death
\item With 4 lines, for the 4 settings of {\tt male} and {\tt tbi\_icd9} 
\item With multiple panels for the following settings\\
 5 rows for {\tt gcstot} values of {\tt (3, 6, 9, 12, 15)} and\\
 3 columns for age {\tt 65, 75, 85}
\item Or a more visually appealing layout of your choice
\end{itemize}

\end{frame}

\begin{frame}[fragile]
\frametitle{HW part 2 (mid-term): NTDB 3D plots}

\begin{itemize}
\item Based on the same models as above
\item Plot 3D contours with {\tt contourplot}
\item Plot 3D heat maps with {\tt levelplot}
\item Plot 3D surfaces with {\tt wireframe}
\item The $x$-axis is {\tt gcstot}
\item The $y$-axis is {\tt age}
\item The $z$-axis is the probability of death
\item For the 4 settings of {\tt male} and {\tt tbi\_icd9} on each page,\\
create 8 pages for the years %2008, 2011 and 2014
\item Or a more visually appealing layout of your choice
\end{itemize}

\end{frame}
\end{comment}

\begin{frame}[fragile]
\frametitle{Graphical exploration with R: two-headed monster}

\begin{itemize}
\item There are two graphics packages for R that have 
a similar syntax which makes learning both of them convenient
\item \textcolor{red}{The {\tt graphics} package for routine daily usage}
is one of the {\it base} packages as described in Section 12
\item Easy to use, it can create high-quality
graphical figures, but there are limits to their customizability
\item You can see the documentation with {\tt library(help=graphics)}
\item Depends on the {\tt grDevices} package which has support for
devices, colors, fonts, etc.
\end{itemize}

\end{frame}

\begin{frame}[fragile]
\frametitle{Graphical exploration with R: two-headed monster}

\begin{itemize}
\item The {\tt lattice} package which is one of the {\it recommended}
packages can create more appealing customizable graphical figures of
publication and grant proposal quality
\item It builds upon {\tt grDevices} and {\tt graphics}
\item With some added capabilities in {\tt latticeExtra}
\item You can see the documentation with {\tt library(help=lattice)}
\item But, \textcolor{blue}{the {\tt lattice} package documentation is 
voluminous} and it can take quite a while to read and understand
\item Yet, the examples are very helpful; there are extra examples
online that will dispell some of the mystery  
\url{http://lmdvr.r-forge.r-project.org/figures/figures.html}
\end{itemize}
\end{frame}

\begin{frame}[fragile]
\frametitle{Graphical exploration with R: two-headed monster}
\begin{itemize}
\item DO NOT USE {\tt ggplot2}!  
\item It is unacceptable for the following reasons
\item The graphs are terrible: a gray background by default?
\item {\tt ggplot2} package doesn't come with R
\item Its syntax doesn't resemble {\tt graphics/lattice}
\item It has no 3D capabilities
%\item  {\tt latticeExtra} 
\end{itemize}

\end{frame}

\begin{frame}[fragile]
\frametitle{Graphical exploration with R: two-headed monster}
\begin{itemize}
\item Common types of plots and their high-level functions
\end{itemize}
\begin{tabular}{l|lll}
Type of plot    & Dimensions & {\tt graphics} & {\tt lattice} \\ \hline
%Bar chart       & {\tt } & {\tt barchart}
Box and whisker & 1 & {\tt boxplot}  & {\tt bwplot} \\
%Dot chart       & {\tt dotchart} & {\tt dotplot} \\
Histogram       & 1 & {\tt hist}     & {\tt histogram} \\
Density         & 1 & {\tt plot.density} & {\tt densityplot} \\
Quantile-quantile & 1 & {\tt qqnorm}   & {\tt qqmath} \\
QQ              & 2 & {\tt qqplot}   & {\tt qq} \\
Scatter         & 2 & {\tt plot} & {\tt xyplot} \\
Contour         & 3 & {\tt contour}  & {\tt contourplot} \\
Heat map        & 3 & {\tt image}    & {\tt levelplot} \\
Surface         & 3 & {\tt persp}    & {\tt wireframe} \\
\end{tabular}

\end{frame}

\begin{frame}[fragile]
\frametitle{Graphical exploration with R: the {\tt graphics} package}

\begin{itemize}
\item Let's explore the high-level {\tt plot} function 
\item Many objects types have their own generic functions\\ {\tt $>$
    plot(object)} will call the function {\tt plot.CLASS} where the
  {\tt object} is of type {\tt CLASS} like {\tt plot.density}
\item But, we are more concerned with the default: {\tt plot.default}
\item Here the arguments of primary interest
that are shared/comparable between {\tt graphics} and {\tt lattice}
% \item \textcolor{red}{\tt col} is the color to be used for the plot
%   points/lines/etc.
% \item It can be a known color string: see {\tt colors}
% \item Or it can be an integer marker number
% \item This is controlled by the color palette: see {\tt $>$ ?palette}
\end{itemize}
\begin{verbatim}
plot(x, y = NULL, col = 1, 
     type = "p", cex = 1, pch = 1, lty = 1, lwd = 1,
     xlim = NULL, ylim = NULL,
     xlab = NULL, ylab = NULL)
\end{verbatim}
\end{frame}

\begin{frame}[fragile]
\frametitle{Graphical exploration with R: the {\tt graphics} package}
\begin{verbatim}
plot(x, y = NULL, col = 1, 
     type = "p", cex = 1, pch = 1, lty = 1, lwd = 1,
     xlim = NULL, ylim = NULL,
     xlab = NULL, ylab = NULL)
\end{verbatim}
\begin{itemize}
\item {\tt x} is a vector for the $x$-axis
\item {\tt y} same thing for the $y$-axis\\
if {\tt y} is not specified, then {\tt x} is plotted as a time series
\item common {\tt type} settings: {\tt "p"} for points, {\tt "l"} for lines,\\ 
{\tt "b"} for both points and lines, {\tt "h"} for vertical lines,\\
{\tt "s"} for stair steps/survival functions and\\ 
{\tt "n"} does not produce anything
\item {\tt cex} for the size of points and {\tt pch} for the kind
\item specify the line type by {\tt lty}
and the line width by {\tt lwd}
%\item {\tt col} specifies a color from the current palette: see {\tt $>$ ?palette}
\item {\tt xlim} is a length 2 vector for the limits of the $x$-axis
%\item {\tt ylim} same thing for the $y$-axis
\item {\tt xlab} is the label for the $x$-axis: 
either a character string\\ or an {\tt expression}
see {\tt $>$ ?plotmath}
\item {\tt ylim/ylab} same thing for the $y$-axis
\end{itemize}

\end{frame}

\begin{frame}[fragile]
\frametitle{Graphical exploration with R: the {\tt graphics} package}
\begin{verbatim}
plot(x, y = NULL, col = 1, 
     type = "p", cex = 1, pch = 1, lty = 1, lwd = 1,
     xlim = NULL, ylim = NULL,
     xlab = NULL, ylab = NULL)
\end{verbatim}
\begin{itemize}
\item \textcolor{red}{\tt col} is the color to be used for the plot
  points/lines/etc.
\item A color's name by character string: see {\tt
    colors} for a list
\item Or it can be an integer {\it marker} number
\item This is based on the current color palette which
can be customized by the {\tt palette} function: default markers
\item \blue{Choosing good colors can be tricky: stick to the powers of 2}
\begin{tabular}{ll|ll|ll}
{\it 0}$\,=2^{-\infty}$ & {\it white} &
{\it 1}$\,=2^0$ & {\it black} &
\textcolor{red}{\it 2$\,=2^1$} & \textcolor{red}{\it red} \\
3 & green & 
\textcolor{blue}{\it 4$\,=2^2$} & \textcolor{blue}{\it blue} &
5 & cyan \\
6 & magenta & 7 & yellow &
\textcolor{gray}{\it 8$\,=2^3$} & \textcolor{gray}{\it gray} \\
\end{tabular}
\begin{comment}
\begin{tabular}{ll|ll|ll|ll}
{\it 1}$\,=2^0$ & {\it black} &
\textcolor{red}{\it 2$\,=2^1$} & \textcolor{red}{\it red} \\
3 & green &
\textcolor{blue}{\it 4$\,=2^2$} & \textcolor{blue}{\it blue} &
5 & cyan & 6 & magenta \\
7 & yellow &
\textcolor{gray}{\it 8$\,=2^3$} & \textcolor{gray}{\it gray} \\
\end{tabular}
\end{comment}
\item If you need more, then see {\tt bass/colors.R}
\end{itemize}
\end{frame}

\begin{frame}[fragile]
\frametitle{Graphical exploration with R: the {\tt graphics} package}
\begin{itemize}
\item Interactively, {\tt plot} will create a graphics window with a panel
\item Or multiple {\tt plot} calls yielding multiple 
panels via {\tt par(mfcol=c(rows, cols))} or\\
 {\tt par(mfrow=c(rows, cols))}\\
%panels via {\tt par(mfcol=c(m, n))} or {\tt par(mfrow=c(m, n))}\\
but often with very poor quality: see {\tt lecture5.R}
\item Low-level functions that overlay graphics on this panel follow
\item Those that are self-explanatory and
\textcolor{blue}{similar} to {\tt lattice}
\item {\tt lines(x, y=NULL, col=1, lty=1, lwd=1)} 
\item {\tt points(x, y=NULL, col=1, pch=1, cex=1)} 
\item {\tt text(x, y=NULL, labels, col=1, cex=1, pos=NULL)} \\
\begin{tabular}{rcc}
\multicolumn{1}{l}{pos} & 3 &  \\
2   & (x, y) & 4 \\
    & 1 \\
\end{tabular} positions go clock-wise starting at 6 o'clock
\item {\tt legend} is \textcolor{red}{NOT similar} to {\tt lattice} 
%adj = c(0.5, 0.5), pos = NULL, offset = 0.5
\item {\tt legend(x, y=NULL, col=1, legend, lty=1, lwd=1,\\ 
\quad pch=1, cex=1, horiz=FALSE)}
\end{itemize}
\end{frame}

\begin{frame}[fragile]
\frametitle{Graphical exploration with R: the {\tt graphics} package}
\begin{itemize}
\item This is the same for the {\tt lattice} package
\item {\tt $>$ ?Devices} lists the graphical file formats supported
\item We are going to restrict our attention to\\
 Adobe Portable Document Format (PDF)
\item In an interactive R session, you can capture what is in
the graphics window at any time by (in batch, this is an error)\\
{\tt $>$ dev.copy2pdf(file="FILENAME.pdf")}
\item In an interactive or batch R session, you can create 
a graphics file without a graphics window
\item
\begin{verbatim}
pdf(file="FILENAME.pdf")
PLOTTING STATEMENTS
dev.off()
\end{verbatim}
\item The latter can create graphics files with multiple pages\\
  each {\tt plot} creates a page (except for multiple panels where
  each of these is a page)
\end{itemize}
\end{frame}

\begin{frame}[fragile]
\frametitle{Graphical exploration with R: the {\tt lattice} package}
\begin{verbatim}
xyplot(formula, data, groups = NULL, col = 1,
     type = "p", cex = 1, pch = 1, lty = 1, lwd = 1,
     xlim = NULL, ylim = NULL, xlab = NULL, ylab = NULL, 
     strip = strip.custom(strip.names = FALSE),
     layout = c(cols, rows), as.table = FALSE)
\end{verbatim}
\begin{itemize}
\item \textcolor{blue}{\tt formula} of the form {\tt y$~$x} for a single panel \\
  or {\tt y$~$x|g1*g2*...} for panels of plots conditioned on the
  variables {\tt g1*g2*...} which are typically
  factors\\
  if not factors, then specify {\tt factor(g1)*factor(g2)*...}
%  bug or feature? sometimes {\tt y$~$x|g1} will not be appealing,\\ but
%  {\tt y$~$x|g1*g2} is more attractive even if {\tt g2} is a constant
\item {\tt data} is a data frame with {\tt x} and {\tt y}, etc.
\item \textcolor{red}{\tt groups=z} overlays data within the same panel
based on different values of the variable {\tt z}
\item {\tt layout} controls how multiple panels are configured, if any
\item {\tt as.table} controls the order multiple plans are drawn\\
{\tt as.table = TRUE} seems more logical than the default
\item {\tt strip} controls how the panels are labeled \\
{\tt strip.names = TRUE} seems more logical than the default
%\item other arguments are the same as the {\tt graphics} package
\end{itemize}

\end{frame}

\begin{frame}[fragile]
\frametitle{Graphical exploration with R: the {\tt lattice} package}
\begin{itemize}
\item Interactively, {\tt xyplot} will create a graphics window 
\item Low-level functions can overlay graphics on this panel
\item Those that are \textcolor{blue}{similar} to {\tt graphics} follow\\
typically, just prefix an {\tt l} on their names for lattice
\item {\tt llines(x, y=NULL, col=1, lty=1, lwd=1)} adds lines
\item {\tt lpoints(x, y=NULL, col=1, pch=1, cex=1)} adds points
\item {\tt ltext(x, y=NULL, col=1, labels, cex=1,\\ 
\quad %adj=c(0.5, 0.5), 
pos=NULL, offset=0.5)} adds text
\item And there is an additional one like {\tt plot} itself \\
  {\tt lplot.xy(list(x=x, y=y), col=1, type="p",\\
    \quad cex=1, pch=1, lty=1,
    lwd=1)} %with coordinates {\tt xy\$x} and {\tt xy\$y}
%\item \textcolor{red}{But, you need to call these from 
%a {\tt panel} function}\\
%see {\tt lecture5.R} FIXME
\end{itemize}
\begin{verbatim}
xyplot(formula, data, ## the easy way
panel=function(...){ panel.abline(h=0);panel.xyplot(...)})  
xyplot(formula, data) ## or the hard way
update(trellis.last.object(),
       panel=function(...) {
           i=panel.number() ## call panel.FUNCTIONS and
           lFUNCTIONS })
\end{verbatim}
\end{frame}

\begin{frame}[fragile]
\frametitle{Graphical exploration with R: the {\tt lattice} package}
\begin{itemize}
\item As we saw, there are three 3D functions in {\tt graphics}:\\
{\tt contour}, {\tt image} and {\tt persp}
\item Each of these functions has an interface like 
{\tt NAME(x, y, Z)} where {\tt x} and {\tt y} are vectors
and {\tt Z} is a matrix
\item This is a confusing interface to say the least\\
 i.e., how does {\tt Z} relate to {\tt x} and {\tt y}?
\item {\tt lattice} has a more intuitive interface for the
  corresponding functions: {\tt contourplot}, {\tt levelplot} and {\tt
    wireframe} as follows (with arguments analogous to {\tt xyplot})
\item {\tt NAME(z$~$x*y|g1*g2*..., data)} where {\tt x}, {\tt y} and
{\tt z} are all vectors in the data frame so their relationship is obvious
%{\tt data} (as are {\tt g1, g2, ...})
\item Many of the same arguments as {\tt xyplot}
\item {\tt groups} is a notable exception
\end{itemize}
\end{frame}

\begin{frame}[fragile]
\frametitle{Graphical exploration with R: the {\tt lattice} package}
\begin{itemize}
\item There is a {\tt legend} argument but it seems difficult to use\\
see {\tt key} instead
\item {\tt key=list(border="black", x=x, y=y, text=list(...), ..., lines=list(...))}\\
or {\tt ..., points=list(...))} \\
where {\tt x} and {\tt y} in [0, 1] of the displayed area
\item {\tt xlim} and {\tt ylim} are too simplistic for many needs\\
{\tt scales} is more flexible
\item {\tt scales=list(\\
x=list(limits=c(low, high), log=TRUE),
y=list(at=c(1, 2, 4, 8),\\ \quad labels=c("a", "b", "c", "d")))}
\item {\tt at} are the tick marks drawn: can be increasing or decreasing
\item {\tt labels} are the values to display there
\end{itemize}
\end{frame}

\begin{frame}[fragile]
\frametitle{Graphical exploration with R: the {\tt lattice} package}
\begin{itemize}
\item {\tt trellis.par.get()} to show default settings
\item and {\tt trellis.par.set()} to alter them
\item for example, slide 13 shows the default for {\tt xyplot}
is {\tt col = 1}\\
however, that is NOT right: more complicated than that
\end{itemize}
\begin{verbatim}
trellis.par.get("plot.symbol")
a$col = 1
trellis.par.set("plot.symbol", a)
\end{verbatim}
\end{frame}

\begin{frame}[fragile]
\frametitle{Graphical exploration with R: the {\tt latticeExtra} package}
\begin{itemize}
%\item Also see the {\tt latticeExtra} package
\item You can overlay two or more {\tt trellis} objects
\item For example, {\tt doubleYScale} to create a second $y$-axis
\item Combine two objects with the {\tt as.layer} function
\item Combine multiple objects with the {\tt c} function
\end{itemize}
\end{frame}

\end{document}
