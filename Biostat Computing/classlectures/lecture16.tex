\documentclass[11pt,pdftex,dvipsnames,usenames,helvetica]{beamer}
\setbeameroption{show notes}
\usepackage[round]{natbib}
\usepackage{textcomp}
\usepackage{amsmath}
\usepackage{amssymb}
\usepackage{graphicx}
\DeclareGraphicsExtensions{.ps,.eps,.pdf,.jpg,.png}
\usefonttheme[onlymath]{serif}
\usepackage[cmintegrals,cmbraces]{newtxmath}
\usetheme{default}
\usecolortheme{dove}

\usepackage{cancel}
\usepackage{tikz}
\usepackage[english]{babel}

\usepackage{verbatim}
\usepackage{color}
\usepackage{pgf} %portable graphics format
\usepackage[autobold]{statex2}
\mode<presentation>
{
  %\usetheme{Warsaw}
  % or ...
  \setbeamercovered{transparent}
  % or whatever (possibly just delete it)

  \setbeamertemplate{navigation symbols}{}
  \usefonttheme[onlysmall]{structurebold}
  %\usefonttheme{structurebold}
}
\addtobeamertemplate{navigation symbols}{}{%
    \usebeamerfont{footline}%
  \setbeamertemplate{navigation symbols}{}
    \usebeamercolor[fg]{footline}%
    \hspace{1em}%
    \insertframenumber/\inserttotalframenumber
}
\newcommand*{\red}[1]{\textcolor{red}{#1}}
\newcommand*{\blue}[1]{\textcolor{blue}{#1}}

\begin{document}
\boldmath
% 0. Intro

\begin{frame}
\frametitle{Graduate School Class Reminders}

\begin{itemize}
% \item Keep your facial covering on, including over your nose
\item Maintain six feet of distancing
\item Please sit in the same chair each class time
\item Observe entry/exit doors as marked
\item Use hand sanitizer when you enter/exit the classroom
\item Use a disinfectant wipe/spray to wipe down your learning space
  before and after class
\item Media Services: 414 955-4357 option 2
%\item START RECORDING
\end{itemize}

\end{frame}

\begin{frame}
\frametitle{Documentation on the web}

\begin{itemize}
\item CRAN: \url{http://cran.r-project.org}
\item R manuals: \url{https://cran.r-project.org/manuals.html}
\item SAS: \url{http://support.sas.com/documentation}
\item 
\textcolor{blue}{SAS 9.3: \url{https://support.sas.com/en/documentation/documentation-for-SAS-93-and-earlier.html}}
\item Step-by-Step Programming with Base SAS 9.4 (SbS): \\
\url{https://documentation.sas.com/api/docsets/basess/9.4/content/basess.pdf}
\item SAS 9.4 Programmer s Guide: Essentials (PGE): \\
\url{https://documentation.sas.com/api/docsets/lepg/9.4/content/lepg.pdf}
\item Wiki: \url{https://wiki.biostat.mcw.edu} 
\textcolor{red}{(MCW/VPN)}
\end{itemize}

\end{frame}

\begin{frame}[fragile]
\frametitle{SAS settings}
\begin{itemize}
\item We have already discussed {\tt autoexec.sas}
\item The following settings in  {\tt autoexec.sas}
were chosen as ideal for debugging errors
without being overly burdensome\\
{\tt options \textcolor{red}{nofmterr} 
\textcolor{blue}{mprint} errors=max noovp dkrocond=error;}
\item {\tt \textcolor{red}{nofmterr}} turns off errors
when a user-defined format is not found
since it happens far too often
\item {\tt \textcolor{blue}{mprint}} shows code generated by SAS macros
in the {\tt .log}
\item There are global config files 
which you can find on our system
{\tt /usr/local/sas/SAS18w47/SASHome/SASFoundation/9.4}
\item There is {\tt setinit.sas} that gives you the
annual expiration date and a list of SAS products installed
\item There is {\tt sasv9.cfg} that is provided by 
SAS which should not be changed
\item There is {\tt sasv9\_local.cfg} where
local settings can be found like {\tt -sasautos}
\end{itemize}

\end{frame}

\begin{frame}[fragile]
\frametitle{Debugging SAS programs}
\begin{itemize}
\item A few tips
\item Use the {\tt .log}: check it after every run
\item {\tt F5} will take you to the first error or 
\textcolor{red}{potential error}\\
Potential errors are {\tt NOTE}s that are likely suspicious\\ 
\textcolor{blue}{as defined by me in the ESS source code}
\item However, don't be complacent and overly rely on {\tt F5}\\
review the {\tt .log} before even considering the {\tt .lst}
\item Likely, the most common error is forgetting a semi-colon
\item Or, accidentally using a colon instead of a semi-colon\\
technically, not a syntax error since SAS uses colons for addresses
(see the {\tt link} statement)\\ 
so the error might appear to be on the next line

\end{itemize}
\end{frame}

\begin{frame}[fragile]
\frametitle{Debugging SAS programs: Potential Errors}
\url{https://github.com/emacs-ess/ESS/blob/a694b2627992bda5489c1b4b5bb750c590aa8d85/lisp/ess-sas-a.el#L732}
\begin{verbatim}
NOTE: MERGE statement has more than one data set with repeats
NOTE: Variable .* is uninitialized.
NOTE: SAS went to a new line when INPUT statement reached past
NOTE 485-185: Informat .* was not found
NOTE: Estimated G matrix is not positive definite.
NOTE: Compressing data set .* increased size by
NOTE: ERROR DETECTED IN ANNOTATE=
WARNING: Apparent symbolic reference .* not resolved.
WARNING: Length of character variable has already been set.
WARNING: Not all variables in the list 
WARNING: RUN statement ignored due to previous errors.
WARNING: Values exist outside the axis range
WARNING: Truncated record.
\end{verbatim}
\end{frame}

\begin{frame}[fragile]
\frametitle{Debugging SAS programs: the {\tt put} statement}
\begin{itemize}
\item Useful for adding debugging info to the {\tt .log}
\item Automatic variable {\it PDV} (program data vector) 
lists that might be useful:
  {\tt \_ALL\_}, {\tt \_NUMERIC\_} and {\tt \_CHARACTER\_}
\item Create your own PDV lists: \textcolor{blue}{\tt VAR1--VARn}
which is all variables from {\tt VAR1} to {\tt VARn}
in order like \textcolor{red}{\tt proc contents VARNUM}
\item {\tt put VALUE1 ...\ VALUEn;} can be variable/PDV lists
\item {\tt VALUEi} is either a character literal, a variable or a
variable array reference\\
but not a numeric literal nor a date literal, etc.
\item {\tt VALUEi} can be followed by a format\\ 
(unless it is a literal)
\item {\tt VALUEi=} puts the variable name before the value
\item Similarly, there is the {\tt \%put ...;} statement
\item And, there is the {\tt list;} statement
for debugging the {\tt input} statement which 
is like {\tt put \_all\_;}
\end{itemize}

\end{frame}

\begin{frame}[fragile]
\frametitle{Debugging SAS programs: suppressing the {\tt .log/.lst}}
\begin{itemize}
\item \textcolor{red}{For the {\tt .log}, this is probably NOT a good idea}
\item However, if it is necessary, you can suppress it with the
{\tt \%\_printto} macro (which relies on {\tt PROC PRINTTO})
\item For UNIX/Linux: {\tt \%\_printto(log=/dev/null)};
\item For Windows: {\tt \%\_printto(log=nul:)};
\item For both: {\tt \%\_printto(log=\%\_null)};
\item To turn the {\tt .log} back on before the end of the program\\
 {\tt \%\_printto()};
\item Similarly, you can turn off the {\tt .lst}\\
\textcolor{blue}{more often desired than turning off the {\tt .log}}
\item For UNIX/Linux: {\tt \%\_printto(/dev/null)};
\item For Windows: {\tt \%\_printto(nul:)};
\item For both: {\tt \%\_printto(\%\_null)};
\item To turn the {\tt .lst} back on before the end of the program\\
 {\tt \%\_printto()};
\end{itemize}
\end{frame}

\begin{frame}[fragile]
\frametitle{Redirecting the {\tt .log/.lst}}
\begin{itemize}
\item For the DATASTEP only: see the \textcolor{red}{\tt file} statement
\item Redirection is also a feature of the {\tt \%\_printto} macro
\item {\tt \%\_printto(log=NAME.log)};
\item To stop redirection of the {\tt .log} before the end of the program\\
 {\tt \%\_printto()};
\item Similarly, you can redirect the {\tt .lst}
\item {\tt \%\_printto(NAME.lst);}
\item To stop redirection of the {\tt .lst} before the end of the program\\
 {\tt \%\_printto()};
\item Notice that I'm keeping the extensions {\tt .log} and {\tt .lst}
so that emacs recognizes the files via their extensions
\item As we have seen, ESS[LOG] uses colors for syntax highlighting
which allows you to get a visual inspection of the source code
\item Currently, ESS[LST] doesn't have much functionality
since it is just text, but that could change
\end{itemize}
\end{frame}

\end{document}
