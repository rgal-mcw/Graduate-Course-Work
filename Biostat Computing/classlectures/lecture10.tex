\documentclass[11pt,pdftex,dvipsnames,usenames,helvetica]{beamer}
\setbeameroption{show notes}
\usepackage[round]{natbib}
\usepackage{textcomp}
\usepackage{amsmath}
\usepackage{amssymb}
\usepackage{graphicx}
\DeclareGraphicsExtensions{.ps,.eps,.pdf,.jpg,.png}
\usefonttheme[onlymath]{serif}
\usepackage[cmintegrals,cmbraces]{newtxmath}
\usetheme{default}
\usecolortheme{dove}

\usepackage{cancel}
\usepackage{tikz}
\usepackage[english]{babel}

\usepackage{verbatim}
\usepackage{color}
\usepackage{pgf} %portable graphics format
\usepackage[autobold]{statex2}
\mode<presentation>
{
  %\usetheme{Warsaw}
  % or ...
  \setbeamercovered{transparent}
  % or whatever (possibly just delete it)

  \setbeamertemplate{navigation symbols}{}
  \usefonttheme[onlysmall]{structurebold}
  %\usefonttheme{structurebold}
}
\addtobeamertemplate{navigation symbols}{}{%
    \usebeamerfont{footline}%
  \setbeamertemplate{navigation symbols}{}
    \usebeamercolor[fg]{footline}%
    \hspace{1em}%
    \insertframenumber/\inserttotalframenumber
}
\newcommand*{\red}[1]{\textcolor{red}{#1}}
\newcommand*{\blue}[1]{\textcolor{blue}{#1}}

\begin{document}
\boldmath
% 0. Intro

\begin{frame}
\frametitle{Graduate School Class Reminders}

\begin{itemize}
% \item Keep your facial covering on, including over your nose
\item Maintain six feet of distancing
\item Please sit in the same chair each class time
\item Observe entry/exit doors as marked
\item Use hand sanitizer when you enter/exit the classroom
\item Use a disinfectant wipe/spray to wipe down your learning space
  before and after class
\item Media Services: 414 955-4357 option 2
%\item START RECORDING
\end{itemize}

\end{frame}

\begin{frame}
\frametitle{Documentation on the web}

\begin{itemize}
\item CRAN: \url{http://cran.r-project.org}
\item R manuals: \url{https://cran.r-project.org/manuals.html}
\item SAS: \url{http://support.sas.com/documentation}
\item 
\textcolor{blue}{SAS 9.3: \url{https://support.sas.com/en/documentation/documentation-for-SAS-93-and-earlier.html}}
\item Step-by-Step Programming with Base SAS 9.4 (SbS): \\
\url{https://documentation.sas.com/api/docsets/basess/9.4/content/basess.pdf}
\item SAS 9.4 Programmer s Guide: Essentials (PGE): \\
\url{https://documentation.sas.com/api/docsets/lepg/9.4/content/lepg.pdf}
\item Wiki: \url{https://wiki.biostat.mcw.edu} 
\textcolor{red}{(MCW/VPN)}
\end{itemize}

\end{frame}

\begin{frame}[fragile]
\frametitle{SAS at MCW and SAS Documentation}
\begin{itemize}
\item SAS is proprietary: NOT open source software
\item You don't buy SAS, you rent it: annual subscription fee
\item MCW has a site-wide license for SAS\\
Biostatistics buys into the license for the Cheese cluster\\
But SAS is not currently available on the RCC cluster\\
Students can get SAS for Windows for free from the help desk\\
\item The SAS online documentation
\url{http://support.sas.com/documentation}
\item SAS was slow to move online and early web pages were terrible
\item But, they have made big improvements recently
\item We are running SAS v9.4: the latest version\\
however, 
\textcolor{red}{the latest documentation is more difficult to traverse}
\item So, I prefer the documentation for SAS v9.3 which 
are now included in the header above
\textcolor{blue}{\url{https://support.sas.com/en/documentation/documentation-for-SAS-93-and-earlier.html}}
\end{itemize}
\end{frame}

\begin{frame}[fragile]
\frametitle{The SAS language}
\begin{itemize}
\item SAS is a fourth-generation language (4GL)\\
4GL's are designed for special purposes
\item SAS is designed for data science\\
analysis, visualization, processing and management
\item A strength is working with large data sets
\item SAS will only use as much memory as you tell it to\\
it is smart and efficient with {\it virtual} memory which
is essentially hard-disk space (unlike R)
\item SAS is case-insensitive and statements end in a semi-colon
\item SAS is made up of multiple complementary products
\item A mixture of the {\it DATASTEP} in Base SAS and\\
{\it PROCEDURES} or {\it PROCS} from Base and \\
other products like SAS/Stat
\item The DATASTEP is vectorized\\
in its own way that's different from R
\item SAS' interactive capabilities are limited
\item Therefore, SAS programs are typically run in batch
in which case the SAS code is 
\textcolor{red}{compiled and then executed: very fast}
\end{itemize}
\end{frame}

\begin{frame}[fragile]
\frametitle{General SAS name rules}
\begin{itemize}
\item SAS names are used for data set names, variable names, format names, 
informat names, etc.
\item A SAS name can contain from one to 32 characters.
\item The first character must be a letter or an underscore 
\item Subsequent characters must be letters, numbers, or underscores
\item Blank spaces cannot appear in SAS names.
\item Exception: format and informat names cannot end in a number and
are usually far shorter but the maximum is/are\\ 
32 characters for numeric and
31 characters for character
\item Special Rules for Variable Names: case preservation\\
\begin{quote}
  SAS remembers the combination of uppercase
  and lowercase letters that you use when you create the variable
  name. Internally, the case of letters does not matter. "CAT," "cat",
  and "Cat" all represent the same variable. But for presentation
  purposes, SAS remembers the initial case of each letter and uses it
  to represent the variable name when printing it.
\end{quote}
\end{itemize}
\end{frame}

\begin{frame}[fragile]
\frametitle{SAS overview}
\begin{itemize}
\item Blocks of code executed together\\
start with {\tt data} for a DATASTEP or {\tt proc} for a PROC\\
end with a {\tt run;} statetment
\item The types of SAS files related to a SAS program\\
a program with file extension {\tt .sas} (like {\tt .R})\\
the output {\tt .lst}  (like {\tt .Rout})\\ and the {\tt .log}
(for debugging: no R equivalent)\\
the output can be in PDF, HTML, MS RTF, etc.,
but typically defaults to text for convenience 
(as we do here)
\item \textcolor{red}{BEWARE}: Don't forget the {\tt run;} statetment\\
that will resulting in a confusing {\tt .log} file
% \item Copy my initialization file\\
% {\tt $>$ cp $~$rsparapa/autoexec.sas $~$}
\item All numeric numbers are floating point: there is no
truly integer type (except for the SAS Macro facility which 
uses true integers and integer arithmetic)
\item Character variables can be quite long nowadays,
but by tradition most character variables are rarely longer than
40 or 200 characters;
and typically much shorter for a large data set
\end{itemize}
\end{frame}

\begin{frame}[fragile]
\frametitle{SAS initialization and comments}
\begin{itemize}
\item The SAS initialization file is {\tt autoexec.sas}
\item SAS looks first in the current directory and then in your
home directory loading ONLY the first {\tt autoexec.sas} it finds
\item Copy mine: {\tt $>$ cp $~$rsparapa/autoexec.sas $~$}
\item SAS has single line comments: \textcolor{red}{RED is commented out}\\
{\tt \textcolor{blue}{*} \textcolor{red}{THIS IS A COMMENT}\textcolor{blue}{;}}
\item Similarly, there are SAS macro single line comments\\
{\tt \textcolor{blue}{\%*} \textcolor{red}{THIS IS A COMMENT}\textcolor{blue}{;}}\\
they do not appear in the {\tt .log} when a SAS macro is run\\
as opposed to the standard single line comments which do
\item SAS has block comments\\
{\tt \textcolor{blue}{/*}\\ 
\textcolor{red}{THIS IS A COMMENT\\
FOR A BLOCK OF CODE}\\
\textcolor{blue}{*/}}
\item You can end a program at any line essentially commenting out
the rest with an {\tt endsas;} statement
\end{itemize}
\end{frame}

\begin{frame}[fragile]
\frametitle{Emacs/ESS and SAS function keys}
\begin{itemize}
\item ESS[SAS] provides fairly comprehensive syntax highlighting\\
(I wrote it) But, it is not bullet-proof and it can be fooled\\
New SAS features are added every year: difficult to keep up\\
\item Definitions reminiscent of the Display Manager which is rarely used today,
but very influential for its time in the 1980s/1990s
\item Key and Approximate Display Manager Equivalent in CAPS
\item \textcolor{red}{\tt TAB} moves margin to the next stop to the right
\item \textcolor{red}{\tt C-TAB} moves margin to the previous stop to the left
\item {\tt F1} help key, same as {\tt C-h}
\item \textcolor{blue}{\tt F2} refresh the buffer with the file contents 
\item \textcolor{red}{\tt F3} SUBMIT 
\item \textcolor{red}{\tt F4} PROGRAM 
\item \textcolor{red}{\tt F5} LOG 
\item \textcolor{red}{\tt F6} OUTPUT 
\item {\tt F7} text file, if any 
\item \textcolor{blue}{\tt F8} go to the *shell* buffer 
\item \textcolor{blue}{\tt F9} open a SAS data set near point for viewing 
\item \textcolor{blue}{\tt F12} open a GSASFILE graphics file near point for viewing 
\end{itemize}
\end{frame}

\begin{frame}[fragile]
\frametitle{Importing A CSV file into a SAS library}
\begin{itemize}
\item SAS libraries are organized by directory
\item For example, see {\tt /data/shared/04224/NTDB/sas}
\item Notice that there is an {\tt autoexec.sas} file here\\
that {\it includes} your ultimate version
\item A good place to create a so-called LIBREF\\ 
{\tt libname LIBREF "DIRECTORY";}
\item Data sets are accessed by one-level or two-level names
\item One-level names are temporary and deleted at job completion
\item The keyword {\tt \_null\_} data set is not created at all
\item Two-level names are permanently saved: {\tt LIBREF.NAME}
\item see {\tt NTDB/sas/import.sas} and {\tt NTDB/sas/dsd.sas}
\end{itemize}
\begin{verbatim}
proc import REPLACE datafile="../NTDB18.csv" 
    out=ntdb.elder;
    GUESSINGROWS=MAX;
run;
proc contents VARNUM;
run;
\end{verbatim}
\end{frame}

\begin{frame}[fragile]
\frametitle{SAS and vectorization: automatic looping}
\begin{itemize}
\item See example {\tt /data/shared/04224/NTDB/sas/vector.sas}
\item The DATASTEP automatically loops over the rows/records
of a data set
\item A convenient form of vectorization 
\end{itemize}
\end{frame}

\begin{frame}[fragile]
\frametitle{HW 1: Importing data with the DATASTEP}
\begin{itemize}
\item Extra Packages for Enterprise Linux (EPEL)\\
is a repository of free software packages: typically binary\\
(already compiled if needed) for Red Hat flavored Linux distros
\item You can see all of the packages installed on gouda by\\
{\tt $>$ yum list installed $\backslash$* \#\# escape the wild card}\\
I have captured this command in the file {\tt /data/shared/04224/gouda.txt} 
\item Chapters 4 and 5 of SbS describe how to read input
data from files in arbitrary formats
\item read this data with SAS into 3 variables\\
the first column is {\tt package}, the second is {\tt version}
and the third is {\tt repo}
\item how many of these packages come from the 
{\tt repo} corresponding to EPEL defined by {\tt "@epel"}?
\item Hint: use the {\tt firstobs} optional parameter
\end{itemize}
\end{frame}

\begin{frame}[fragile]
\frametitle{HW 2: Importing ECGs}
\begin{itemize}
\item \url{https://www.physionet.org/content/ptbdb/1.0.0}
\item the PTB database of ECGs has been copied to {\tt /data/shared/04224/PTB}
\item The database contain 549 conventional 12-lead resting ECGs 
with the corresponding 3 Frank lead ECGs
\item There are 217 patients: see {\tt CASES.txt}
\item For example, 11 patients had left ventricular hypertrophy
086, 201, 210, 216, 217, 218, 219,
221, 222, 226 and 228
\item And there 54 healthy controls: see {\tt CONTROLS.txt}
\item Patient 086 data is in the directory {\tt patient086}
\item The ECG data is in {\tt s0316lre.dat}
\item The data header is {\tt s0316lre.hea}
\item The first line says that there 15 leads (12+3)
\item Sampled at 1000 Hz (1 per ms) for 115200 values per lead
\item The next 15 lines are the leads
\item The fifth number is the first value for that lead
\item The sixth number is the sum of values for that lead
\item However, the sixth number often doesn't seem to match
\end{itemize}
\end{frame}

\begin{frame}[fragile]
\frametitle{HW 2: Importing ECGs and signed integers}
\begin{itemize}
\item We can read in the data with an \textcolor{blue}{\tt informat}
  like those described in Chapters 4 and 5\\
see {\tt /data/shared/04224/patient086.sas}
\item The reason that the sixth number and SAS' sum don't agree
is due to the way that the arithmetic was calculated by PTB
\item They used 16-bit signed integers with two's complement\\
\item A type that SAS doesn't have: see the {\tt length} statement docs\\
16-bit is old-fashioned: 32 or 64-bit are more common now
\item Two's complement can \textcolor{blue}{ONLY} represent integers\\
 from $-2^{\,b-1}$ to $2^{\,b-1}-1$ where $b$ is the number of bits
\item And these values \textcolor{red}{wrap-around}\\ 
$2^{\,b-1}-1+1$ is $-2^{\,b-1}$ and NOT $2^{\,b-1}$\\
similarly, $-2^{\,b-1}-1$ is $2^{\,b-1}-1$ \\
\item So, if you calculate the \textcolor{red}{wrap-around}
sums accordingly, then does it match the sixth number?
\end{itemize}
\end{frame}

\end{document}
