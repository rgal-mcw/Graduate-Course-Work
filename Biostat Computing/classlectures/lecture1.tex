\documentclass[11pt,pdftex,dvipsnames,usenames,helvetica]{beamer}
\setbeameroption{show notes}
\usepackage[round]{natbib}
\usepackage{textcomp}
\usepackage{amsmath}
\usepackage{amssymb}
\usepackage{graphicx}
\DeclareGraphicsExtensions{.ps,.eps,.pdf,.jpg,.png}
\usefonttheme[onlymath]{serif}
\usepackage[cmintegrals,cmbraces]{newtxmath}
\usetheme{default}
\usecolortheme{dove}

\usepackage{cancel}
\usepackage{tikz}
\usepackage[english]{babel}

\usepackage{verbatim}
\usepackage{color}
\usepackage{pgf} %portable graphics format
\usepackage[autobold]{statex2}
\mode<presentation>
{
  %\usetheme{Warsaw}
  % or ...
  \setbeamercovered{transparent}
  % or whatever (possibly just delete it)

  \setbeamertemplate{navigation symbols}{}
  \usefonttheme[onlysmall]{structurebold}
  %\usefonttheme{structurebold}
}
\addtobeamertemplate{navigation symbols}{}{%
    \usebeamerfont{footline}%
  \setbeamertemplate{navigation symbols}{}
    \usebeamercolor[fg]{footline}%
    \hspace{1em}%
    \insertframenumber/\inserttotalframenumber
}
\newcommand*{\red}[1]{\textcolor{red}{#1}}
\newcommand*{\blue}[1]{\textcolor{blue}{#1}}

\begin{document}
\boldmath
% 0. Intro

\begin{frame}
\frametitle{Graduate School Class Reminders}

\begin{itemize}
% \item Keep your facial covering on, including over your nose
\item Maintain six feet of distancing
\item Please sit in the same chair each class time
\item Observe entry/exit doors as marked
\item Use hand sanitizer when you enter/exit the classroom
\item Use a disinfectant wipe/spray to wipe down your learning space
  before and after class
\item Media Services: 414 955-4357 option 2
%\item START RECORDING
\end{itemize}

\end{frame}

\begin{frame}
\frametitle{Documentation on the web}

\begin{itemize}
\item CRAN: \url{http://cran.r-project.org}
\item R manuals: \url{https://cran.r-project.org/manuals.html}
\item SAS: \url{http://support.sas.com/documentation}
\item Step-by-Step Programming with Base SAS 9.4 (SbS): \\
\url{https://documentation.sas.com/api/docsets/basess/9.4/content/basess.pdf}
\item SAS 9.4 Programmer’s Guide: Essentials (PGE): \\
\url{https://documentation.sas.com/api/docsets/lepg/9.4/content/lepg.pdf}
\item Wiki: \url{https://wiki.biostat.mcw.edu}
\end{itemize}

\end{frame}

\begin{frame}
\frametitle{HW 0: Homework problem for today}

\begin{itemize}
\item RTFM and increase the font size of your screen
\item I need to be able to look over your shoulder
\item And I can't do that if the font size is small
\item I'm optically challenged
\item You don't have to submit this: just show me
\item If not already addressed: who are the teams?
\item Take turns submitting the HW problems to me
\item Make them organized so I can easily review
\end{itemize}
\end{frame}

\begin{frame}
\frametitle{HW 1: Homework problem for today}

\begin{itemize}
\item Create a shell script to find the largest file per directory
\item Starting in a directory find the largest file\\
among this directory's file and its sub-directories\\
for final results: {\tt /data/shared/04224}
%for final results: {\tt /data/shared/gamma/export/home/klein}
\item Use recursion: in each directory, return largest file's name 
\item We can sort these files to find the largest file
\item Hints: use pipes, {\tt ls}, {\tt head/tail} and ({\tt echo} for debugging) 
\item {\tt ls -lS \textasciigrave biggest\textasciigrave\ | head -n 1}
%\item {\tt ls -lSr \textasciigrave biggest\textasciigrave\ | tail -n 1}
\item add a directory for scripts to {\tt PATH}:\\ 
{\tt gouda\$ PATH="\${PATH}:\${PWD}"}
\item white space in shell scripts is very important/inflexible\\
check docs and pay attention to Emacs' syntax highlighting 
\item spaces in filenames/directories is tricky\\ 
{\tt eval FILE="$\diagdown$"\$PWD/\$i$\diagdown$""}
\item Ignore files with special characters in their names like \$ 
\item You have to be fault-tolerant/defensive-programming \\
 {\tt gouda\$ cd "\$NEW" 2$>$ /dev/null \&\& biggest}
\item Emacs Signals menu useful for debugging: BREAK and KILL
\end{itemize}

\end{frame}

\begin{frame}
\frametitle{Outline for today}

\begin{itemize}
\item Introduction to bash: Bourne-Again shell
\item Bash shell initialization
\item Bash shell scripting
\item Hands-on with Emacs/ESS and R
\end{itemize}
\end{frame}


\begin{frame}
\frametitle{Bash shell initialization}
\begin{itemize}
\item {\tt /etc/profile.d/custom.sh }
\item {\tt $\sim$/.bash\_profile}
\item {\tt $\sim$/.bashrc}
\end{itemize}
\end{frame}

\begin{frame}
\frametitle{Bash shell scripting}
\begin{itemize}
\item Bash shell scripts allow you to create your own commands
\item Powerful feature that promotes code re-use
\item We have already seen the example {\tt emacs-26.3}
\item A very simple example: {\tt /usr/local/bin/path}
\item The environment variable {\tt PATH} is a list of directories\\
containing commands to type at the command line:
 {\tt gouda\$}
\item Let's look at another example: {\tt /usr/local/bin/hidden }
\item The first line (if present) is a comment {\tt \#} followed by
  {\tt !}  called ``shebang'' for {\tt \#!} and the name of the shell
  like {\tt \#!/bin/bash}
\item Shell choices (worst to best): {\tt /bin/sh} for the Bourne shell,\\
{\tt /bin/ksh} for the Korn shell, {\tt /bin/zsh} for the Z shell\\
and {\tt /bin/bash} for the Bourne-Again shell\\
\red{Never use the C shell {\tt /bin/csh}}
\item Return code 0 is a success and 1 is a failure: {\tt \$?}
\item {gouda\$ hidden $~$/.Rprofile; echo \$?}
\item {gouda\$ hidden emacs-26.3; echo \$?}
\item {gouda\$ hidden file-does-not-exist; echo \$?}
\item And a more complex example: {\tt /usr/local/bin/pf}
\end{itemize}
\end{frame}

\begin{frame}
\frametitle{Bash shell scripting}
\begin{itemize}
\item Simple syntax documented in the man page: {\tt M-x man bash}
\item Let's look at the following entries of \textcolor{red}{built-ins}
\item {\tt [[ expression ]]}
%\item {\tt if} command block
\item CONDITIONAL EXPRESSIONS
%\item {\tt for} which is also very useful from the command line
%\item Simple loops: {\tt for i in \{1..3\} }
%\item {\tt for} example {\tt /data/shared/TorqueCluster/synch/yum-installed}
\item Other commands often found in scripts: \textcolor{blue}{NOT built-ins} 
\item the {\tt find} command: {\tt M-x man find}
\item the {\tt sort} command: {\tt M-x man sort}
\item white space is very important/inflexible
\item pay attention to Emacs' syntax highlighting 
\end{itemize}
\end{frame}

\begin{frame}
\frametitle{{\tt if} command syntax: \textcolor{red}{RED} and 
\textcolor{blue}{BLUE} lines optional}

{\tt if IF-CONDITIONAL-EXPRESSION\\
then IF-THEN-LINE \\
\textcolor{red}{ADDITIONAL-LINES-AS-NEEDED} \\
\textcolor{blue}{elif ELIF-CONDITIONAL-EXPRESSION-1\\
then ELIF-THEN-LINE-1} \\
\textcolor{red}{ADDITIONAL-LINES-AS-NEEDED}\\
$\vdots$ \\
\textcolor{blue}{elif ELIF-CONDITIONAL-EXPRESSION-M\\
then ELIF-THEN-LINE-M} \\
\textcolor{red}{ADDITIONAL-LINES-AS-NEEDED}\\
\textcolor{blue}{else ELSE-LINE}\\
\textcolor{red}{ADDITIONAL-LINES-AS-NEEDED}\\
fi}

\end{frame}

\begin{frame}
\frametitle{{\tt for} command syntax: \textcolor{red}{RED} and 
\textcolor{blue}{BLUE} clauses are optional}

{\tt for VARIABLE in \textcolor{blue}{VALUE1 VALUE2 \dots}\\
do LINE\\
\textcolor{red}{ADDITIONAL-LINES-AS-NEEDED}\\
done}

\begin{itemize}
\item Suppose that {\tt VARIABLE} is {\tt i}
\item Typically, {\tt LINE} and/or 
{\tt \textcolor{red}{ADDITIONAL-LINES-AS-NEEDED}} will have references 
to the variable, {\tt \$i}, which loops over {\tt VALUES}
\item {\tt for} example, see 
{\tt /data/shared/TorqueCluster/synch/yum-installed}
\end{itemize}

{\tt for i in x y emacs-* \{1..3\}\\
do echo \$i\\
done}

\end{frame}

\begin{frame}
\frametitle{Emacs/ESS and R}
\begin{itemize}
\item Launch the latest version of emacs with your script: {\tt emacs-26.3}
\item From the command line, you can get brief documentation
\item In the {\tt *shell*} buffer: {\tt gouda\$ emacs --help}
\item Similarly: {\tt gouda\$ R --help} or  {\tt gouda\$ R -h}
\item Consult the man page for more: {\tt M-x man emacs}
\item Or {\tt M-x man R}
\item For the complete emacs manual: {\tt F1 r}
\item Or for the other manuals available: {\tt M-x info} or {\tt F1 i}
\end{itemize}
\end{frame}

\begin{frame}
\frametitle{Emacs/ESS and R}
\begin{itemize}
\item Copy my R profile to your home directory: press {\tt F8} \\ 
{\tt gouda\$ cp $~$rsparapa/.Rprofile $~$}
\item Open it with {\tt C-x C-f .Rprofile}
\item Notice the file's title at the top, the menus and the toolbar
\item Check out the mode-line at the bottom
\item Create a directory for your personal library: press {\tt F8} \\
{\tt gouda\$ mkdir -p $~$/R/4.0.4/lib64/R/library}
\item Let's create an R program: {\tt C-x C-f lecture2.R}
\item Launch the default version, 4.0.4, of R: {\tt M-x R}
\item If you wanted the previous version 3.6.2: {\tt M-x R-3.6.2}
% \item In the {\tt *R:$~$*} at the {\tt $>$}, type {\tt ??} for 
% \textcolor{red}{help on help}
% \item To search R help and installed packages help,\\
%  type {\tt $>$ help.start()}\\
%  VERY SLOW IF LOTS OF PACKAGES ARE INSTALLED
%\item type {\tt \$ R} to start from the command line (without emacs)
\end{itemize}
\end{frame}

\begin{frame}
\frametitle{Emacs/ESS and R}
\begin{itemize}
\item There used to be {\tt <-} keys on old keyboards\\
\item Even before my time and I go back to the mid-80s
\item But, what key should we use for assignment today?
\item Now, we should just use {\tt =}
\item Since {\tt ==} is used to test for equivalence (as we'll see)
\item R snobs still use {\tt <-} but it is an anachronism
\item Type in a simple R program (don't type the comments {\tt \#})
\item Or copy it from the shared directory: {\tt lecture2.R}
\end{itemize}
{\tt a = installed.packages() \# first line} \\
{\tt \# = creates an object to the left} \\
%{\tt str(a)} \\
{\tt table(a[ , "LibPath"])} \\
{\tt b $<$- a} \\
{\tt \# $<$- creates an object to the left like =} \\
{\tt b -$>$ c} \\
{\tt \# -$>$ creates an object to the right} \\
{\tt ls()} \\
%{\tt a['BART', ] \# last line}
\end{frame}

\begin{frame}
\frametitle{Emacs/ESS and R}
\begin{itemize}
\item In order of likely use and importance
\item To submit a paragraph (code block bounded by blank lines):\\
 {\tt C-c C-p}
\item To submit the whole buffer: {\tt C-c C-b}
\item To submit a single line: {\tt C-Enter}
\item To submit a region (a highlighted area):\\
 {\tt C-c C-r}
\end{itemize}
\end{frame}

\end{document}
