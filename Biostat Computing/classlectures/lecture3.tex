\documentclass[11pt,pdftex,dvipsnames,usenames,helvetica]{beamer}
\setbeameroption{show notes}
\usepackage[round]{natbib}
\usepackage{textcomp}
\usepackage{amsmath}
\usepackage{amssymb}
\usepackage{graphicx}
\DeclareGraphicsExtensions{.ps,.eps,.pdf,.jpg,.png}
\usefonttheme[onlymath]{serif}
\usepackage[cmintegrals,cmbraces]{newtxmath}
\usetheme{default}
\usecolortheme{dove}

\usepackage{cancel}
\usepackage{tikz}
\usepackage[english]{babel}

\usepackage{verbatim}
\usepackage{color}
\usepackage{pgf} %portable graphics format
\usepackage[autobold]{statex2}
\mode<presentation>
{
  %\usetheme{Warsaw}
  % or ...
  \setbeamercovered{transparent}
  % or whatever (possibly just delete it)

  \setbeamertemplate{navigation symbols}{}
  \usefonttheme[onlysmall]{structurebold}
  %\usefonttheme{structurebold}
}
\addtobeamertemplate{navigation symbols}{}{%
    \usebeamerfont{footline}%
  \setbeamertemplate{navigation symbols}{}
    \usebeamercolor[fg]{footline}%
    \hspace{1em}%
    \insertframenumber/\inserttotalframenumber
}
\newcommand*{\red}[1]{\textcolor{red}{#1}}
\newcommand*{\blue}[1]{\textcolor{blue}{#1}}

\begin{document}
\boldmath
% 0. Intro

\begin{frame}
\frametitle{Graduate School Class Reminders}

\begin{itemize}
% \item Keep your facial covering on, including over your nose
\item Maintain six feet of distancing
\item Please sit in the same chair each class time
\item Observe entry/exit doors as marked
\item Use hand sanitizer when you enter/exit the classroom
\item Use a disinfectant wipe/spray to wipe down your learning space
  before and after class
\item Media Services: 414 955-4357 option 2
%\item START RECORDING
\end{itemize}

\end{frame}

\begin{frame}
\frametitle{Documentation on the web}

\begin{itemize}
\item CRAN: \url{http://cran.r-project.org}
\item R manuals: \url{https://cran.r-project.org/manuals.html}
\item SAS: \url{http://support.sas.com/documentation}
\item Step-by-Step Programming with Base SAS 9.4 (SbS): \\
\url{https://documentation.sas.com/api/docsets/basess/9.4/content/basess.pdf}
\item SAS 9.4 Programmer s Guide: Essentials (PGE): \\
\url{https://documentation.sas.com/api/docsets/lepg/9.4/content/lepg.pdf}
\item Wiki: \url{https://wiki.biostat.mcw.edu} 
\textcolor{red}{(MCW/VPN)}
\end{itemize}

\end{frame}

\begin{frame}
\frametitle{HW 3: Probability distributions}

\begin{enumerate}
\item[0.] Normal {\tt norm} Example:
  $Z~\N{\mbox{mean}=10}{\mbox{sd}=25}, \P{Z<c}=0.025$: what is $c$?
\item[1.] Beta {\tt beta} (the incomplete Beta function):
  $Z~\Beta{\mbox{shape1}=10}{\mbox{shape2}=25}, \P{Z<c}=0.025,
  \P{Z<d}=0.975$: what is $c$ and $d$?
\item[2.] Gamma {\tt gamma} (the incomplete Gamma function):
  $Z~\Gam{\mbox{shape}=10}{\mbox{rate}=25}, \P{Z<c}=0.025,
  \P{Z<d}=0.975$: what is $c$ and $d$?
\item[3.] F {\tt f}:
  $Z~\F{\mbox{df1}=10}{\mbox{df2}=25}, \P{Z<c}=0.025, \P{Z<d}=0.975$:
  what is $c$ and $d$?
\item[4.] Poisson {\tt pois}:
  $Z~\Poi{\mbox{lambda}=25}, \P{Z \le 15}=p$: what is $p$?
\item[5.] Normal {\tt norm}: $Z~\N{\mbox{mean}=10}{\mbox{sd}=25}$\\
  Generate 10000 random $Z$ and graphically compare the empirical CDF,
  {\tt ecdf()}, with the true CDF.  And calculate $\sd^2$ by
Monte Carlo integration to compare with the true value.
\end{enumerate}

\end{frame}

\begin{frame}
\frametitle{Matrices}
Mathematically, a matrix is represented as follows.
\begin{align*}
A=\wrap{\begin{array}{cccc}
a_{11} & a_{12} & \cdots & a_{1n} \\
a_{21} & a_{22} & \cdots & a_{2n} \\
\vdots& \vdots & \ddots & \vdots \\
a_{m1} & a_{m2} & \cdots & a_{mn} \\
\end{array}}
\end{align*}
{R} is a column-major language, i.e., matrices are laid out
in consecutive memory locations by traversing the columns:
$\wrap{a_{11}, a_{21}, \., a_{12}, a_{22}, \.}$.  {R} is written in
{C} and {Fortran} where {Fortran} is a column-major language as well.
However, {C} and {C++} are row-major languages, i.e., matrices are
laid out in consecutive memory locations by traversing the rows:
$\wrap{a_{11}, a_{12}, \., a_{21}, a_{22}, \.}$.  So, if you have
written an {R} function in {C}/{C++}, then pass it $A$ transpose
rather than $A$ itself, i.e., {\tt t(A)} as {\tt t()} is the
R transpose function.
\end{frame}

\begin{frame}
\frametitle{Section 5.7: Matrices (two-dimensional arrays)}

\begin{itemize}
\item {\tt A=matrix(DATA, nrow=n, ncol=p)} creates {\tt A$_{n\times p}$}\\
{\tt DATA} goes down the columns filling in column-major order
\item {\tt A=matrix(DATA, nrow=n, ncol=p, byrow=TRUE)}\\
{\tt DATA} goes across the rows filling in row-major order\\
but {\tt A$_{n\times p}$} is still a column-major matrix!
\item {\tt diag(n)} creates an identity matrix {\tt n} by {\tt n}
%\item {\tt state.x77} is a matrix 50 by 8
%\item {\tt Matrix} package is one of the Recommended packages that
%  comes with your R installation
\item {\tt A \%*\% B} is matrix multiplication
\item {\tt crossprod(A, B)} is the cross-product\\
like {\tt t(A) \%*\% B} but more efficient
\item {\tt crossprod(A)} is like {\tt t(A) \%*\% A}
\item {\tt solve(A, b)} solves the equation {\tt b = A \%*\% x } for {\tt x}
\item {\tt solve(A)} creates {\tt A$^{-1}$}, but matrix inversion can be unstable
\item For example, when calculating {\tt b \%*\% A$^{-1}$ \%*\% b}, it is 
more numerically stable as {\tt b \%*\% solve(A, b) }
\end{itemize}

\end{frame}

\begin{frame}
\frametitle{Section 5.4: The recycling rule for mixed vector and array
  (or matrix) arithmetic} 

The precise rule affecting element by element mixed calculations with
vectors and arrays is somewhat quirky and hard to find in the
references.
\begin{itemize}
\item The expression is scanned from left to right.
\item Any short vector operands are extended by recycling their values
  until they match the size of any other operands.
\item As long as short vectors and arrays only are encountered, the
  arrays must all have the same dim attribute or an error results.
\item Any vector operand longer than a matrix or array operand
  generates an error.
\item If array structures are present and no error or coercion to
  vector has been precipitated, the result is an array structure with
  the common dim attribute of its array operands.
\end{itemize}

\end{frame}

\begin{frame}
\frametitle{Section 5.4: The recycling rule for mixed vector and array
  (or matrix) arithmetic} 
Examples
\begin{align*}
A_{n\times p} - b_n & = A-B \where B=\wrap{b_n^1, \dots, b_n^p}_{n \times p} \\
& \mbox{Recycling is what you likely intended} \\
A_{n\times p} - b_p & = A-C %\where C=\wrap{b^1, \dots, b^n}_{n \times p} \\
\mbox{\ difficult to demonstrate in general}\\
& \mbox{so assume $n=4,\ p=3$ and\ } b=\wrap{b_1, b_2, b_3}\\
& \mbox{which results in\ } C=\wrap{\begin{array}{ccc} 
b_1 & b_2 & b_3 \\
b_2 & b_3 & b_1 \\
b_3 & b_1 & b_2 \\
b_1 & b_2 & b_3
\end{array}} \\
& \mbox{\red{Recycling is NOT what you likely intended}} \\
A^t_{p \times n} - b_p & = A^t- D \where D=\wrap{b_p^1, \dots, b_p^n}_{p \times n} \\
& \mbox{Recycling is what you likely intended} \\
\end{align*}
\end{frame}

\begin{frame}
\frametitle{Section 6: Lists}

%{\tt $>$ Lst = list(husband="Fred", wife="Mary", no.children=3,\\
%\qquad child.ages=c(4, 7, 9))}
{\tt case = list(type="breast", ICDO="C50.3", stage="IIA",\\
\qquad sex="F", side="L", birth=as.Date("1970/04/05"),\\
\qquad diag=as.Date("2019/12/05"), \\
\qquad surg=as.Date("2020/03/02"), \\
\qquad T=list(P="T1a", size=0.45),\\
\qquad N=list(P="N1", C="N0", surg="SLNB",\\
\qquad positive=3, removed=7, max=0.25), M="M0")}
\begin{itemize}
\item A {\it list} is an object consisting of an ordered collection of
  objects of singular or diverse types known as {\it components}
\item Components are {\it numbered} and can always be referred to as such:
{\tt case[[1]], case[[2]], case[[3]], \dots}
\item Components of lists may also be {\it named}, and in this case
  the component may be referred to either by giving the component name
  or as a character string
\item So, the first component can be referenced by
{\tt case[[1]]} or {\tt case\$type} or {\tt case\$"type"}
\item Or, rarely, as a list itself called a {\it sublist}: {\tt case[1]}
\end{itemize}

\end{frame}

\begin{frame}
\frametitle{Comma separated values (CSV) files}

\begin{itemize}
\item CSV files are a standard that arose from Fortran data files
evolving late-60s to mid-70s: roughly what we have today
\item Sadly, subject matter experts (SME) LOVE Excel and CSVs\\ 
(we tell them to stop, but they don't listen)
\item So CSV files are the de facto standard for data exchange
\item But CSVs from spreadsheets are \textcolor{red}{not type safe}\\
 e.g., a numeric column can have text entries, etc.
\item In the column for weights, they can type {\tt deceased} in a cell
\item So the {\tt weight} column will be character instead of numeric
\item Although, theoretically, they are trivial to import,\\
the practice of SMEs will make your life difficult
\item There does not appear to be a complete solution to this problem
with R (other than manual editing)
\item SAS has a few tricks that might help as we will see later
\end{itemize}

\end{frame}


\begin{frame}
\frametitle{Section 8: Probability distributions }

\begin{itemize}
\item Consider a discrete random variable (RV): $Y \in \{a, \ldots, b\} $
\item Probability density function (PDF):
  $f(c) = \P{Y=c}\I{c \in \{a, \ldots, b\}} \in [0, 1]$
\item Cumulative density function (CDF): $F(c) = \P{Y\le c}= 
\sum_{y \in \{a, \ldots, b\} \cap y\le c} f(y) \in [0, 1]$
\item $\P{c<Y \le d}=\P{Y \le d}-\P{Y \le c}=F(d)-F(c)$
\item Consider a continuous RV: $Y \in (a, b)$
\item PDF: $f(c)\I{c \in (a, b)} \ge 0 $
\item CDF: $F(c) = %\P{Y\le c}= 
\P{Y < c}=
\I{a < c} \int_{a}^{\min(b,c)} f(y) \mathrm{d} y \in [0, 1]$
\item $\P{c<Y < d}=\P{Y < d}-\P{Y < c}=F(d)-F(c)$
\item For either a discrete or a continuous RV $Y$\\
Inverse CDF: $p=F(c) \Rightarrow c = F^{-1}(p) = F^{-1}(F(c))$
\end{itemize}

\end{frame} 

\begin{frame}
\frametitle{Pseudo-random numbers and Monte Carlo integration }

\begin{itemize}
\item CDFs can be used for many integral/summation calculations
\item The Expectation of a function of a random variable, $g(Y)$\\ 
Discrete RV: $\E{g(Y)} = \sum_{y \in \{a, \ldots, b\}} g(y) f(y)$\\ 
Continuous RV: $\E{g(Y)} = \int_{a}^{b} g(y) f(y) \mathrm{d} y$
\item Mean $=\E{Y}=\mu$, i.e., $g(Y)=Y$, the identity function
\item Variance $=\E{Y^2}-\E{Y}^2=\sd^2$
\item For some $g(Y)$ functions, there is no closed form solution
\item Monte Carlo Integration: 
generate a {\it large} sample of pseudo-random numbers $Y_1, \dots, Y_N$
and approximate \\
$\E{g(Y)} \approx N^{-1}\sum_{i=1}^N g(Y_i)$ 
\end{itemize}

\end{frame} 

\begin{frame}
\frametitle{Section 8: Probability distributions }

\begin{itemize}
\item The PDF R function is {\tt dxxx} for density
\item The CDF R function is {\tt pxxx} for probability
\item The Inverse CDF R function is {\tt qxxx} for quantiles
\item The Random Number R function is {\tt rxxx} for random\\
(paired with \textcolor{red}{\tt set.seed} for reproducibility)
\item {\tt xxx} is one of the following suffixes:\\
{\tt beta, binom, cauchy, chisq, exp, f, gamma, geom, hyper, lnorm, logis, nbinom, norm, pois, signrank, t, unif, weibull, wilcox}
\item These functions mean that we don't need all of those tables at
  the end of our textbooks
\item We can calculate probabilities,
  summations/integrals, etc.\ for these distributions with R
 \end{itemize}

\end{frame} 

\begin{frame}
\frametitle{Probability and discrete RVs: sports examples}
\begin{itemize}
\item Coach K and the Duke Blue Devils college basketball team
have been to 13 NCAA Final Fours
\item During these appearances, they have won 5 NCAA Championships
\item What is the probability that they would have won less than 5?
\item Aaron Rodgers and the Green Bay Packers football team are\\ 
1 for 4 in the NFC Championship Playoff Game?
\item What is the probability that they would have won more than 1 game?
\end{itemize}
\end{frame}

\begin{frame}
\frametitle{Section 8.3: Hypothesis testing}

\begin{itemize}
\item This course is a co-requisite with\\
 Statistical Models and Methods I
\item The intent is not to teach you statistics, but to be prepared
\item For example, what hypothesis tests are available?
\item Here's a list of those in the {\tt stats} package
\item {\tt $>$ ls("package:stats", pattern="test\$")}
 \end{itemize}

\end{frame} 

\end{document}
