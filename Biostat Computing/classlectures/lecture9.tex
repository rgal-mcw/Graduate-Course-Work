\documentclass[11pt,pdftex,dvipsnames,usenames,helvetica]{beamer}
\setbeameroption{show notes}
\usepackage[round]{natbib}
\usepackage{textcomp}
\usepackage{amsmath}
\usepackage{amssymb}
\usepackage{graphicx}
\DeclareGraphicsExtensions{.ps,.eps,.pdf,.jpg,.png}
\usefonttheme[onlymath]{serif}
\usepackage[cmintegrals,cmbraces]{newtxmath}
\usetheme{default}
\usecolortheme{dove}

\usepackage{cancel}
\usepackage{tikz}
\usepackage[english]{babel}

\usepackage{verbatim}
\usepackage{color}
\usepackage{pgf} %portable graphics format
\usepackage[autobold]{statex2}
\mode<presentation>
{
  %\usetheme{Warsaw}
  % or ...
  \setbeamercovered{transparent}
  % or whatever (possibly just delete it)

  \setbeamertemplate{navigation symbols}{}
  \usefonttheme[onlysmall]{structurebold}
  %\usefonttheme{structurebold}
}
\addtobeamertemplate{navigation symbols}{}{%
    \usebeamerfont{footline}%
  \setbeamertemplate{navigation symbols}{}
    \usebeamercolor[fg]{footline}%
    \hspace{1em}%
    \insertframenumber/\inserttotalframenumber
}
\newcommand*{\red}[1]{\textcolor{red}{#1}}
\newcommand*{\blue}[1]{\textcolor{blue}{#1}}

\begin{document}
\boldmath
% 0. Intro

\begin{frame}
\frametitle{Graduate School Class Reminders}

\begin{itemize}
% \item Keep your facial covering on, including over your nose
\item Maintain six feet of distancing
\item Please sit in the same chair each class time
\item Observe entry/exit doors as marked
\item Use hand sanitizer when you enter/exit the classroom
\item Use a disinfectant wipe/spray to wipe down your learning space
  before and after class
\item Media Services: 414 955-4357 option 2
%\item START RECORDING
\end{itemize}

\end{frame}

\begin{frame}
\frametitle{Documentation on the web}

\begin{itemize}
\item CRAN: \url{http://cran.r-project.org}
\item R manuals: \url{https://cran.r-project.org/manuals.html}
\item SAS: \url{http://support.sas.com/documentation}
\item Step-by-Step Programming with Base SAS 9.4 (SbS): \\
\url{https://documentation.sas.com/api/docsets/basess/9.4/content/basess.pdf}
\item SAS 9.4 Programmer s Guide: Essentials (PGE): \\
\url{https://documentation.sas.com/api/docsets/lepg/9.4/content/lepg.pdf}
\item Wiki: \url{https://wiki.biostat.mcw.edu} 
\textcolor{red}{(MCW/VPN)}
\end{itemize}

\end{frame}

\begin{frame}[fragile]
\frametitle{R debugging tips: common pitfalls}
\begin{itemize}
\item \textcolor{red}{subsetting a matrix with a variable length index:\\
    when resolving to one column (or row) returns a vector}\\
  and your code is expecting a matrix to be returned\\
  counter-measures: when returning a subset of columns (rows), encase
  within {\tt cbind} ({\tt rbind}) which creates a matrix if needed
\item \textcolor{blue}{unintentional backward {\tt for} loops}\\
\begin{verbatim}
L=length(cuts) ## a list
xinfo=rbind(cuts[[1]]) ## creating a matrix
## NEXT LINE IS WRONG
## for(i in 2:L) xinfo=rbind(xinfo, cuts[[i]])
## IF LENGTH IS ONE: THE ABOVE LOOP RUNS BACKWARDS
## NEXT LINE IS RIGHT: DEFENSIVE PROGRAMMING
if(L>1) for(i in 2:L) xinfo=rbind(xinfo, cuts[[i]])
\end{verbatim}
\item logical subset of a vector with 0/1 rather than {\tt FALSE/TRUE}\\
  use {\tt which} religiously to avoid this {\it error-less} hard to find bug
\item typos: {\tt ==} in comparisons vs.\ {\tt =} in assignments
\item scalar {\tt \&\&/||} vs.\ vector {\tt \&/|} 
({\tt all/any} might be of help)
\item similarly, scalar {\tt min/max} vs.\ vector {\tt pmin/pmax}
\end{itemize}

\end{frame}

\begin{frame}[fragile]
\frametitle{Building and installing R packages: CRAN}
\begin{itemize}
\item The Comprehensive R Archive Network (CRAN) has\\
 16335 R add-on packages as of this writing\\
there will be more by the time you read this
\item On gouda, we have 14449 installed mainly from CRAN
\item To install an R package from CRAN\\
The two most reliable, and likely complete, nearby mirrors are 
\url{http://lib.stat.cmu.edu/R/CRAN} at Carnegie-Mellon and
\url{http://cran.wustl.edu} at Washington University in St.L.\\
N.B. \textcolor{red}{http} NOT  \textcolor{blue}{https} 
\end{itemize}
{\tt $>$~options(repos=c(CRAN="http://lib.stat.cmu.edu/R/CRAN")) \\
%$>$ install.packages("sem", dependencies=TRUE) \\
$>$ install.packages("remotes", dependencies=TRUE) }\\
To install all CRAN packages (takes hours: we run this over-night)\\
{\tt $>$ install.packages(available.packages()[ , 1])}\\
Some of them will fail for missing system dependencies like
device drivers, required software, etc.
\end{frame}

\begin{frame}[fragile]
\frametitle{Building and installing R packages: Bioconductor}
\begin{itemize}
\item The Bioconductor Project produces R packages for bioinformatics: 
\url{http://bioconductor.org}
\item Bioconductor versions are tied to specific R versions\\
{\tt $>$~tools:::.BioC\_version\_associated\_with\_R\_version()}\\
returns {\tt '3.10'} with R 3.6.2
\item  To install the package named {\tt limma} (and R or Bioconductor package dependencies, if any)\\
{\tt $>$ source("http://bioconductor.org/biocLite.R")\\
$>$ biocLite("limma")}
\item 
To install all Bioconductor packages (takes a while):\\
{\tt $>$ biocLite(all\_group())}
\end{itemize}
\end{frame}

\begin{frame}[fragile]
\frametitle{Building and installing R packages: command line}
\begin{itemize}
\item You can build and install R packages from the command line
\item This works with your own R packages or those of others
\item If it is your own in the sub-directory {\tt PACKAGE},
then build it:\\ {\tt $>$ R CMD build PACKAGE}
\item For others, download the archive of source files\\ either a tar file 
ending in {\tt .tar.gz} or {\tt .tgz}\\ or a ZIP file {\tt .zip}
\item Unpack it: {\tt $>$ tar xzf TARFILE} or 
{\tt $>$ unzip ZIPFILE} which should create the {\tt PACKAGE} sub-directory
\item You don't have to change directories
\item Build the package: {\tt $>$ R CMD build PACKAGE}
\item Sometimes the vignettes crash the build or take a long time\\
{\tt $>$ R CMD build --no-build-vignettes PACKAGE}
\item So now you have created {\tt PACKAGE.VERSION.tar.gz}
\item Install it: {\tt $>$ R CMD INSTALL PACKAGE.VERSION.tar.gz}
\item And you can remove it later: {\tt $>$ R CMD REMOVE PACKAGE}
\end{itemize}

\end{frame}

\begin{frame}[fragile]
\frametitle{Building and installing R packages: {\tt remotes} package}
\begin{itemize}
\item You can build and install R packages from anywhere on the
internet with the remotes package
\item For example, former CRAN packages that have 
been Archived: \url{https://cran.r-project.org/src/contrib/Archive}
\item These can be installed with the {\tt install\_url} function
\item Or R packages on \url{https://github.com}
\item These can be installed with the {\tt install\_github} function
\item However, R 3.6.2 or higher appears to be necessary
\item And installing from the command line seems to be much faster
\item Typically, there are also Orphaned packages, however, currently
there are none (as of this writing)\\
\url{https://cran.r-project.org/src/contrib/Orphaned}
\end{itemize}

\end{frame}

\begin{frame}[fragile]
\frametitle{HW Hands-on: Building and installing R packages}
\begin{itemize}
% \item Let's install the {\tt DPpackage}
% from\\ \url{https://cran.r-project.org/src/contrib/Archive/DPpackage}\\
% with the {\tt install\_url} function
\item Let's install the {\tt BART3} R package 
from {\tt https://github.com/rsparapa/bnptools} 
\item  with the {\tt install\_github} function
\item and with the command line using {\tt git}
\item create your own sub-directory: {\tt $>$ mkdir $~$/git; cd $~$/git}
\end{itemize}
{\tt $>$ git clone~https://github.com/rsparapa/bnptools.git}\\
{\tt $>$ cd bnptools}

\end{frame}

\end{document}
