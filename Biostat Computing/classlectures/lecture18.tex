\documentclass[11pt,pdftex,dvipsnames,usenames,helvetica]{beamer}
\setbeameroption{show notes}
\usepackage[round]{natbib}
\usepackage{textcomp}
\usepackage{amsmath}
\usepackage{amssymb}
\usepackage{graphicx}
\DeclareGraphicsExtensions{.ps,.eps,.pdf,.jpg,.png}
\usefonttheme[onlymath]{serif}
\usepackage[cmintegrals,cmbraces]{newtxmath}
\usetheme{default}
\usecolortheme{dove}

\usepackage{cancel}
\usepackage{tikz}
\usepackage[english]{babel}

\usepackage{verbatim}
\usepackage{color}
\usepackage{pgf} %portable graphics format
\usepackage[autobold]{statex2}
\mode<presentation>
{
  %\usetheme{Warsaw}
  % or ...
  \setbeamercovered{transparent}
  % or whatever (possibly just delete it)

  \setbeamertemplate{navigation symbols}{}
  \usefonttheme[onlysmall]{structurebold}
  %\usefonttheme{structurebold}
}
\addtobeamertemplate{navigation symbols}{}{%
    \usebeamerfont{footline}%
  \setbeamertemplate{navigation symbols}{}
    \usebeamercolor[fg]{footline}%
    \hspace{1em}%
    \insertframenumber/\inserttotalframenumber
}
\newcommand*{\red}[1]{\textcolor{red}{#1}}
\newcommand*{\blue}[1]{\textcolor{blue}{#1}}

\begin{document}
\boldmath
% 0. Intro

\begin{frame}
\frametitle{Graduate School Class Reminders}

\begin{itemize}
% \item Keep your facial covering on, including over your nose
\item Maintain six feet of distancing
\item Please sit in the same chair each class time
\item Observe entry/exit doors as marked
\item Use hand sanitizer when you enter/exit the classroom
\item Use a disinfectant wipe/spray to wipe down your learning space
  before and after class
\item Media Services: 414 955-4357 option 2
%\item START RECORDING
\end{itemize}

\end{frame}

\begin{frame}
\frametitle{Documentation on the web}

\begin{itemize}
\item CRAN: \url{http://cran.r-project.org}
\item R manuals: \url{https://cran.r-project.org/manuals.html}
\item SAS: \url{http://support.sas.com/documentation}
\item 
\textcolor{blue}{SAS 9.3: \url{https://support.sas.com/en/documentation/documentation-for-SAS-93-and-earlier.html}}
\item Step-by-Step Programming with Base SAS 9.4 (SbS): \\
\url{https://documentation.sas.com/api/docsets/basess/9.4/content/basess.pdf}
\item SAS 9.4 Programmer s Guide: Essentials (PGE): \\
\url{https://documentation.sas.com/api/docsets/lepg/9.4/content/lepg.pdf}
\item Wiki: \url{https://wiki.biostat.mcw.edu} 
\textcolor{red}{(MCW/VPN)}
\end{itemize}

\end{frame}

\begin{frame}[fragile]
\frametitle{Custom output with the DATASTEP}
\begin{itemize}
\item As we have seen the {\tt put} statement output appears in the {\tt .log}
by default
\item However, if you want to create custom output, then the {\tt .log} is
a poor choice
\item With the {\tt file} statement, you can specify other locations such 
as {\tt .lst} or others
\item \textcolor{red}{\tt file print;} specifies {\tt .lst}
\item \textcolor{blue}{\tt file "FILENAME";} specifies a new file
\item The default is to replace the file: the {\tt mod} option appends
\item \textcolor{red}{\tt file print mod;} 
\item \textcolor{blue}{\tt file "FILENAME" mod;} 
\item By default, titles defined are used: {\tt notitles} turns them off
\item By default, footnotes defined are NOT used: {\tt footnotes} turns them on
\end{itemize}

\end{frame}

\begin{frame}[fragile]
\frametitle{Page size and line size}
\begin{itemize}
\item The {\tt options} statement has two options for output
\item \textcolor{blue}{\tt options ps=LINES;} controls how many lines
  are on a page
\item \textcolor{red}{\tt options ls=CHARACTERS;} controls how many
  characters are in a line
\item In our {\tt $~$/autoexec.sas}, we have two settings for convenience
\item For portrait, {\tt options \&portrait;} (which is the default)
where \textcolor{blue}{\tt \%let portrait = ps=54 ls=76;}
\item For landscape, {\tt options \&landscape;} \\
where \textcolor{red}{\tt \%let landscape= ps=49 ls=122;}
\end{itemize}

\end{frame}

\begin{frame}[fragile]
\frametitle{The Output Delivery System (ODS)}
\begin{itemize}
\item With version 7 (1998), SAS introduced ODS: a more flexible
  output management system useful for many common tasks
\item As we have seen, you can choose not to produce
output with the {\tt \%\_printto} macro (that relies 
on {\tt proc printto;})
\item With ODS, you can do the same by \textcolor{blue}{\tt ods
    listing close;}
\item And you can turn it back it on by \textcolor{blue}{\tt ods
    listing;}
\end{itemize}

\end{frame}

\begin{frame}[fragile]
\frametitle{The Output Delivery System (ODS)}
\begin{itemize}
\item One of the main reasons for ODS' creation is to capture\\ 
SAS output in a data set for further processing
\item \textcolor{red}{\tt ods output \textcolor{blue}{ODSTABLENAME}=NEW;} \\
where {\tt NEW} is the data set created 
\item But how do we know what \textcolor{blue}{\tt ODSTABLENAME} 
is supposed to be?
\item Prior to the PROC statement, place \textcolor{blue}{\tt ods trace on;}
\item After the {\tt \textcolor{blue}{run;} \textcolor{red}{quit;}}
  you can turn it off \textcolor{red}{\tt ods trace off;}
\item Don't forget the {\tt \textcolor{blue}{run;} \textcolor{red}{quit;}}\\
{\tt \textcolor{red}{quit;}} is generally optional, but it is needed for ODS
\item You can selectively generate output with\\
\textcolor{red}{\tt ods select \textcolor{blue}{ODSTABLENAME};}\\
which will place ONLY the selected choices in your {\tt .lst} 
\item And you can selectively exclude output with\\
\textcolor{red}{\tt ods exclude \textcolor{blue}{ODSTABLENAME};}
\item see example {\tt PTB/acf.sas}
\end{itemize}

\end{frame}

\begin{frame}[fragile]
\frametitle{The Output Delivery System (ODS)}
\begin{itemize}
\item ODS also controls Statistical Graphics PROCs: SGPROCs
\item So to capture PDF output\\
\textcolor{red}{\tt ods graphics on / imagename="\&fnroot" outputfmt=pdf;}
\item \textcolor{blue}{\tt "\&fnroot"} 
(which is created by {\tt \%\_fn;} in {\tt $~$/autoexec.sas})\\
your SAS program is named {\tt \&fnroot..sas}
so the\\ \textcolor{blue}{FIRST} PDF file is called {\tt \&fnroot..pdf}
\item Subsequent PDF files are named {\tt \&fnroot.n.pdf}\\
where {\tt n} is a number: 1, 2, ...
\item If we call the macro {\tt \%\_pdfjam;}\\ then these files
are combined into one PDF: {\tt \&fnroot..pdf}
\item And pressing {\tt F12} will open the \textcolor{blue}{FIRST} PDF
for viewing by default: you can over-ride this in the mini-buffer
\item see example {\tt PTB/acf.sas}
\end{itemize}

\end{frame}

\end{document}

