\documentclass[11pt,pdftex,dvipsnames,usenames,helvetica]{beamer}
\setbeameroption{show notes}
\usepackage[round]{natbib}
\usepackage{textcomp}
\usepackage{amsmath}
\usepackage{amssymb}
\usepackage{graphicx}
\DeclareGraphicsExtensions{.ps,.eps,.pdf,.jpg,.png}
\usefonttheme[onlymath]{serif}
\usepackage[cmintegrals,cmbraces]{newtxmath}
\usetheme{default}
\usecolortheme{dove}

\usepackage{cancel}
\usepackage{tikz}
\usepackage[english]{babel}

\usepackage{verbatim}
\usepackage{color}
\usepackage{pgf} %portable graphics format
\usepackage[autobold]{statex2}
\mode<presentation>
{
  %\usetheme{Warsaw}
  % or ...
  \setbeamercovered{transparent}
  % or whatever (possibly just delete it)

  \setbeamertemplate{navigation symbols}{}
  \usefonttheme[onlysmall]{structurebold}
  %\usefonttheme{structurebold}
}
\addtobeamertemplate{navigation symbols}{}{%
    \usebeamerfont{footline}%
  \setbeamertemplate{navigation symbols}{}
    \usebeamercolor[fg]{footline}%
    \hspace{1em}%
    \insertframenumber/\inserttotalframenumber
}
\newcommand*{\red}[1]{\textcolor{red}{#1}}
\newcommand*{\blue}[1]{\textcolor{blue}{#1}}

\begin{document}
\boldmath
% 0. Intro

\begin{frame}
\frametitle{Graduate School Class Reminders}

\begin{itemize}
% \item Keep your facial covering on, including over your nose
\item Maintain six feet of distancing
\item Please sit in the same chair each class time
\item Observe entry/exit doors as marked
\item Use hand sanitizer when you enter/exit the classroom
\item Use a disinfectant wipe/spray to wipe down your learning space
  before and after class
\item Media Services: 414 955-4357 option 2
%\item START RECORDING
\end{itemize}

\end{frame}

\begin{frame}
\frametitle{Documentation on the web}

\begin{itemize}
\item CRAN: \url{http://cran.r-project.org}
\item R manuals: \url{https://cran.r-project.org/manuals.html}
\item SAS: \url{http://support.sas.com/documentation}
\item 
\textcolor{blue}{SAS 9.3: \url{https://support.sas.com/en/documentation/documentation-for-SAS-93-and-earlier.html}}
\item Step-by-Step Programming with Base SAS 9.4 (SbS): \\
\url{https://documentation.sas.com/api/docsets/basess/9.4/content/basess.pdf}
\item SAS 9.4 Programmer s Guide: Essentials (PGE): \\
\url{https://documentation.sas.com/api/docsets/lepg/9.4/content/lepg.pdf}
\item Wiki: \url{https://wiki.biostat.mcw.edu} 
\textcolor{red}{(MCW/VPN)}
\end{itemize}

\end{frame}

\begin{frame}[fragile]
\frametitle{DATASTEP: arrays}
\begin{itemize}
\item Arrays are a temporary construct of the DATASTEP\\
Their component variables may, or may not, be saved in the 
{\tt NEW} data set, but the array's definition is NOT
\item Numeric: {\tt array \_VAR(n) VAR1 ...\ VARn;}\\
{\tt \_VAR(n)} can be an asterisk: {\tt \_VAR(*)}
\item Character: {\tt array \_CHAR(m) \$ L CHAR1 ...\ CHARm;}
%\item The examples below can be either numeric or character
\item {\tt array \_VAR(n:m) VARn ...\ VARm;} 
\item Temporary: variables are not named and not saved\\
{\tt array \_VAR(n) \_TEMPORARY\_;}\\
{\tt \_VAR(n)} can NOT be an asterisk\
\item Optionally, with initial values assigned\\
 {\tt array \_VAR(n) \_TEMPORARY\_ 
(VALUE1 ...\ VALUEn);}
\item Two dimension arrays: \textcolor{red}{row major}\\
{\tt array \_VAR(n, m) VAR1 ...\ VARnm;} \\
{\tt \_VAR(n, m)} can NOT be an asterisk\\
\item Multiple dimension arrays\\
{\tt array \_VAR(n, ..., m) ...;} 
\end{itemize}
\end{frame}

\begin{frame}[fragile]
\frametitle{DATASTEP: arrays}
\begin{itemize}
\item You reference array values like so
\item For {\tt array \_VAR(n) VAR1 ...\ VARn;} \\
\textcolor{red}{\tt \_VAR(i)}
\item However, that is ambiguous since now SAS will NOT call
the function {\tt \_VAR()}\\
a reason that I like to use the underscore
in array names since there are no such functions \\
(YET: we will re-visit that next lecture)
\item But, for most flexibility\\
 (such as in the {\tt PROC FCMP}
which we will see next time)\\
use \textcolor{blue}{\tt \_VAR[i]}
\end{itemize}
\end{frame}

\begin{frame}[fragile]
\frametitle{DATASTEP: iterative DO loops}
\begin{itemize}
\item The {\tt do} loop has a few variants
\item {\tt do VAR=VALUE1 to VALUE2 by VALUE3; ...; end;}\\
if {\tt VALUE3} not given, it defaults to 1
\item {\tt do VAR=VALUE1 to VALUE2 until(CONDITION); ...; end;}\\
  increments until the {\tt CONDITION} is true or {\tt VALUE2}\\
  whichever comes first
\item {\tt do VAR=VALUE1 to VALUE2 while(CONDITION); ...; end;}\\
  increments until the {\tt CONDITION} is false or {\tt VALUE2}\\
  whichever comes first
\end{itemize}
\end{frame}

\begin{frame}[fragile]
\frametitle{SAS formats}
\begin{itemize}
\item In Chapters 4 and 5, we saw informats for inputting data
\item For output, formats end in a period: {\tt FORMAT.}\\
with an optional WIDTH {\tt FORMATw.} potentially along with\\ 
a number of DECIMAL places {\tt FORMATw.d}
\item Common numeric formats:\\ 
{\tt data NEW; set OLD; format X best12.\ Y 6.4 Z z4.;}
\item {\tt Zw.} and {\tt Zw.d} are filled with leading zeros
\item Common character formats: they start with dollar sign\\ 
{\tt data NEW; set OLD; format X \$40.\ Y \$char200.;}
\item SAS supplies many formats and informats
\item You can make your own with {\tt proc format}
\end{itemize}
\end{frame}

\begin{frame}[fragile]
\frametitle{SAS dates}
\begin{itemize}
\item There are other data types besides numeric and character\\ 
the most common is a SAS date\\
with literals that look like this\\
{\tt "DD\textcolor{blue}{MON}YYYY"\textcolor{red}{d}} GOOD\\
 {\tt "DD\textcolor{blue}{MON}YY"\textcolor{red}{d}} BAD\\
due to Y2K, and other types of 2-digit year errors,\\
 the former is far preferable to the latter (see below)
\item For example, {\tt "12OCT2020"\textcolor{red}{d}}
which is the numeric value 22200: the number of days
from the SAS origin 01/01/1960 
(which is the typical choice for IBM mainframes\\
as opposed to 01/01/1970 of Unix/R)
\item They are NOT ISO 8601 compliant: dates prior to 1582 generate an error
\item And a Y2K-like bug/feature\\
{\tt "12OCT0000"\textcolor{red}{d}}
  is equivalent to {\tt "12OCT2000"\textcolor{red}{d}}\\
  {\tt "0000"} is considered to be the two-digit year {\tt "00"} or
  {\tt "2000"}
  this behavior is documented so it is a {\it feature}\\
  (SAS is pretty good about fixing bugs due to their rental fees)\\
\end{itemize}
\end{frame}

\begin{frame}[fragile]
\frametitle{DATASTEP: SAS dates}
\begin{itemize}
\item There are many useful SAS features for SAS dates
\item Such as the SAS format {\tt date9.}
\item And the automatic macro variable {\tt sysdate9}
which is formatted as {\tt date9.}
\item There are several useful SAS data functions
\item For example, {\tt intnx} and {\tt intck} 
\item {\tt intnx} creates a date offset into the future or the past
\item {\tt intck} counts the number of intervals between two dates\\
useful for calculating age
\item see {\tt lecture13.sas}
\item There are also times {\tt "HH:MM:SS"\textcolor{red}{t}}
\item And datetimes 
{\tt "DD{MON}YYYY:HH:MM:SS"\textcolor{blue}{dt}}
\item But these are uncommon except for EMR/clinical trials
\end{itemize}
\end{frame}

\begin{frame}[fragile]
\frametitle{DATASTEP: {\tt SELECT} statement type 1\\
\textcolor{red}{RED} and
\textcolor{blue}{BLUE} are optional}
{\tt select(EXPRESSION);\\
when(VALUE1) BLOCK1\\
\textcolor{red}{when(VALUE2) BLOCK2\\
\qquad $\vdots$\\
when(VALUEn) BLOCKn}\\
\textcolor{blue}{otherwise BLOCKO}\\
end;}
\begin{itemize}
\item if {\tt VALUE1, ..., VALUEn} are not exhaustive\\
AND there is no {\tt otherwise} clause,\\ then errors
are generated for those observations that remain
\end{itemize}
\end{frame}

\begin{frame}[fragile]
\frametitle{DATASTEP: {\tt SELECT} statement type 2\\
\textcolor{red}{RED} and
\textcolor{blue}{BLUE} are optional}
{\tt select;\\
when(CONDITION1) BLOCK1\\
\textcolor{red}{when(CONDITION2) BLOCK2\\
\qquad $\vdots$\\
when(CONDITIONn) BLOCKn}\\
\textcolor{blue}{otherwise BLOCKO}\\
end;}
\begin{itemize}
\item if {\tt CONDITION1, ..., CONDITIONn} are not exhaustive\\
AND there is no {\tt otherwise} clause,\\ then errors
are generated for those observations that remain
\end{itemize}
\end{frame}

\begin{frame}[fragile]
\frametitle{HW part 1: create a date format for the ISO 8601\\
 Proleptic Gregorian Calendar}
\begin{itemize}
\item The details are in lecture 2, slides 3-5
\item See my example of the Proleptic Julian Calendar
{\tt julian.sas}
\end{itemize}
\end{frame}

\begin{frame}[fragile]
\frametitle{HW part 2: NTDB hot-decking}
\begin{itemize}
\item Write a SAS DATASTEP program to perform hot-decking\\
 for the NTDB example:
see the details in lecture 4, slide 4
%\item Hints: use arrays and \textcolor{blue}{\tt set ...\ point}\\
\item Hints: for random number generation from the Uniform
  distribution, use the {\tt rand("unif")} and {\tt call
    streaminit(SEED);} to set the seed.
 \item Use arrays with the {\tt array} statement
 \item Use the {\tt set} statement with the {\tt point} option to
   iterate through the data set BEWARE: with {\tt point}, don't forget
   the {\tt \textcolor{red}{stop;} run;} at the end of the DATASTEP
\item With many functions you can use an {\tt of} clause with a variable
list, e.g., \textcolor{blue}{\tt nmiss(of VAR1-VARn)}\\
instead of \textcolor{red}{\tt nmiss(VAR1, ..., VARn)}
\item Or with the STARTS-WITH colon operator
 \textcolor{blue}{\tt nmiss(of VAR:)}
% \item Use a macro variable for the list of variables to
% hot-deck like {\tt \%let var=white black other hispanic male sbp pulse rr;}
% \item To count the number of items in a macro variable, 
% use the {\tt \%\_count()} macro like {\tt \%let k=\%\_count(\&var);}
\item Use the {\tt \%\_nobs()} macro to determine how many
observations are in a data set like {\tt \%\_nobs(data=ntdb.elder)}
\end{itemize}
\end{frame}

\end{document}
