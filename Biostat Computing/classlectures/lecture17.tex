\documentclass[11pt,pdftex,dvipsnames,usenames,helvetica]{beamer}
\setbeameroption{show notes}
\usepackage[round]{natbib}
\usepackage{textcomp}
\usepackage{amsmath}
\usepackage{amssymb}
\usepackage{graphicx}
\DeclareGraphicsExtensions{.ps,.eps,.pdf,.jpg,.png}
\usefonttheme[onlymath]{serif}
\usepackage[cmintegrals,cmbraces]{newtxmath}
\usetheme{default}
\usecolortheme{dove}

\usepackage{cancel}
\usepackage{tikz}
\usepackage[english]{babel}

\usepackage{verbatim}
\usepackage{color}
\usepackage{pgf} %portable graphics format
\usepackage[autobold]{statex2}
\mode<presentation>
{
  %\usetheme{Warsaw}
  % or ...
  \setbeamercovered{transparent}
  % or whatever (possibly just delete it)

  \setbeamertemplate{navigation symbols}{}
  \usefonttheme[onlysmall]{structurebold}
  %\usefonttheme{structurebold}
}
\addtobeamertemplate{navigation symbols}{}{%
    \usebeamerfont{footline}%
  \setbeamertemplate{navigation symbols}{}
    \usebeamercolor[fg]{footline}%
    \hspace{1em}%
    \insertframenumber/\inserttotalframenumber
}
\newcommand*{\red}[1]{\textcolor{red}{#1}}
\newcommand*{\blue}[1]{\textcolor{blue}{#1}}

\begin{document}
\boldmath
% 0. Intro

\begin{frame}
\frametitle{Graduate School Class Reminders}

\begin{itemize}
% \item Keep your facial covering on, including over your nose
\item Maintain six feet of distancing
\item Please sit in the same chair each class time
\item Observe entry/exit doors as marked
\item Use hand sanitizer when you enter/exit the classroom
\item Use a disinfectant wipe/spray to wipe down your learning space
  before and after class
\item Media Services: 414 955-4357 option 2
%\item START RECORDING
\end{itemize}

\end{frame}

\begin{frame}
\frametitle{Documentation on the web}

\begin{itemize}
\item CRAN: \url{http://cran.r-project.org}
\item R manuals: \url{https://cran.r-project.org/manuals.html}
\item SAS: \url{http://support.sas.com/documentation}
\item 
\textcolor{blue}{SAS 9.3: \url{https://support.sas.com/en/documentation/documentation-for-SAS-93-and-earlier.html}}
\item Step-by-Step Programming with Base SAS 9.4 (SbS): \\
\url{https://documentation.sas.com/api/docsets/basess/9.4/content/basess.pdf}
\item SAS 9.4 Programmer s Guide: Essentials (PGE): \\
\url{https://documentation.sas.com/api/docsets/lepg/9.4/content/lepg.pdf}
\item Wiki: \url{https://wiki.biostat.mcw.edu} 
\textcolor{red}{(MCW/VPN)}
\end{itemize}

\end{frame}

\begin{frame}[fragile]
\frametitle{{\tt proc print} for producing output}
\begin{itemize}
\item {\tt proc print} is the go-to for many printing tasks
\item The {\tt var} statement lists the variables to output
\item The {\tt uniform} option makes each page's columns look like all
  the others
%\item Take advantage of SAS functionality such as {\tt where} clauses and {\tt FORMAT.}
\item Subset the data for printing with a {\tt where} statment
\item Specify SAS formats with a {\tt format} statement
\item Specify labels for variable names with the 
\textcolor{red}{\tt label} statement that is used in tandem with the
  \textcolor{blue}{\tt label} option\\ don't forget to use both!\\
{\tt proc print \textcolor{blue}{\tt label} data=NAME;
var X;\\
 \textcolor{red}{label} X="LABEL"; run;}
\item A {\tt by} clause will force each {\tt by} group on their own pages\\
can be fine-tuned with the {\tt pageby} statement
\end{itemize}

\end{frame}

\begin{frame}[fragile]
\frametitle{{\tt title} and {\tt footnote}}
\begin{itemize}
\item These two statements are similar
\item They can appear anywhere before a {\tt run;} statement
\item You can have up to 10 titles and 10 footnotes
\item \textcolor{red}{\tt title "TITLE";} and 
\textcolor{blue}{\tt footnote "FOOTNOTE";}\\
produce the top title or footnote respectively
\item \textcolor{red}{\tt titlen "TITLE";} and 
\textcolor{blue}{\tt footnoten "FOOTNOTE";} where {\tt n} is a number
from 1 to 10 to
 produce the n$^{th}$ title or footnote %respectively
\item \textcolor{red}{\tt title;} and 
\textcolor{blue}{\tt footnote;} clear all titles or footnotes
\item \textcolor{red}{\tt titlen;} and 
\textcolor{blue}{\tt footnoten;} where {\tt n} is a number
produce, clear the n$^{th}$ through 10$^{th}$ title or footnote respectively
\end{itemize}

\end{frame}

\begin{frame}[fragile]
\frametitle{{\tt proc plot} for producing plots}
\begin{itemize}
\item These are primitive TEXT plots
\item For quick and dirty looks at the data
\item Much easier to use than more advanced PROCs but
extremely limited functionality
\item {\tt proc plot data=NAME;\\ \qquad plot Y*X="CHARACTER"; run;}
where {\tt CHARACTER} is the symbol to use:
a symbol with a mark at its center is preferable
like {\tt "+"} rather than {\tt "."}
\item You can specify multiple plots\\
{\tt proc plot data=NAME;\\ \qquad plot Y1*X1="C1" ...\ Yn*Xn="Cn"; run;}
\item Similarly, plot them on the same page/axis\\
{\tt proc plot data=NAME;\\ \qquad plot Y1*X="C1" ...\ Yn*X="Cn" / overlay; run;}
\end{itemize}

\end{frame}

\begin{frame}[fragile]
\frametitle{{\tt proc chart} for producing bar charts}
\begin{itemize}
\item These are primitive TEXT charts
\item For quick and dirty looks at the data
\item Much easier to use than more advanced PROCs but
extremely limited functionality
\item {\tt proc chart data=NAME;\\ \qquad vbar X; run;} for vertical bars
\item {\tt proc chart data=NAME;\\ \qquad hbar Y; run;} for horizontal bars
\item You can specify multiple charts\\
{\tt proc chart data=NAME;\\ \qquad vbar X; hbar Y; ...; run;} 
\item Use the {\tt discrete} option for categorical variables
\item {\tt proc chart data=NAME;\\ \qquad vbar X / discrete; run;} for vertical bars
\end{itemize}

\end{frame}

\begin{frame}[fragile]
  \frametitle{SAS/Graph, graphics stream files
and {\tt $~$/autoexec.sas}}
\begin{itemize}
\item Using the SAS/Graph product to produce graphics stream files
\item This is a still relevant legacy system of \textcolor{red}{G}PROCs
  which\\ predates modern ODS and SAS Statistical Graphics
  \textcolor{blue}{SG}PROCs
\item Typically, \textcolor{red}{G}PROCs are named {\tt proc
    \textcolor{red}{G}NAME} and\\ Statistical Graphics
  \textcolor{blue}{SG}PROCs are named {\tt proc \textcolor{blue}{SG}NAME}
\item We have this setting in our {\tt $~$/autoexec.sas} file
\item {\tt goptions device=pslcmyk gsfmode=replace gsfname=gsasfile
    colors=(black red blue green cyan magenta yellow);}
\item This specifies Adobe PostScript color graphics stream files
with Cyan/Magenta/Yellow/blacK (CMYK) color standard
\item And this setting as well\\
 {\tt filename gsasfile "\&fnroot..ps";}\\
so, if your SAS program is named {\tt NAME.sas} then the 
macro variable {\tt fnroot} resolves to {\tt NAME}\\
{\tt fnroot} is a global macro variable created by the {\tt \_fn}
macro
\end{itemize}

\end{frame}

\begin{frame}[fragile]
  \frametitle{SAS/Graph, graphics stream files
and {\tt $~$/autoexec.sas}}
\begin{itemize}
\item So, the smart defaults in {\tt $~$/autoexec.sas}
result in a GPROC replacing the file {\tt \&fnroot..ps}\\
(or creating it if it does not exist)
\item Notice that there are two does in {\tt \&fnroot..ps}\\
{\tt \&fnroot.ps} would resolve to {\tt NAMEps}\\
one dot is macro variable concatenation, so the second dot is necessary
to produce {\tt NAME.ps} instead
\item If you press {\tt F12}, Emacs/ESS will open a viewer of
the file\\ (by reading your {\tt .log} to find the stream file you created)
\item Since the GPROC is replacing the file by default, if you want to
  produce more than one you need to specify \textcolor{blue}{append}
  rather than
  \textcolor{red}{replace} between the first graph creation and the second\\
  {\tt goptions gsfmode=\textcolor{blue}{append};}\\
For the rest of the SAS program, \textcolor{blue}{append} will 
be in effect which is likely what you want
\item Of course, you can over-ride these settings at any time
as necessary, but smart defaults will save a lot of your time 
\end{itemize}

\end{frame}

\begin{frame}[fragile]
  \frametitle{{\tt proc univariate}'s {\tt histogram} statement}
\begin{itemize}
\item {\tt proc univariate} uses the GPROC facility with its 
{\tt histogram} statement for a continuous variable
\item {\tt proc univariate noprint data=NAME; histogram X; run;}
\item You can only produce one histogram from a \\
{\tt proc univariate noprint data=NAME; ...\ run;} block
\item In between the first and the second, you need to 
switch to \textcolor{blue}{append}, then you can append as many 
histrograms as needed
\end{itemize}
\begin{verbatim}
proc univariate noprint data=NAME; 
histogram X;
run;
goptions gsfmode=append;
proc univariate noprint data=NAME; 
histogram Y;
run;
proc univariate noprint data=NAME; 
histogram Z;
run;
\end{verbatim}
\end{frame}

\end{document}
