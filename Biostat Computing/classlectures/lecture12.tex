\documentclass[11pt,pdftex,dvipsnames,usenames,helvetica]{beamer}
\setbeameroption{show notes}
\usepackage[round]{natbib}
\usepackage{textcomp}
\usepackage{amsmath}
\usepackage{amssymb}
\usepackage{graphicx}
\DeclareGraphicsExtensions{.ps,.eps,.pdf,.jpg,.png}
\usefonttheme[onlymath]{serif}
\usepackage[cmintegrals,cmbraces]{newtxmath}
\usetheme{default}
\usecolortheme{dove}

\usepackage{cancel}
\usepackage{tikz}
\usepackage[english]{babel}

\usepackage{verbatim}
\usepackage{color}
\usepackage{pgf} %portable graphics format
\usepackage[autobold]{statex2}
\mode<presentation>
{
  %\usetheme{Warsaw}
  % or ...
  \setbeamercovered{transparent}
  % or whatever (possibly just delete it)

  \setbeamertemplate{navigation symbols}{}
  \usefonttheme[onlysmall]{structurebold}
  %\usefonttheme{structurebold}
}
\addtobeamertemplate{navigation symbols}{}{%
    \usebeamerfont{footline}%
  \setbeamertemplate{navigation symbols}{}
    \usebeamercolor[fg]{footline}%
    \hspace{1em}%
    \insertframenumber/\inserttotalframenumber
}
\newcommand*{\red}[1]{\textcolor{red}{#1}}
\newcommand*{\blue}[1]{\textcolor{blue}{#1}}

\begin{document}
\boldmath
% 0. Intro

\begin{frame}
\frametitle{Graduate School Class Reminders}

\begin{itemize}
% \item Keep your facial covering on, including over your nose
\item Maintain six feet of distancing
\item Please sit in the same chair each class time
\item Observe entry/exit doors as marked
\item Use hand sanitizer when you enter/exit the classroom
\item Use a disinfectant wipe/spray to wipe down your learning space
  before and after class
\item Media Services: 414 955-4357 option 2
%\item START RECORDING
\end{itemize}

\end{frame}

\begin{frame}
\frametitle{Documentation on the web}

\begin{itemize}
\item CRAN: \url{http://cran.r-project.org}
\item R manuals: \url{https://cran.r-project.org/manuals.html}
\item SAS: \url{http://support.sas.com/documentation}
\item 
\textcolor{blue}{SAS 9.3: \url{https://support.sas.com/en/documentation/documentation-for-SAS-93-and-earlier.html}}
\item Step-by-Step Programming with Base SAS 9.4 (SbS): \\
\url{https://documentation.sas.com/api/docsets/basess/9.4/content/basess.pdf}
\item SAS 9.4 Programmer s Guide: Essentials (PGE): \\
\url{https://documentation.sas.com/api/docsets/lepg/9.4/content/lepg.pdf}
\item Wiki: \url{https://wiki.biostat.mcw.edu} 
\textcolor{red}{(MCW/VPN)}
\end{itemize}

\end{frame}

\begin{frame}[fragile]
\frametitle{HW: stratified random sampling and the NTDB}
\begin{itemize}
\item Write a SAS DATASTEP program to perform stratified random sampling:
see the details in lecture 4, slide 7
\item Hints: use the {\tt rand("unif")} function and the {\tt ordinal}
function to create permutations
\item A variable list can be used in many functions with the\\ 
{\tt of} clause like \textcolor{red}{\tt of VAR1-VARn}\\
for example, 
{\tt ordinal(m, of VAR1-VARn)} instead of {\tt ordinal(m, VAR1, ..., VARn)}
\end{itemize}
\end{frame}


\begin{frame}[fragile]
\frametitle{Sorting data sets with {\tt proc sort}}
\begin{itemize}
\item A big part of learning the \textcolor{red}{\it SAS way} of doing things
is working with sorted data sets 
\item You can sort a data set in ascending order\\ 
{\tt proc sort data=OLD out=NEW; by VAR1 ...\ VARn: run;}\\
Or {\tt proc sort data=OLD; by VAR1 ...\ VARn: run;}\\
\textcolor{blue}{if you have write access to {\tt OLD}}
\item Or descending order: each corresponding {\tt VAR} needs the
{\tt descending} modifier since ascending is the default\\ 
{\tt proc sort data=OLD out=NEW; by descending VAR; run;}
\end{itemize}

\end{frame}

\begin{frame}[fragile]
\frametitle{What is a unique key?}
\begin{itemize}
\item What is a unique key?
\item Each NTDB patient is anonymized by the identifier {\tt inc\_key}:\\
  \textcolor{blue}{is {\tt inc\_key} a unique key}, i.e., ONE record
  for each distinct value?
\item If so, then the {\tt /UNIQUE} option will succeed\\
if NOT, it will generate an error
\item Each hospital is represented by the anonymized identifier {\tt
    traumactr}: \textcolor{red}{is {\tt traumactr} a unique key?}
\end{itemize}
\begin{verbatim}
proc sort data=ntdb.elder 
    out=traumactr(index=(inc_key/UNIQUE));
    by traumactr inc_key;
run;
\end{verbatim}
\end{frame}

\begin{frame}[fragile]
\frametitle{Creating a unique key with PROC SORT}
\begin{itemize}
\item There are several ways to create a unique key when one doesn't exist
\item For example, there is the PROC SORT option {\tt NODUPKEY}
\end{itemize}
\begin{verbatim}
proc sort NODUPKEY data=ntdb.elder out=nodupe;
    by inc_key;
run;
\end{verbatim}
\end{frame}

\begin{frame}[fragile]
\frametitle{DATASTEP automatic variables\\ and automatic macro variables}
\begin{itemize}
\item {\it Automatic} variables 
  are temporary and not stored in the {\tt NEW} data set:
typically, they start and end with an underscore
\item \textcolor{blue}{\tt \_N\_} is the number of the current observation\\
{\tt \_N\_=1} for the first, etc.
\item {\tt \_ERROR\_} is 1 if an error has occurred in the
current observation and 0 otherwise
\item {\tt set OLD end=LAST} produces the variable {\tt LAST} which is
  1 for the last observation and 0 otherwise
\item {\tt data NEW; set OLD end=LAST; if LAST; run;}
\item Also, there are {\it PDV} lists: {\tt \_ALL\_}, 
{\tt \_NUMERIC\_} and {\tt \_CHARACTER\_}
\item Like an automatic variable, the keyword {\tt \_last\_} is the
  last data set actually created\\
\textcolor{red}{\tt data NEW2; set \_last\_; run;}
\item The same as the {\tt syslast} automatic macro variable\\
\textcolor{red}{\tt data NEW2; set \&syslast; run;}
\item {\it Automatic} macro variables start with {\tt sys}
\end{itemize}

\end{frame}

\begin{frame}[fragile]
\frametitle{By-group processing}
\begin{itemize}
\item In the DATASTEP, the \textcolor{red}{\tt by} statement is very useful
  when the data set is sorted by one or more variables\\
  {\tt data NEW; set OLD; by VAR1 ...\ VARn;}
\item There are two {\it automatic} variables for each {\tt VAR}
\item \textcolor{blue}{\tt FIRST.VAR} is 1 at the first of observation of a {\it by-group}\\
  and 0 for all others
\item \textcolor{blue}{\tt LAST.VAR} is 1 at the last of observation of a {\it by-group}\\
  and 0 for all others
\item If there is only one record in the {\it by-group},
then {\tt FIRST.VAR=LAST.VAR=1}
\item If there is more than one record per {\it by-group}, you can
  create a unique key, if needed, with a subsetting IF: {\tt if
    FIRST.VAR=1;}
\item Or alternatively: {\tt if LAST.VAR=1;} 
\end{itemize}

\end{frame}

\begin{frame}[fragile]
\frametitle{Summaries of multiple observations}
\begin{itemize}
\item The {\tt retain} statement creates a variable 
that RETAINs its value across DATASTEP observations\\
unlike variables in a data set which acquire a new value
from each observation due to automatic looping
\item {\tt retain VAR1 VALUE1 ...\ VARn VALUEn;}\\
the starting values are {\tt VAR1=VALUE1; ...; VARn=VALUEn;}\\
\item for no starting value, place the variables at the end\\
{\tt retain X1 VALUE1 ...\ Xn VALUEn Y1 ...\ Ym;}\\
{\tt Y1 ...\ Ym} are missing
\item For example, suppose that you want a total of the 
variable {\tt Z}\\
{\tt data NEW; set OLD end=LAST; retain TOT 0;\\ keep TOT; 
\textcolor{red}{TOT+Z;} if LAST; run;}
\item \textcolor{blue}{NTDB: let's calculate average annual
case volume for each trauma center}
\item see {\tt NTDB/sas/volume.sas}
\end{itemize}

\end{frame}

\end{document}

