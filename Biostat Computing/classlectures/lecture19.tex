\documentclass[11pt,pdftex,dvipsnames,usenames,helvetica]{beamer}
\setbeameroption{show notes}
\usepackage[round]{natbib}
\usepackage{textcomp}
\usepackage{amsmath}
\usepackage{amssymb}
\usepackage{graphicx}
\DeclareGraphicsExtensions{.ps,.eps,.pdf,.jpg,.png}
\usefonttheme[onlymath]{serif}
\usepackage[cmintegrals,cmbraces]{newtxmath}
\usetheme{default}
\usecolortheme{dove}

\usepackage{cancel}
\usepackage{tikz}
\usepackage[english]{babel}

\usepackage{verbatim}
\usepackage{color}
\usepackage{pgf} %portable graphics format
\usepackage[autobold]{statex2}
\mode<presentation>
{
  %\usetheme{Warsaw}
  % or ...
  \setbeamercovered{transparent}
  % or whatever (possibly just delete it)

  \setbeamertemplate{navigation symbols}{}
  \usefonttheme[onlysmall]{structurebold}
  %\usefonttheme{structurebold}
}
\addtobeamertemplate{navigation symbols}{}{%
    \usebeamerfont{footline}%
  \setbeamertemplate{navigation symbols}{}
    \usebeamercolor[fg]{footline}%
    \hspace{1em}%
    \insertframenumber/\inserttotalframenumber
}
\newcommand*{\red}[1]{\textcolor{red}{#1}}
\newcommand*{\blue}[1]{\textcolor{blue}{#1}}

\begin{document}
\boldmath
% 0. Intro

\begin{frame}
\frametitle{Graduate School Class Reminders}

\begin{itemize}
% \item Keep your facial covering on, including over your nose
\item Maintain six feet of distancing
\item Please sit in the same chair each class time
\item Observe entry/exit doors as marked
\item Use hand sanitizer when you enter/exit the classroom
\item Use a disinfectant wipe/spray to wipe down your learning space
  before and after class
\item Media Services: 414 955-4357 option 2
%\item START RECORDING
\end{itemize}

\end{frame}

\begin{frame}
\frametitle{Documentation on the web}

\begin{itemize}
\item CRAN: \url{http://cran.r-project.org}
\item R manuals: \url{https://cran.r-project.org/manuals.html}
\item SAS: \url{http://support.sas.com/documentation}
\item 
\textcolor{blue}{SAS 9.3: \url{https://support.sas.com/en/documentation/documentation-for-SAS-93-and-earlier.html}}
\item Step-by-Step Programming with Base SAS 9.4 (SbS): \\
\url{https://documentation.sas.com/api/docsets/basess/9.4/content/basess.pdf}
\item SAS 9.4 Programmer s Guide: Essentials (PGE): \\
\url{https://documentation.sas.com/api/docsets/lepg/9.4/content/lepg.pdf}
\item Wiki: \url{https://wiki.biostat.mcw.edu} 
\textcolor{red}{(MCW/VPN)}
\end{itemize}

\end{frame}

\begin{frame}[fragile]
\frametitle{Cheese cluster SAS program recovery system}
\begin{itemize}
\item Since SAS programs are run, debugged, re-run, etc., it is very
  easy to make a programming error (whether syntactically or
  semantically) between runs or even accidentally to delete your
  program entirely (these dangers seem less pronounced for R
  programming although they do exist to some extent)
\item On the server, all of our data files are backed up nightly
\item However, if you create a new SAS program today and make an error,
  then it won't be backed up until later so you will not be able to
  recover it since a backup copy does not yet exist
\item Also, even if it is backed up, it may take several hours to
get the sysadmin to recover your SAS program which might be too late
if you have an impending deadline
\item Therefore, we have an automated system that 
\textcolor{blue}{saves the last 9 versions of
your SAS program, your {\tt .log}, {\tt .lst}, etc.}
\item There is a hidden sub-directory {\tt .USERNAME} where
these are saved \textcolor{red}{triggered by each run
of the SAS program}
\item You can see how these are created by shell scripts: {\tt
    /usr/local/bin} directory with 
\textcolor{blue}{\tt sas} and {\tt backup.ksh}
\end{itemize}
\end{frame}

\begin{frame}[fragile]
\frametitle{SAS \textcolor{blue}{S}tatistical \textcolor{blue}{G}raphics:
\textcolor{blue}{SG}PROCs }
\begin{itemize}
\item The legacy \textcolor{red}{G}PROCs remain for old code and some edge cases\\
we will not be discussing these further
\item The new \textcolor{blue}{SG}PROCs largely cover the same
  territory and have many advantages in ease-of-use and superior
  capabilities
\item There are two \textcolor{blue}{SG}PROCs in particular
\item \textcolor{blue}{SG}PLOT will create many of the graphical displays
that we are accustomed to: scatter plots, bar charts, histograms, box plots,
heatmaps, time series and more
\item \textcolor{blue}{SG}PANEL has many of the same capabilities with 
the added ability to create panels for multiple displays on the same page
\item Each of these have many opportunities for customization within the
\textcolor{blue}{SG}PROCs themselves
\item In addition, there is an \textcolor{blue}{SG}ANNOTATE facility
so that you can overlay text, points, lines, etc.
\item We will need to use the latest documentation for SAS 9.4 since
the \textcolor{blue}{SG}PROCs are still developing (sometime buggy)
\item SAS has recently made the 9.4 documentation easier to navigate
(9.4 documentation is now on par with 9.3 or earlier)
\end{itemize}

\end{frame}


\begin{frame}[fragile]
\frametitle{SAS \textcolor{blue}{S}tatistical \textcolor{blue}{G}raphics:
\textcolor{blue}{SG}PROCs }
\begin{itemize}
\item In {\tt $~$/autoexec.sas}, we have this line commented out\\
{\tt ods graphics on / reset imagename="\&fnroot" imagefmt=pdf;
\textcolor{red}{footnote;}}
\item You place this line before your \textcolor{blue}{SG}PROC
to create\\ Adobe Portable Document Format graphics stream files
\item if your SAS program is {\tt NAME.sas}, then the first page of
your plot is {\tt NAME.pdf}, the second page is {\tt NAME1.pdf}, etc.\\
since the {\tt \&fnroot} macro variable resolves to {\tt NAME}\\
(created by the {\tt \%\_fn;} macro called in {\tt $~$/autoexec.sas})\\
and turn off the footnote it creates which is ugly for graphics
\item For publication/presentations, it is helpful to have each page
  in a separate file.  But, for programming and debugging, it is OFTEN
  easier to have them all in one file.
\item So place the line {\tt \%\_pdfjam;} after the \textcolor{blue}{SG}PROC
which will utilize the {\tt pdfjam} utility (from TeX Live) to create
a single PDF: {\tt NAME.pdf} (see the man page: {\tt man pdfjam})
\item Press {\tt F12} to view it
\item Once you are happy, then comment it out: {\tt \%*\_pdfjam;}
\end{itemize}
\end{frame}

\begin{frame}[fragile]
\frametitle{SAS \textcolor{blue}{S}tatistical \textcolor{blue}{G}raphics:
PROC \textcolor{blue}{SG}PLOT }
\begin{itemize}
\item You can get multiple plots on one page:
\textcolor{red}{RED} are optional\\
{\tt %\textcolor{red}{ods trace on;}\\
proc sgplot data=NAME;\\
STATEMENT1 ...;\\
\textcolor{red}{STATEMENT2 ...;}\\
$\vdots$\\
\textcolor{red}{STATEMENTn ...;}\\
run;}\\
where {\tt STATEMENT1 ...\ STATEMENTn} are {\it compatible}
\item Similarly, there is a {\tt \textcolor{blue}{group=VAR}}
where \textcolor{blue}{BLUE} is optional\\
{\tt proc sgplot data=NAME;\\
STATEMENT1 ... / \textcolor{blue}{group=VAR1};\\
\textcolor{red}{STATEMENT2 ... / \textcolor{blue}{group=VAR2};}\\
$\vdots$\\
\textcolor{red}{STATEMENTn ... / \textcolor{blue}{group=VARn};}\\
run;}
\item You can mix and match {\tt \textcolor{blue}{STATEMENT}s}
with and without a {\tt \textcolor{blue}{group}} option
\end{itemize}
\end{frame}

\begin{frame}[fragile]
\frametitle{SAS \textcolor{blue}{S}tatistical \textcolor{blue}{G}raphics:
PROC \textcolor{blue}{SG}PLOT }
\begin{itemize}
\item Two {\tt STATEMENTs} are compatible if they are the same
\item The {\tt STATEMENTs} that produce graphical output that 
we will consider are {\tt HISTOGRAM},  {\tt DENSITY}, {\tt VBOX/HBOX},
{\tt HEATMAP}, {\tt SERIES} and {\tt SCATTER} (N.B. there
are others: see the docs)
\item There are compatible statments that are NOT the same\\
  such as {\tt DROPLINE}, {\tt INSET}, {\tt KEYLEGEND/LEGENDITEM},
  {\tt REFLINE}, {\tt STYLEATTRS}, {\tt SYMBOLCHAR},
{\tt XAXIS} and {\tt YAXIS}
\item As well as the usual SAS statements like
{\tt FOOTNOTEn/TITLEn}, {\tt FORMAT}, {\tt LABEL} and {\tt WHERE}
\item And ODS statements like those about output data sets\\
{\tt \textcolor{red}{ods trace on;}\\
\textcolor{blue}{ods output NAME=NEW;}\\
proc sgplot data=OLD;\\
STATEMENT ...;\\
run;\\
\textcolor{red}{ods trace off;}}\\
VERY useful to understand what is being displayed
\end{itemize}
\end{frame}

\begin{frame}[fragile]
\frametitle{SAS \textcolor{blue}{S}tatistical \textcolor{blue}{G}raphics:
PROC \textcolor{blue}{SG}PLOT {\tt HISTOGRAM}}
\begin{itemize}
\item \textcolor{red}{RED} and \textcolor{blue}{BLUE} are optional\\
{\tt proc sgplot data=NAME;\\
histogram X1 ...  / \textcolor{blue}{group=Y1};\\
\textcolor{red}{histogram X2 ... / \textcolor{blue}{group=Y2};}\\
$\vdots$\\
\textcolor{red}{histogram Xn ... / \textcolor{blue}{group=Yn};}\\
run;}
\item The default is \textcolor{blue}{\tt scale=percent}
which is usually what you want, but not always
\item The alternative is \textcolor{blue}{\tt scale=count}\\
see {\tt NTDB/sas/histogram.sas}
\end{itemize}
\end{frame}

\begin{frame}[fragile]
\frametitle{SAS \textcolor{blue}{S}tatistical \textcolor{blue}{G}raphics:
PROC \textcolor{blue}{SG}PLOT {\tt DENSITY}}
\begin{itemize}
\item BEWARE: The default is {\tt type=normal} for Normality\\
THIS IS EXTREMELY DANGEROUS! NEVER USE THE DEFAULT
  EVEN IF YOU THINK NORMALITY MAKES SENSE.  LET THE DATA BE YOUR GUIDE.\\
USE THE OPTION {\tt type=kernel} 
\item \textcolor{red}{RED} and \textcolor{blue}{BLUE} are optional\\
{\tt proc sgplot data=NAME;\\
density X1  / type=kernel ... \textcolor{blue}{GROUP=Y1};\\
\textcolor{red}{density X2 / type=kernel ... \textcolor{blue}{GROUP=Y2};}\\
$\vdots$\\
\textcolor{red}{density Xn / type=kernel ... \textcolor{blue}{GROUP=Yn};}\\
run;}
\item see {\tt NTDB/sas/density.sas}
\end{itemize}
\end{frame}

\begin{frame}[fragile]
\frametitle{SAS \textcolor{blue}{S}tatistical \textcolor{blue}{G}raphics:
PROC \textcolor{blue}{SG}PLOT {\tt VBOX}}
\begin{itemize}
\item Box-plots are a standard way to summarize continuous outcomes
often for two or more discrete categories
\item You can overlay box-plots, but this is usually not
visually appealing
\item The {\tt VBOX} statement creates vertical box-plots
\item There is also {\tt HBOX} for horizontal box-plots
\item \textcolor{red}{RED} are optional\\
{\tt proc sgplot data=NAME;\\
vbox VAR1 \textcolor{red}{/ group=VAR2};\\
run;}
\item Let's see the documentation to see how a box-plot is defined
\item And {\tt NTDB/sas/vbox.sas}
\end{itemize}
\end{frame}

\begin{frame}[fragile]
\frametitle{SAS \textcolor{blue}{S}tatistical \textcolor{blue}{G}raphics:
PROC \textcolor{blue}{SG}PLOT {\tt HEATMAP}}
\begin{itemize}
\item A heat-map is a 2D plot of a 3D relationship with variables,\\
  {\tt X} and {\tt Y}, where the color plotted represents the height
  of the joint density in an {\tt NXBINS $\times$ NYBINS} grid of cells
\item Generally, the two variables are continuous
\item {\tt HEATMAP} supports categorical variables BADLY with options\\  
{\tt discretex} for {\tt X} and/or {\tt discretey} for {\tt Y}\\
the support is a little buggy at the moment
\item Although you can overlay multiple heat-maps, typically that
  will NOT be a good idea\\
{\tt proc sgplot data=NAME;\\
heatmap X=VAR1 Y=VAR2 \textcolor{red}{/ nxbins=N nybins=M};\\
run;}
\item GO TO SGANNOTATE AND RETURN HERE
\item See {\tt NTDB/sas/heatmap.sas}
\end{itemize}
\end{frame}

\begin{frame}[fragile]
\frametitle{SAS \textcolor{blue}{S}tatistical \textcolor{blue}{G}raphics:
PROC \textcolor{blue}{SG}PLOT {\tt SERIES}}
\begin{itemize}
\item \textcolor{red}{RED} and \textcolor{blue}{BLUE} are optional\\
{\tt proc sgplot data=NAME;\\
series X=VARX Y=VARY  / ... \textcolor{blue}{GROUP=VAR1};\\
\textcolor{red}{series X=VARX Y=VARY / ... \textcolor{blue}{GROUP=VAR2};}\\
$\vdots$\\
\textcolor{red}{series X=VARX Y=VARY / ... \textcolor{blue}{GROUP=VARn};}\\
run;}
\item see {\tt EHR/sas/series.sas}
\end{itemize}
\end{frame}

\begin{frame}[fragile]
\frametitle{SAS \textcolor{blue}{S}tatistical \textcolor{blue}{G}raphics:
PROC \textcolor{blue}{SG}PLOT {\tt SCATTER}}
\begin{itemize}
\item \textcolor{red}{RED} and \textcolor{blue}{BLUE} are optional\\
{\tt proc sgplot data=NAME;\\
scatter X=VARX Y=VARY  / ... \textcolor{blue}{GROUP=VAR1};\\
\textcolor{red}{scatter X=VARX Y=VARY / ... \textcolor{blue}{GROUP=VAR2};}\\
$\vdots$\\
\textcolor{red}{scatter X=VARX Y=VARY / ... \textcolor{blue}{GROUP=VARn};}\\
run;}
\item see {\tt EHR/sas/scatter.sas}
\end{itemize}
\end{frame}

\begin{frame}[fragile]
\frametitle{SAS \textcolor{blue}{S}tatistical \textcolor{blue}{G}raphics:
\textcolor{blue}{SG}ANNOTATE {\tt \%sganno} macro START HERE }
\begin{itemize}
\item The ANNOTATE facility is to further customize graphics 
stream files with text, lines, etc.
\item There are two versions
\item The legacy \textcolor{red}{G}PROC ANNOTATE facility
\item And the new \textcolor{blue}{SG}ANNOTATE facility
\item In {\tt $~$/autoexec.sas}, we call two ANNOTATE macros:
{\tt \%annomac;} to load the legacy \textcolor{red}{G}PROC ANNOTATE
facility which we will not discuss further
\item And {\tt \%sganno;} to load the new \textcolor{blue}{SG}ANNOTATE
  facility which we will describe
\item Each of these macros start with {\tt \textcolor{blue}{\%SG}name}
\item The most common arguments are\\
 {\tt x1=VARX} for the $x$-axis\\
 {\tt y1=VARY} for the $y$-axis\\
and {\tt drawspace="DATAVALUE"} for drawing in the {\it data} space
\item Typically, the most important arguments for a macro are at 
the top of the help and the rest have reasonable defaults
\end{itemize}
\end{frame}

\begin{frame}[fragile]
\frametitle{SAS \textcolor{blue}{S}tatistical \textcolor{blue}{G}raphics:
\textcolor{blue}{SG}ANNOTATE {\tt \%sganno} macro }
\begin{itemize}
\item So in your {\tt .log} you will see the following near the top
\end{itemize}
\begin{verbatim}
 Following macros are available 
     %SGANNO_HELP 
     %SGARROW 
     %SGIMAGE 
     %SGLINE 
     %SGOVAL 
     %SGPOLYCONT 
     %SGPOLYGON 
     %SGPOLYLINE 
     %SGRECTANGLE 
     %SGTEXT 
     %SGTEXTCONT 
Enter %SGANNO_HELP(macroname) for details 
   or %SGANNO_HELP(ALL) for details on all SGANNO macros.
\end{verbatim}
See {\tt PTB/ecgleadi.sas}
\end{frame}


\begin{frame}[fragile]
\frametitle{SAS \textcolor{blue}{S}tatistical \textcolor{blue}{G}raphics:
PROC \textcolor{blue}{SG}PANEL}
\begin{itemize}
\item PROC \textcolor{blue}{SG}PANEL follows roughly the same syntax
as\\ PROC \textcolor{blue}{SG}PLOT
\item But, it allows you to create {\it panels} of your data
\item Use the {\tt PANELBY} statement\\
typically with the {\tt ONEPANEL} option to place
them on one page \\
and the {\tt COLUMNS=m} option for the number of panel columns
{\tt 
proc sgpanel data=NAME;\\
PANELBY VAR1 ...\ VARn / onepanel columns=m;\\ 
STATEMENT1 ...;\\
\textcolor{red}{STATEMENT2 ...;}\\
$\vdots$\\
\textcolor{red}{STATEMENTn ...;}\\
run;}\\
\end{itemize}
See {\tt PTB/ecgpanel.sas}
\end{frame}

\begin{frame}[fragile]
\frametitle{SAS \textcolor{blue}{S}tatistical \textcolor{blue}{G}raphics:
PROC \textcolor{blue}{SG}PLOT {\tt STYLEATTRS}}
\begin{itemize}
\item Controls the colors of the lines/markers\\
{\tt styleattrs\\ datacontrastcolors=(black darkgray lightgray);}
\item Controls the colors of the {\it fill} pattern\\
{\tt styleattrs\\ datacolors=(black darkgray lightgray);}
\item You can see the SAS colors and their associated names
by running the following\\ (see {\tt /data/shared/04224/colors.sas}) 
\end{itemize}
\begin{verbatim}
proc registry list startat="COLORNAMES";
run;
\end{verbatim}
\end{frame}

\end{document}
