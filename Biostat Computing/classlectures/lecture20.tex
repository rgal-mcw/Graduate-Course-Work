\documentclass[11pt,pdftex,dvipsnames,usenames,helvetica]{beamer}
\setbeameroption{show notes}
\usepackage[round]{natbib}
\usepackage{textcomp}
\usepackage{amsmath}
\usepackage{amssymb}
\usepackage{graphicx}
\DeclareGraphicsExtensions{.ps,.eps,.pdf,.jpg,.png}
\usefonttheme[onlymath]{serif}
\usepackage[cmintegrals,cmbraces]{newtxmath}
\usetheme{default}
\usecolortheme{dove}

\usepackage{cancel}
\usepackage{tikz}
\usepackage[english]{babel}
\usepackage{hyperref}
\usepackage{verbatim}
\usepackage{color}
\usepackage{pgf} %portable graphics format
\usepackage[autobold]{statex2}
\mode<presentation>
{
  %\usetheme{Warsaw}
  % or ...
  \setbeamercovered{transparent}
  % or whatever (possibly just delete it)

  \setbeamertemplate{navigation symbols}{}
  \usefonttheme[onlysmall]{structurebold}
  %\usefonttheme{structurebold}
}
\addtobeamertemplate{navigation symbols}{}{%
    \usebeamerfont{footline}%
  \setbeamertemplate{navigation symbols}{}
    \usebeamercolor[fg]{footline}%
    \hspace{1em}%
    \insertframenumber/\inserttotalframenumber
}
\newcommand*{\red}[1]{\textcolor{red}{#1}}
\newcommand*{\blue}[1]{\textcolor{blue}{#1}}

\begin{document}
\boldmath
% 0. Intro

\begin{frame}
\frametitle{Graduate School Class Reminders}

\begin{itemize}
% \item Keep your facial covering on, including over your nose
\item Maintain six feet of distancing
\item Please sit in the same chair each class time
\item Observe entry/exit doors as marked
\item Use hand sanitizer when you enter/exit the classroom
\item Use a disinfectant wipe/spray to wipe down your learning space
  before and after class
\item Media Services: 414 955-4357 option 2
%\item START RECORDING
\end{itemize}

\end{frame}

\begin{frame}
\frametitle{Documentation on the web}

\begin{itemize}
\item CRAN: \url{http://cran.r-project.org}
\item R manuals: \url{https://cran.r-project.org/manuals.html}
\item SAS: \url{http://support.sas.com/documentation}
\item 
\textcolor{blue}{SAS 9.3: \url{https://support.sas.com/en/documentation/documentation-for-SAS-93-and-earlier.html}}
\item Step-by-Step Programming with Base SAS 9.4 (SbS): \\
\url{https://documentation.sas.com/api/docsets/basess/9.4/content/basess.pdf}
\item SAS 9.4 Programmer s Guide: Essentials (PGE): \\
\url{https://documentation.sas.com/api/docsets/lepg/9.4/content/lepg.pdf}
\item Wiki: \url{https://wiki.biostat.mcw.edu} 
\textcolor{red}{(MCW/VPN)}
\end{itemize}

\end{frame}

\begin{frame}[fragile]
\frametitle{R and GCC upgrade}

\begin{itemize}
\item
\url{https://wiki.biostat.mcw.edu/Running_R_on_Clusters}
\end{itemize}
\begin{verbatim}
#!/bin/bash
## ~/emacs-26.3 shell script 
(module load gcc/9.2 emacs/26.3 R/3.6.2; emacs "$@")
\end{verbatim}

\end{frame}

\begin{frame}[fragile]
\frametitle{Advanced SAS Macros}
\begin{itemize}
\item We are going to look at some exemplars from the RASmacro
library for inspiration
\item It is located locally at {\tt /usr/local/sasmacro}
\item And it is on github (but it needs to be updated)
\item These macros will give us ideas for how to use
SAS macros, how to write them and what further can be
done with them
\item In particular, {\it hardening} of macros to
quotes, embedded ampersands, etc.
\end{itemize}

\end{frame}

\begin{frame}[fragile]
\frametitle{Advanced SAS Macros: Building Blocks}
\begin{center}
\begin{tabular}{ll}
Name               & Description \\ \hline
{\tt \_abend.sas}  & Abend the SAS program \\
{\tt \_count.sas}  & Number of items in a list \\
{\tt \_dsexist.sas}& Does a data set exist? \\
{\tt \_exist.sas}  & Does a file exist? \\
{\tt \_fn.sas}     & Create footnotes with the SAS program name \\
{\tt \_level.sas}  & Create a macro value from a data set var.\\
{\tt \_list.sas}   & Create a list: more general than a var. list \\
{\tt \_nobs.sas}   & Number of obs. in a data set \\
{\tt \_pdfjam.sas} & Combine ODS graphic files \\
{\tt \_require.sas}& Is a required argument present? \\
{\tt \_retain.sas} & RETAIN with BY-group processing\\
{\tt \_sort.sas}   & Sort a data set if necessary \\
{\tt \_substr.sas} & Smart sub-string function \\
{\tt \_unwind.sas} & OS-specific statements: UNIX/Linux vs. Windows \\
{\tt \_vorder.sas} & Order the var. in a data set\\
\end{tabular}
\end{center}
\end{frame}

\begin{frame}[fragile]
\frametitle{Advanced SAS Macros: Higher-level}
\begin{center}
\begin{tabular}{ll}
Name               & Description \\ \hline
{\tt \_dropobs.sas}& Drop obs. of missing variables\\
{\tt \_dropvar.sas}& Drop var. that's always missing/constant\\
{\tt \_summary.sas}& Flexible tables with the statistics\\
& that you want: correctness vs.\ prettiness \\
\end{tabular}
\end{center}
\end{frame}

\begin{frame}[fragile]
\frametitle{HW part 1: SAS macro for hot-decking and the NTDB}
\begin{itemize}
\item Write a SAS macro to perform hot-decking\\
 for the NTDB example: see the details in lecture 4, slide 4
%\item Hints: use arrays and \textcolor{blue}{\tt set ...\ point}\\
\item Hints: for random number generation from the Uniform
  distribution, use the {\tt rand("unif")} and {\tt call
    streaminit(SEED);} to set the seed.  This seed is 
an argument to the macro, {\tt seed=REQUIRED}.
\item Other arguments are an input data set, {\tt data=REQUIRED},
and an output data set, {\tt out=REQUIRED}.
\item Previous hints from the HW in lecture 13 still apply.
 \item Use arrays with the {\tt array} statement
 \item Use the {\tt set} statement with the {\tt point} option to
   iterate through the data set BEWARE: with {\tt point}, don't forget
   the {\tt \textcolor{red}{stop;} run;} at the end of the DATASTEP
\item With many functions you can use an {\tt of} clause with a variable
list, e.g., \textcolor{blue}{\tt nmiss(of VAR1-VARn)}\\
instead of \textcolor{red}{\tt nmiss(VAR1, ..., VARn)}
\item Or with the STARTS-WITH colon operator
 \textcolor{blue}{\tt nmiss(of VAR:)}
% \item Use a macro variable for the list of variables to
% hot-deck like {\tt \%let var=white black other hispanic male sbp pulse rr;}
% \item To count the number of items in a macro variable, 
% use the {\tt \%\_count()} macro like {\tt \%let k=\%\_count(\&var);}
\item Use the {\tt \%\_nobs()} macro to determine how many
observations are in a data set like {\tt \%\_nobs(data=ntdb.elder)}
\end{itemize}
\end{frame}

\begin{frame}[fragile]
\frametitle{HW part 2: stratified random sampling macro for the NTDB
data set}
\begin{itemize}
\item Write a SAS macro to perform stratified random sampling:
see the details in lecture 4, slide 7
\item Hints: use the {\tt call
    \_permute} subroutine to create permutations. Use {\tt call
    streaminit(SEED);} to set the seed.  This seed is an argument to
  the macro, {\tt seed=REQUIRED}.
\item Other arguments are an input data set, {\tt data=REQUIRED},
an output data set, {\tt out=REQUIRED}, and a variable
list to be kept, {\tt var=REQUIRED}.
\item Previous hints from the HW in lecture 13 still apply.
\item A variable list can be used in many functions with the\\ 
{\tt of} clause like \textcolor{red}{\tt of VAR1-VARn}\\
% for example, 
% {\tt ordinal(m, of VAR1-VARn)} instead of {\tt ordinal(m, VAR1, ..., VARn)}
\end{itemize}
\end{frame}

\end{document}
