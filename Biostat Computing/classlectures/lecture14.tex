\documentclass[11pt,pdftex,dvipsnames,usenames,helvetica]{beamer}
\setbeameroption{show notes}
\usepackage[round]{natbib}
\usepackage{textcomp}
\usepackage{amsmath}
\usepackage{amssymb}
\usepackage{graphicx}
\DeclareGraphicsExtensions{.ps,.eps,.pdf,.jpg,.png}
\usefonttheme[onlymath]{serif}
\usepackage[cmintegrals,cmbraces]{newtxmath}
\usetheme{default}
\usecolortheme{dove}

\usepackage{cancel}
\usepackage{tikz}
\usepackage[english]{babel}

\usepackage{verbatim}
\usepackage{color}
\usepackage{pgf} %portable graphics format
\usepackage[autobold]{statex2}
\mode<presentation>
{
  %\usetheme{Warsaw}
  % or ...
  \setbeamercovered{transparent}
  % or whatever (possibly just delete it)

  \setbeamertemplate{navigation symbols}{}
  \usefonttheme[onlysmall]{structurebold}
  %\usefonttheme{structurebold}
}
\addtobeamertemplate{navigation symbols}{}{%
    \usebeamerfont{footline}%
  \setbeamertemplate{navigation symbols}{}
    \usebeamercolor[fg]{footline}%
    \hspace{1em}%
    \insertframenumber/\inserttotalframenumber
}
\newcommand*{\red}[1]{\textcolor{red}{#1}}
\newcommand*{\blue}[1]{\textcolor{blue}{#1}}

\begin{document}
\boldmath
% 0. Intro

\begin{frame}
\frametitle{Graduate School Class Reminders}

\begin{itemize}
% \item Keep your facial covering on, including over your nose
\item Maintain six feet of distancing
\item Please sit in the same chair each class time
\item Observe entry/exit doors as marked
\item Use hand sanitizer when you enter/exit the classroom
\item Use a disinfectant wipe/spray to wipe down your learning space
  before and after class
\item Media Services: 414 955-4357 option 2
%\item START RECORDING
\end{itemize}

\end{frame}

\begin{frame}
\frametitle{Documentation on the web}

\begin{itemize}
\item CRAN: \url{http://cran.r-project.org}
\item R manuals: \url{https://cran.r-project.org/manuals.html}
\item SAS: \url{http://support.sas.com/documentation}
\item 
\textcolor{blue}{SAS 9.3: \url{https://support.sas.com/en/documentation/documentation-for-SAS-93-and-earlier.html}}
\item Step-by-Step Programming with Base SAS 9.4 (SbS): \\
\url{https://documentation.sas.com/api/docsets/basess/9.4/content/basess.pdf}
\item SAS 9.4 Programmer s Guide: Essentials (PGE): \\
\url{https://documentation.sas.com/api/docsets/lepg/9.4/content/lepg.pdf}
\item Wiki: \url{https://wiki.biostat.mcw.edu} 
\textcolor{red}{(MCW/VPN)}
\end{itemize}

\end{frame}

\begin{frame}[fragile]
\frametitle{ISO 8601 and the Proleptic Gregorian Calendar Again}
\begin{itemize}
\item Creating a SAS format for ISO 8601 dates seemed like a logical
approach
\item However, this created a rather large formats catalog with 2.3M
records and containing 169MB
\item SAS is fairly good at handling large data sets, but this seems
very inefficient
\item So, we are going to create new SAS functions instead
\item Typically, SAS macros have filled this niche
\item However, SAS macros mainly generate SAS code to be run\\
Creating text and performing 
integer arithmetic \textcolor{blue}{which is fast}
\item Recently (circa 2004) SAS added a function compiler with\\
 PROC FCMP so we can write our own DATASTEP functions, 
but \textcolor{red}{they are relatively slow}
\item Now, we have some options to consider
\item N.B. you can create functions in C/C++ and
link to them with {\tt PROC PROTO} (circa 2004): see {\tt protolibs.sas}\\
(but {\tt PROC PROTO} is restricted in v9.4 unlike 9.0 to 9.3)
\end{itemize}
\end{frame}

\begin{frame}[fragile]
\frametitle{SAS macros}
\begin{itemize}
\item You can create macro variables by {\tt \%let VAR=VALUE;}
\item Access their values by {\tt \&VAR} 
which generates {\tt VALUE}\\
quotes are not typically used so if you need quotes\\
\textcolor{blue}{double quotes allow expansion: {\tt "\&VAR"} yields {\tt "VALUE"}}\\
\textcolor{red}{but single quotes don't: {\tt '\&VAR'} yields {\tt '\&VAR'}}
which can be useful if you need to generate a literal with an ampersand
\item You can create array-like macro variables like so\\
{\tt \%let VAR\&i=VALUEi;}\\
{\tt \&\&VAR\&i} yields {\tt VALUEi}\\
 when the macro variable {\tt \&i} is the number {\tt i}
\item Macro variable are often lists\\
{\tt \%let var=dead male black other hispanic sbp;}\\
and you access them all at once by {\tt \&var} or individually by
{\tt \%let var\&i=\%scan(\&var, \&i, \%str( ));}\\
with {\tt \&\&var\&i}
\end{itemize}
\end{frame}

\begin{frame}
\frametitle{SAS macros: {\tt \%if-\%then-\%else}\\ 
\textcolor{red}{RED} and 
\textcolor{blue}{BLUE} lines optional}
{\tt \%if IF-CONDITIONAL-EXPRESSION \%then IF-THEN-BLOCK \\
\textcolor{blue}{\%else \%if ELIF-CONDITIONAL-EXPRESSION-1\\
\%then ELSE-IF-BLOCK-1} \\
$\vdots$ \\
\textcolor{blue}{\%else \%if ELIF-CONDITIONAL-EXPRESSION-M\\
\%then ELSE-IF-BLOCK-M} \\
\textcolor{red}{\%else ELSE-BLOCK}}
\begin{itemize}
\item {\tt BLOCK} can either be a single line followed by a semi-colon
\item or it can be several lines surrounded by {\tt \%do;} and {\tt \%end;}\\
\end{itemize}
{\tt \%do;\\ \qquad LINE1;\\ \qquad $\vdots$\\ \qquad LINEn;\\ \%end;}
\end{frame}

\begin{frame}
\frametitle{SAS macros: macro expressions}
\begin{itemize}
\item Simpler expressions than DATASTEP expressions,\\
but follow somewhat similar rules so it is relatable
\item Numeric: only integer expressions by default\\
the {\tt \%eval()} function performs integer arithmetic\\
{\tt \%eval(\&i+1)} produces 11 if {\tt \&i} is 10\\
instead of {\tt \&i+1} which produces 10+1\\
{\tt \%if \%eval(\&i+1)=11 \%then ...;} is TRUE/executed
\item {\tt \%sysevalf()} for floating point expressions\\
{\tt \%if \%sysevalf(2.5=(age/10), boolean) \%then ...;}
\item Character: often helpful to place these in quotes\\
{\tt \%if "\&\&var\&i"="dead" \%then ...;}
\end{itemize}
\end{frame}

\begin{frame}[fragile]
\frametitle{SAS macros: \%DO loops}
\begin{itemize}
\item The {\tt \%do} statement has other variants besides a block
%\item A block of code {\tt \%do; ...; \%end;}
\item {\tt \%do VAR=VALUE1 \%to VALUE2 \%by VALUE3; ...; \%end;}\\
if {\tt VALUE3} not given, it defaults to 1\\
{\tt \%let var=dead male black other hispanic sbp;}\\
{\tt \%do i=1 \%to \%\_count(\&var);\\ 
\%let var\&i=\%scan(\&var, \&i, \%str( )); \%end;}
\item {\tt \%do \%until(CONDITION); ...; \%end;}\\
  executes until the {\tt CONDITION} is true
\item {\tt \%do \%while(CONDITION); ...; \%end;}\\
  executes until the {\tt CONDITION} is false 
\end{itemize}
\end{frame}

\begin{frame}[fragile]
\frametitle{SAS macros}
\begin{itemize}
\item You can create macros as follows\\
{\tt \%macro NAME; ...; \%mend NAME;}\\
and call it by {\tt \%NAME;}
\item These macros can have arguments as well\\
{\tt \%macro NAME(ARG1, ..., ARGn); ...; \%mend NAME;}\\
and call it by {\tt \%NAME(VALUE1, ..., VALUEn);}\\
\textcolor{blue}{within the macro, the argument values are accessed
as {\tt \&ARGi}}
\item These arguments can have defaults 
\textcolor{red}{which is very convenient}\\
{\tt \%macro NAME(ARG1, ..., ARGn,\\ 
\qquad ARGn+1=VALUE1, ..., ARGn+m=VALUEm);\\
\qquad ...;\\
 \%mend NAME;}\\
and call it by {\tt \%NAME(VALUE1, ..., VALUEn,
ARGn+i=VALUEi, ARGn+j=VALUEj, ...);}\\
\end{itemize}
\end{frame}

\begin{frame}[fragile]
\frametitle{SAS macros}
\begin{itemize}
\item Our SAS installation can be found in the directory
{\tt /usr/local/sas/SAS18w47/SASHome/SASFoundation/9.4}
\item And the SAS macros can be found in the {\tt sasautos}
sub-directory
\item However, there are big gaps in their capabilities
\item I have created a supplementary library called RASmacro\\
(my middle name is Allen) that is GPL free software
\item Most of them are on {\tt github} (or can be soon\\
I haven't uploaded recently): {\tt https://github.com/rsparapa/rasmacro}
\item Most of them start with an underscore so that they are distinct
from what SAS provides
\item On gouda they are installed in {\tt /usr/local/sasmacro}
and the documentation is provided within the files themselves\\
it cannot get separated from the SAS program that way
\end{itemize}
\end{frame}

\begin{frame}[fragile]
\frametitle{RASmacro}
\begin{itemize}
\item We have seen some of these already
\item {\tt \%\_nobs} returns the number of observations in a data set
\item {\tt \%\_list} creates a SAS list with slightly more general rules
than {\tt VAR1-VARn}
\item Let's take a look {\tt \%\_julian} and {\tt \%\_gregory}
\item These provide a mathematical formula to calculate ISO 8601
dates with the SAS origin of 01/01/1960
\end{itemize}
\end{frame}

\begin{frame}[fragile]
\frametitle{PROC FCMP: the function compiler}
\begin{itemize}
\item \textcolor{red}{A SAS macro is impractical for ISO 8601 dates
    since the arguments would have to be integer literals}
\item \textcolor{blue}{We would like to have the flexibility for the
  arguments to be DATASTEP variables}
\item See my example of doing just that by creating the
function {\tt \_julian} with the SAS program {\tt \_julian.sas}
based on the {\tt \%\_julian} RASmacro
\end{itemize}
\end{frame}

\begin{frame}[fragile]
\frametitle{HW 1: create the DATASTEP function {\tt \_gregory}}
\begin{itemize}
\item Hint: base this on the RASmacro {\tt \%\_gregory}
\item Verify that this function is creating ISO 8601
compatible dates by comparing it with the SAS format
that you created for ISO 8601
\item \textcolor{blue}{BEWARE}: user-defined functions are
  comparatively slow which is magnified by the size of the data set
like we have here with \textcolor{red}{2.3M} observations
\end{itemize}
\end{frame}

\begin{frame}[fragile]
\frametitle{HW 2: create the DATASTEP subroutine: {\tt call \_permute}}
\begin{itemize}
\item This subroutine will generate random permutations
since such capability does not appear to exist within 
SAS currently
\item It just needs to take 1 argument: the max
value {\tt M}\\ i.e., create random permutations of 
{\tt 1, \dots, M}
\item Hint: base this on your stratified random sampling code
\item Test to make sure that the seed value works the way it should
\item It is easier if we consider {\tt M} has some maximum
value like 100
\item You can use {\tt M=5} for testing
\item \textcolor{blue}{\tt subroutine \_permute(\_w[*]) varargs; outargs \_w;} 
\item For array initialization of the uniform random variables, 
use the {\tt \_repeat} macro
\end{itemize}
\begin{verbatim}
%let strata=100;
array _u(&strata) u1-u&strata (%_repeat(%str( 1), &strata));
\end{verbatim}
\end{frame}

\end{document}
